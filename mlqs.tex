\documentclass[reqno]{amsart}
\usepackage{setspace,tikz,xcolor,mathrsfs,listings,multicol}
\usepackage{amssymb}
%\usepackage{amsmath}
%\usepackage{amsthm}
\usepackage{rotating}
\usepackage[vcentermath]{youngtab}
%\usepackage{subcaption}
%\usepackage{fullpage}
\usepackage{enumerate}
\usepackage[all,cmtip]{xy}
\usetikzlibrary{arrows,matrix}
\usepackage{comment}
\usepackage{color}
\usepackage{ifthen}
\usepackage[sc]{mathpazo}
\usepackage[T1]{fontenc}
\usepackage{needspace}
\usepackage{tabls}
%\usepackage[margin=1.25in]{geometry}
%\onehalfspacing

\usepackage[colorlinks=true, pdfstartview=FitV, linkcolor=blue, citecolor=blue, urlcolor=blue]{hyperref}

% use these commands for typesetting doi and arXiv references in the bibliography
\newcommand{\doi}[1]{\href{http://dx.doi.org/#1}{\texttt{doi:#1}}}
\newcommand{\arxiv}[1]{\href{http://arxiv.org/abs/#1}{\texttt{arXiv:#1}}}

\newcommand{\iso}{\cong}
%\newcommand{\qedbox}{\rule{2mm}{2mm}}
%\renewcommand{\qedsymbol}{\qedbox}
\newcommand{\qbinom}[3]{\genfrac{[}{]}{0pt}{}{#1}{#2}_{#3}}
\newcommand{\absval}[1]{\left\lvert #1 \right\rvert}
\newcommand{\case}[1]{\vspace{12pt}\noindent\underline{#1}:}
\newcommand{\fs}{\mathcal{S}} % flagged Schur function

\DeclareMathOperator{\supp}{supp} % Support
\DeclareMathOperator{\wt}{wt} % weight
\DeclareMathOperator{\pr}{pr} % promotion
\DeclareMathOperator{\id}{id} % identity
\DeclareMathOperator{\ch}{ch} % character
\DeclareMathOperator{\gr}{gr} % grading

\newcommand{\xx}{\mathbf{x}}
\newcommand{\mm}{\mathbf{m}}
\newcommand{\MLQ}{\mathbf{S}}

\newcommand{\mcA}{\mathcal{A}}
\newcommand{\mcF}{\mathcal{F}}
\newcommand{\mcM}{\mathcal{M}}

\newcommand{\ZZ}{\mathbb{Z}}
\newcommand{\QQ}{\mathbb{Q}}
\newcommand{\RR}{\mathbb{R}}
\newcommand{\CC}{\mathbb{C}}

\newcommand{\bze}{\overline{0}}
\newcommand{\bon}{\overline{1}}
\newcommand{\btw}{\overline{2}}
\newcommand{\bth}{\overline{3}}
\newcommand{\bfo}{\overline{4}}
\newcommand{\bfive}{\overline{5}}
\newcommand{\bsix}{\overline{6}}
\newcommand{\bseven}{\overline{7}}
\newcommand{\beight}{\overline{8}}
\newcommand{\bi}{\overline\imath}
\newcommand{\bk}{\overline{k}}
\newcommand{\brr}{\overline{r}}
\newcommand{\bn}{\overline{n}}
\newcommand{\ellbar}{\overline{\ell}}

\let\sumnonlimits\sum
\let\prodnonlimits\prod
\let\cupnonlimits\bigcup
\let\capnonlimits\bigcap
\renewcommand{\sum}{\sumnonlimits\limits}
\renewcommand{\prod}{\prodnonlimits\limits}
\renewcommand{\bigcup}{\cupnonlimits\limits}
\renewcommand{\bigcap}{\capnonlimits\limits}

\newenvironment{verlong}{}{}
\newenvironment{vershort}{}{}
\newenvironment{noncompile}{}{}
\excludecomment{verlong}
\includecomment{vershort}
\excludecomment{noncompile}
\newcommand{\rev}{\operatorname{rev}}
\newcommand{\conncomp}{\operatorname{conncomp}}
\newcommand{\NN}{\mathbb{N}}
\newcommand{\powset}[2][]{\ifthenelse{\equal{#2}{}}{\mathcal{P}\left(#1\right)}{\mathcal{P}_{#1}\left(#2\right)}}
% $\powset[k]{S}$ stands for the set of all $k$-element subsets of
% $S$. The argument $k$ is optional, and if not provided, the result
% is the whole powerset of $S$.
\newcommand{\set}[1]{\left\{ #1 \right\}}
% $\set{...}$ yields $\left\{ ... \right\}$.
\newcommand{\abs}[1]{\left| #1 \right|}
% $\abs{...}$ yields $\left| ... \right|$.
\newcommand{\tup}[1]{\left( #1 \right)}
% $\tup{...}$ yields $\left( ... \right)$.
\newcommand{\ive}[1]{\left[ #1 \right]}
% $\ive{...}$ yields $\left[ ... \right]$.
\newcommand{\verts}[1]{\operatorname{V}\left( #1 \right)}
% $\verts{...}$ yields $\operatorname{V}\left( ... \right)$.
\newcommand{\edges}[1]{\operatorname{E}\left( #1 \right)}
% $\edges{...}$ yields $\operatorname{E}\left( ... \right)$.
\newcommand{\arcs}[1]{\operatorname{A}\left( #1 \right)}
% $\arcs{...}$ yields $\operatorname{A}\left( ... \right)$.
\newcommand{\underbrack}[2]{\underbrace{#1}_{\substack{#2}}}
% $\underbrack{...1}{...2}$ yields
% $\underbrace{...1}_{\substack{...2}}$. This is useful for doing
% local rewriting transformations on mathematical expressions with
% justifications.

% Dark red emphasis
\definecolor{darkred}{rgb}{0.7,0,0} % darkred color
\newcommand{\defn}[1]{{\color{darkred}\emph{#1}}} % emphasis of a definition


%% For typesetting code listings                                                
\usepackage{listings}
\lstdefinelanguage{Sage}[]{Python}
{morekeywords={False,sage,True},sensitive=true}
\lstset{
  frame=single,
  showtabs=False,
  showspaces=False,
  showstringspaces=False,
  commentstyle={\ttfamily\color{dgreencolor}},
  keywordstyle={\ttfamily\color{dbluecolor}\bfseries},
  stringstyle={\ttfamily\color{dgraycolor}\bfseries},
  language=Sage,
  basicstyle={\footnotesize\ttfamily},
  aboveskip=0.75em,
  belowskip=0.75em,
  xleftmargin=.15in,
}
\definecolor{dblackcolor}{rgb}{0.0,0.0,0.0}
\definecolor{dbluecolor}{rgb}{0.01,0.02,0.7}
\definecolor{dgreencolor}{rgb}{0.2,0.4,0.0}
\definecolor{dgraycolor}{rgb}{0.30,0.3,0.30}
\newcommand{\dblue}{\color{dbluecolor}\bf}
\newcommand{\dred}{\color{dredcolor}\bf}
\newcommand{\dblack}{\color{dblackcolor}\bf}

\usepackage{xparse}

\makeatletter
% \specialmergetwolists{<coupler>}{<list1>}{<list2>}{<return macro>}
% \specialmergetwolists*{<coupler>}{<listcmd1>}{<listcmd2>}{<return macro>}
\protected\def\specialmergetwolists{%
  \begingroup
  \@ifstar{\def\cnta{1}\@specialmergetwolists}
    {\def\cnta{0}\@specialmergetwolists}%
}
\def\@specialmergetwolists#1#2#3#4{%
  \def\tempa##1##2{%
    \edef##2{%
      \ifnum\cnta=\@ne\else\expandafter\@firstoftwo\fi
      \unexpanded\expandafter{##1}%
    }%
  }%
  \tempa{#2}\tempb\tempa{#3}\tempa
  \def\cnta{0}\def#4{}%
  \foreach \x in \tempb{%
    \xdef\cnta{\the\numexpr\cnta+1}%
    \gdef\cntb{0}%
    \foreach \y in \tempa{%
      \xdef\cntb{\the\numexpr\cntb+1}%
      \ifnum\cntb=\cnta\relax
        \xdef#4{#4\ifx#4\empty\else,\fi\x#1\y}%
        \breakforeach
      \fi
    }%
  }%
  \endgroup
}
\makeatother

\theoremstyle{plain}
\newtheorem{thm}{Theorem}[section]
\newtheorem{lemma}[thm]{Lemma}
\newtheorem{conj}[thm]{Conjecture}
\newtheorem{prop}[thm]{Proposition}
\newtheorem{cor}[thm]{Corollary}
\theoremstyle{definition}
\newtheorem{dfn}[thm]{Definition}
\newtheorem{ex}[thm]{Example}
\newtheorem{remark}[thm]{Remark}
\numberwithin{equation}{section}
%\numberwithin{figure}{section}
%\numberwithin{table}{section}
%\setcounter{section}{-1}

% For breaking equations across multiple pages
% \allowdisplaybreaks[1]

\usepackage[colorinlistoftodos]{todonotes}
\newcommand{\darij}[1]{\todo[size=\tiny,color=red!30]{#1 \\ \hfill --- Darij}}
\newcommand{\Darij}[1]{\todo[size=\tiny,inline,color=red!30]{#1
      \\ \hfill --- Darij}}
\newcommand{\travis}[1]{\todo[size=\tiny,color=blue!30]{#1 \\ \hfill --- Travis}}
\newcommand{\Travis}[1]{\todo[size=\tiny,inline,color=blue!30]{#1
      \\ \hfill --- Travis}}



%%%%%%%%%%%%%%%%%%%%%%%%%%%%%%%%%%%%%%%%

\begin{document}
\title[MLQs]{Multiline queues with special parameters}

\author[D.~Grinberg]{Darij Grinberg}
\address[D. Grinberg]{School of Mathematics, University of Minnesota, 206 Church St. SE, Minneapolis, MN 55455}
\email{darij.grinberg@gmail.com}
\urladdr{http://www.cip.ifi.lmu.de/~grinberg/}

\author[T.~Scrimshaw]{Travis Scrimshaw}
%\address[T. Scrimshaw]{School of Mathematics, University of Minnesota, 206 Church St. SE, Minneapolis, MN 55455}
\email{tcscrims@gmail.com}
\urladdr{https://sites.google.com/view/tscrim/home}

\keywords{multiline queue, flagged Schur function}
\subjclass[2010]{05E10, 17B37}

%\thanks{TS was partially supported by the National Science Foundation RTG grant NSF/DMS-1148634.}

\begin{abstract}
We prove the spectral parameter version of the MLQs conjecture.
\end{abstract}

\maketitle



%=====================================================================
\section{Introduction}
\label{sec:introduction}

To be written last.






%=====================================================================
\section{Background}
\label{sec:background}

\subsection{Multiline queues}

Fix a positive integer $N$.
Let $\ive{N}$ denote the set $\set{1, 2, \ldots, N}$.

\darij{Remember to fix the spelling of ``labelling''.}

\begin{dfn}
A \defn{labeling} of a set $X$ means a map
$X \to \set{1, 2, 3, \ldots}$.
If $f$ is a labeling of $X$, then the image of
a given element $x \in X$ under $f$ is called the
\defn{label} of $x$ (in $f$).
\end{dfn}

\begin{dfn}
If $p$ is an element of $\ive{N}$,
and if $S$ is a nonempty subset of $\ive{N}$,
then an element \defn{$\min_p S$} of $S$ is defined as
follows:
If there exists an element of $S$ that is greater
or equal to $p$, then we set $\min_p S$ to be the
smallest such element;
otherwise, we set $\min_p S$ to be the smallest
element of $S$.

(Roughly speaking, $\min_p S$ is the first element
of $S$ you encounter when traversing the elements
$p, p+1, \ldots, n, 1, 2, \ldots, p-1$.)
\end{dfn}

\begin{dfn} \label{def.mlqs.down}
Let $U$ and $V$ be two subsets of $\ive{N}$ satisfying
$\abs{U} \leq \abs{V}$.
Let $f$ be a labeling of $U$.
Let $k$ be a positive integer.

Then, a labeling \defn{$f \downarrow_{V, k}$} of $V$ is defined
as follows (recursively over $\abs{U}$):

\begin{itemize}
 \item If $U = \varnothing$, then $f \downarrow_{V, k}$ sends
       each element of $V$ to $k$.
 \item Otherwise, pick an element $u$ of $U$ having minimum
       label (i.e., having minimum $f \tup{u}$).
       Let $v = \min_u V$.
       Let $g$ be the restriction of $f$ to the subset
       $U \setminus \set{u}$.
       Let $g' = g \downarrow_{V \setminus \set{v}, k}$.
       Then, the labeling $f \downarrow_{V, k}$ is defined by
       setting
       \[
        \tup{f \downarrow_{V, k}} \tup{s}
        =
        \begin{cases}
         f \tup{u}, & \text{ if } s = u; \\
         g \tup{s}, & \text{ if } s \neq u
        \end{cases}
        \qquad \text{ for all } s \in V .
       \]
       Note that this does not depend on the choice of $u$
       (according to Proposition~\ref{prop.mlqs.down.indep}).
\end{itemize}

\end{dfn}

\begin{prop} \label{prop.mlqs.down.indep}
The definition of $f \downarrow_{V, k}$ given above does not
depend on the choice of $u$.
\end{prop}

\begin{proof}[Proof of Proposition~\ref{prop.mlqs.down.indep}.]
Let $U, V, f, k$ be as in Definition~\ref{def.mlqs.down}.
Let $u_1$ and $u_2$ be two elements of $U$ having minimum
label (in $f$).
We must prove that the labeling
$f \downarrow_{V, k}$ obtained by taking $u = u_1$ in
Definition~\ref{def.mlqs.down}
is identical with the labeling
$f \downarrow_{V, k}$ obtained by taking $u = u_2$ in
Definition~\ref{def.mlqs.down}.

......
\end{proof}

Next, we shall define the notion of an
\textit{MLQ} (short for \textit{multiline queue}).
This notion corresponds to what is called a
``discrete MLQ'' in \cite[\S 2.2]{AasLin17}; we omit the
word ``discrete'', as we will not encounter any other kind
of MLQs.

\begin{dfn} \label{def.mlqs.mlq}
Fix a nonnegative integer $n$,
and an $n$-tuple $\mm = \tup{m_1, m_2, \ldots, m_n}$ of nonnegative
integers.
An \defn{MLQ} of type $\mm$ is
an $\tup{n+1}$-tuple $\tup{S_0, S_1, \ldots, S_n}$ of subsets of
$\ive{N}$
such that each $i \in \set{0, 1, \ldots, n}$ satisfies
$\abs{S_i} = m_1 + m_2 + \cdots + m_i$.
(In particular, $S_0 = \varnothing$.)

We shall represent an MLQ $\tup{S_0, S_1, \ldots, S_n}$ as an
$n \times N$-array, each row of which has some of its boxes
marked. Namely, for each $i \in \set{1, 2, \ldots, n}$ and
each $j \in S_i$, we mark the $j$-th box in row $i$.
(The rows and the columns are indexed as usual for a matrix.)
\end{dfn}

\begin{dfn} \label{def.mlqs.can-lab}
Let $n$ and $\mm$ be as in Definition~\ref{def.mlqs.mlq}.
Let $\MLQ = \tup{S_0, S_1, \ldots, S_n}$ be an MLQ of type $\mm$.

Then, for each $i \in \set{0, 1, \ldots, n}$, we can
define a labeling $f_i$ of the set $S_i$ as follows (by
recursion over $i$):
\begin{itemize}
 \item The labeling $f_0$ on $S_0$ is defined in the
       obvious way (since $S_0 = \varnothing$).
 \item If $f_{i-1}$ is defined for some $i \geq 1$,
       then we define $f_i$ by
       $f_i = f_{i-1} \downarrow_{S_i, i}$.
\end{itemize}
These labelings $f_0, f_1, \ldots, f_n$ are called the
\defn{canonical labelings} of the MLQ $\MLQ$.

The pair $\tup{S_n, f_n}$ is called the
\defn{bottom row} of the MLQ $\MLQ$.
\end{dfn}

\begin{dfn} \label{def.mlq.sw}
Let $n$ and $\mm$ be as in Definition~\ref{def.mlqs.mlq}.
Let $\MLQ = \tup{S_0, S_1, \ldots, S_n}$ be an MLQ of type $\mm$.

The \defn{spectral weight} $\wt\tup{\MLQ}$
of $\MLQ$ is defined to be the
monomial
\[
 \prod_{i=0}^{n-1} \prod_{s \in S_i} x_s
\]
in the indeterminates $x_1, x_2, \ldots, x_N$.
\end{dfn}


\subsection{Flagged Schur functions}

Let $\lambda$ be a partition of length $m$ and $b = (b_1 \leq b_2 \leq \cdots \leq b_m)$.
Let
\[
h_d(k) = \sum_{i_1 \leq \cdots \leq i_d \leq k} x_{i_1} \cdots x_{i_d}
\]
be the complete homogeneous symmetric function of degree $d$ in the variables $x_1, \dotsc, x_k$.
A \defn{flagged Schur function} is
\[
\fs_{\lambda}(b; \xx) = \det\big[ h_{\lambda_i - i + j}(b_i) \bigr]_{i,j=1}^m.
\]
We can also express the flagged Schur function combinatorially using $\mcF_{\lambda}(b)$, the semistandard tableaux of shape $\lambda$ such that the max entry in row $i$ is at most $b_i$.
From~\cite{Wachs85}, we have
\[
\fs_{\lambda}(b; \xx) = \sum_{T \in \mcF_{\lambda}(b)} x^T.
\]








%=====================================================================
\section{Main result}
\label{sec:result}


Let
\[
\mcM_k(b; \xx) = \sum_M \wt(M),
\]
where we sum over all MLQs with final positions $b$ and shape $k$.

\begin{thm} \label{thm.decreasing}
Fix a nonnegative integer $n$,
and an $n$-tuple $\mm = \tup{m_1, m_2, \ldots, m_n}$ of positive
integers.

Fix a subset $S = \set{b_1 < b_2 < \cdots < b_n}$ of $\ive{N}$.
% Let $b$ be the sequence $\tup{b_1, b_2, \ldots, b_n}$.
Let $f : S \to \set{1, 2, \ldots, n}$ be the unique
map that is weakly decreasing on $S$ and satisfies
$\abs{f^{-1} \tup{i}} = m_i$ for all $i$.

Let $\mcM \tup{S, f}$ be the set of all MLQs of type $\mm$
with bottom row $\tup{S, f}$.

We have
\[
\sum_{\MLQ \in \mcM \tup{S, f}} \wt\tup{\MLQ}
= \det \tup{
             \tup{ h_{\gamma_i - i + j} \tup{x_1, x_2, \ldots, x_{b_i}}
                 }_{1 \leq i \leq n, \  1 \leq j \leq n}
           },
\]
where $\gamma_i$ is the number of all
$k \in \set{1, 2, \ldots, n-1}$ satisfying
$m_{k + 1} + m_{k + 2} + \cdots + m_n < i$.
\end{thm}

\begin{proof}
For ordered $b$, the proof that there is no wrapping is the same as in~\cite{AasLin17} when all entries of $b$ are distinct.
Hence the only possibility for lattice paths to intersect is when the two paths have the same color.

If two paths have the same color and intersect, separate them into a non-intersecting lattice path by shifting up 1 step the entire path and all other paths above it.

Thus, by the Lindstr\"om--Gessel--Viennot Lemma, we have
\[
\det\big[ h_{j-k_{n+1-i}(S)'}(b_{n+1-i}) \bigr]_{i,j=1}^n.
\]
Thus the claim follows.
\end{proof}

\Travis{This theorem is not the correct statement.}

\begin{comment}
 We hope to have
\[
\sum_{\MLQ \in \mcM \tup{S, f}} \wt\tup{\MLQ}
= \fs_{\lambda}(b; \xx),
\]
where $\lambda$ is the partition $\tup{1^{m_1}, 2^{m_2}, 3^{m_3}, \ldots}$
(in multiplicity notation), or at least some partition.
\end{comment}


\begin{thm}
Consider the action of the Young subgroup $S_{\mu}$.
Then, we have
\[
\mcM_{\mu}(b; \xx) = \mcM_k(b; \xx).
\]
\end{thm}

\begin{proof}
The only wrapping the can occur is within the action of a particular Young subgroup.
This is similar to the proof that there is no wrapping in~\cite{AasLin17}.
\end{proof}







\bibliographystyle{alpha}
\bibliography{queue}{}
\end{document}
