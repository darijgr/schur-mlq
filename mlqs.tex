\documentclass[reqno]{amsart}
\usepackage{setspace,tikz,xcolor,mathrsfs,listings,multicol}
\usepackage{amssymb}
%\usepackage{amsmath}
%\usepackage{amsthm}
\usepackage{rotating}
\usepackage[vcentermath]{youngtab}
%\usepackage{subcaption}
%\usepackage{fullpage}
\usepackage{enumerate}
\usepackage[all,cmtip]{xy}
\usetikzlibrary{arrows,matrix}
\usepackage{comment}
\usepackage{color}
\usepackage{ifthen}
\usepackage[sc]{mathpazo}
\usepackage[T1]{fontenc}
\usepackage{needspace}
\usepackage{tabls}
%\usepackage[margin=1.25in]{geometry}
%\onehalfspacing

\usepackage[colorlinks=true, pdfstartview=FitV, linkcolor=blue, citecolor=blue, urlcolor=blue]{hyperref}

% use these commands for typesetting doi and arXiv references in the bibliography
\newcommand{\doi}[1]{\href{http://dx.doi.org/#1}{\texttt{doi:#1}}}
\newcommand{\arxiv}[1]{\href{http://arxiv.org/abs/#1}{\texttt{arXiv:#1}}}

\newcommand{\iso}{\cong}
%\newcommand{\qedbox}{\rule{2mm}{2mm}}
%\renewcommand{\qedsymbol}{\qedbox}
\newcommand{\qbinom}[3]{\genfrac{[}{]}{0pt}{}{#1}{#2}_{#3}}
\newcommand{\absval}[1]{\left\lvert #1 \right\rvert}
\newcommand{\case}[1]{\vspace{12pt}\noindent\underline{#1}:}
\newcommand{\fs}{\mathcal{S}} % flagged Schur function
\newcommand{\mbf}{\mathbf}

\DeclareMathOperator{\supp}{supp} % Support
\DeclareMathOperator{\wt}{wt} % weight
\DeclareMathOperator{\pr}{pr} % promotion
\DeclareMathOperator{\id}{id} % identity
\DeclareMathOperator{\ch}{ch} % character
\DeclareMathOperator{\gr}{gr} % grading

\newcommand{\xx}{\mathbf{x}}
\newcommand{\mm}{\mathbf{m}}
\newcommand{\MLQ}{\mathbf{S}}

\newcommand{\mcA}{\mathcal{A}}
\newcommand{\mcF}{\mathcal{F}}
\newcommand{\mcM}{\mathcal{M}}

\newcommand{\ZZ}{\mathbb{Z}}
\newcommand{\QQ}{\mathbb{Q}}
\newcommand{\RR}{\mathbb{R}}
\newcommand{\CC}{\mathbb{C}}

\newcommand{\bze}{\overline{0}}
\newcommand{\bon}{\overline{1}}
\newcommand{\btw}{\overline{2}}
\newcommand{\bth}{\overline{3}}
\newcommand{\bfo}{\overline{4}}
\newcommand{\bfive}{\overline{5}}
\newcommand{\bsix}{\overline{6}}
\newcommand{\bseven}{\overline{7}}
\newcommand{\beight}{\overline{8}}
\newcommand{\bi}{\overline\imath}
\newcommand{\bk}{\overline{k}}
\newcommand{\brr}{\overline{r}}
\newcommand{\bn}{\overline{n}}
\newcommand{\ellbar}{\overline{\ell}}

\let\sumnonlimits\sum
\let\prodnonlimits\prod
\let\cupnonlimits\bigcup
\let\capnonlimits\bigcap
\renewcommand{\sum}{\sumnonlimits\limits}
\renewcommand{\prod}{\prodnonlimits\limits}
\renewcommand{\bigcup}{\cupnonlimits\limits}
\renewcommand{\bigcap}{\capnonlimits\limits}

\newenvironment{verlong}{}{}
\newenvironment{vershort}{}{}
\newenvironment{noncompile}{}{}
\excludecomment{verlong}
\includecomment{vershort}
\excludecomment{noncompile}
\newcommand{\rev}{\operatorname{rev}}
\newcommand{\conncomp}{\operatorname{conncomp}}
\newcommand{\NN}{\mathbb{N}}
\newcommand{\powset}[2][]{\ifthenelse{\equal{#2}{}}{\mathcal{P}\left(#1\right)}{\mathcal{P}_{#1}\left(#2\right)}}
% $\powset[k]{S}$ stands for the set of all $k$-element subsets of
% $S$. The argument $k$ is optional, and if not provided, the result
% is the whole powerset of $S$.
\newcommand{\set}[1]{\left\{ #1 \right\}}
% $\set{...}$ yields $\left\{ ... \right\}$.
\newcommand{\abs}[1]{\left| #1 \right|}
% $\abs{...}$ yields $\left| ... \right|$.
\newcommand{\tup}[1]{\left( #1 \right)}
% $\tup{...}$ yields $\left( ... \right)$.
\newcommand{\ive}[1]{\left[ #1 \right]}
% $\ive{...}$ yields $\left[ ... \right]$.
\newcommand{\verts}[1]{\operatorname{V}\left( #1 \right)}
% $\verts{...}$ yields $\operatorname{V}\left( ... \right)$.
\newcommand{\edges}[1]{\operatorname{E}\left( #1 \right)}
% $\edges{...}$ yields $\operatorname{E}\left( ... \right)$.
\newcommand{\arcs}[1]{\operatorname{A}\left( #1 \right)}
% $\arcs{...}$ yields $\operatorname{A}\left( ... \right)$.
\newcommand{\underbrack}[2]{\underbrace{#1}_{\substack{#2}}}
% $\underbrack{...1}{...2}$ yields
% $\underbrace{...1}_{\substack{...2}}$. This is useful for doing
% local rewriting transformations on mathematical expressions with
% justifications.

% Dark red emphasis
\definecolor{darkred}{rgb}{0.7,0,0} % darkred color
\newcommand{\defn}[1]{{\color{darkred}\emph{#1}}} % emphasis of a definition


%% For typesetting code listings                                                
\usepackage{listings}
\lstdefinelanguage{Sage}[]{Python}
{morekeywords={False,sage,True},sensitive=true}
\lstset{
  frame=single,
  showtabs=False,
  showspaces=False,
  showstringspaces=False,
  commentstyle={\ttfamily\color{dgreencolor}},
  keywordstyle={\ttfamily\color{dbluecolor}\bfseries},
  stringstyle={\ttfamily\color{dgraycolor}\bfseries},
  language=Sage,
  basicstyle={\footnotesize\ttfamily},
  aboveskip=0.75em,
  belowskip=0.75em,
  xleftmargin=.15in,
}
\definecolor{dblackcolor}{rgb}{0.0,0.0,0.0}
\definecolor{dbluecolor}{rgb}{0.01,0.02,0.7}
\definecolor{dgreencolor}{rgb}{0.2,0.4,0.0}
\definecolor{dgraycolor}{rgb}{0.30,0.3,0.30}
\newcommand{\dblue}{\color{dbluecolor}\bf}
\newcommand{\dred}{\color{dredcolor}\bf}
\newcommand{\dblack}{\color{dblackcolor}\bf}

\usepackage{xparse}

\makeatletter
% \specialmergetwolists{<coupler>}{<list1>}{<list2>}{<return macro>}
% \specialmergetwolists*{<coupler>}{<listcmd1>}{<listcmd2>}{<return macro>}
\protected\def\specialmergetwolists{%
  \begingroup
  \@ifstar{\def\cnta{1}\@specialmergetwolists}
    {\def\cnta{0}\@specialmergetwolists}%
}
\def\@specialmergetwolists#1#2#3#4{%
  \def\tempa##1##2{%
    \edef##2{%
      \ifnum\cnta=\@ne\else\expandafter\@firstoftwo\fi
      \unexpanded\expandafter{##1}%
    }%
  }%
  \tempa{#2}\tempb\tempa{#3}\tempa
  \def\cnta{0}\def#4{}%
  \foreach \x in \tempb{%
    \xdef\cnta{\the\numexpr\cnta+1}%
    \gdef\cntb{0}%
    \foreach \y in \tempa{%
      \xdef\cntb{\the\numexpr\cntb+1}%
      \ifnum\cntb=\cnta\relax
        \xdef#4{#4\ifx#4\empty\else,\fi\x#1\y}%
        \breakforeach
      \fi
    }%
  }%
  \endgroup
}
\makeatother

\theoremstyle{plain}
\newtheorem{thm}{Theorem}[section]
\newtheorem{lemma}[thm]{Lemma}
\newtheorem{conj}[thm]{Conjecture}
\newtheorem{prop}[thm]{Proposition}
\newtheorem{cor}[thm]{Corollary}
\theoremstyle{definition}
\newtheorem{dfn}[thm]{Definition}
\newtheorem{ex}[thm]{Example}
\newtheorem{remark}[thm]{Remark}
\numberwithin{equation}{section}
%\numberwithin{figure}{section}
%\numberwithin{table}{section}
%\setcounter{section}{-1}

% For breaking equations across multiple pages
% \allowdisplaybreaks[1]

\usepackage[colorinlistoftodos]{todonotes}
\newcommand{\darij}[1]{\todo[size=\tiny,color=red!30]{#1 \\ \hfill --- Darij}}
\newcommand{\Darij}[1]{\todo[size=\tiny,inline,color=red!30]{#1
      \\ \hfill --- Darij}}
\newcommand{\travis}[1]{\todo[size=\tiny,color=blue!30]{#1 \\ \hfill --- Travis}}
\newcommand{\Travis}[1]{\todo[size=\tiny,inline,color=blue!30]{#1
      \\ \hfill --- Travis}}



%%%%%%%%%%%%%%%%%%%%%%%%%%%%%%%%%%%%%%%%

\begin{document}
\title[MLQs]{Multiline queues with special parameters}

\author[E.~Aas]{Erik Aas}
\address[E. Aas]{Department of Mathematics, Pennsylvania State University, ??, ??}
\email{eaas@kth.se}

\author[D.~Grinberg]{Darij Grinberg}
\address[D. Grinberg]{School of Mathematics, University of Minnesota, 206 Church St. SE, Minneapolis, MN 55455}
\email{darij.grinberg@gmail.com}
\urladdr{http://www.cip.ifi.lmu.de/~grinberg/}

\author[T.~Scrimshaw]{Travis Scrimshaw}
%\address[T. Scrimshaw]{School of Mathematics, University of Minnesota, 206 Church St. SE, Minneapolis, MN 55455}
\email{tcscrims@gmail.com}
\urladdr{https://sites.google.com/view/tscrim/home}

\keywords{multiline queue, flagged Schur function}
\subjclass[2010]{05E10, 17B37}

%\thanks{TS was partially supported by the National Science Foundation RTG grant NSF/DMS-1148634.}

\begin{abstract}
We prove the spectral parameter version of the MLQs conjecture.
\end{abstract}

\maketitle



%=====================================================================
\section{Introduction}
\label{sec:introduction}

To be written last.






%=====================================================================
\section{Background}
\label{sec:background}

\subsection{Multiline queues}

Fix a positive integer $N$.
Let $\ive{N}$ denote the set $\set{1, 2, \ldots, N}$.

\darij{Remember to fix the spelling of ``labelling''.}

\begin{dfn}
A \defn{labeling} of a set $X$ means a map
$X \to \set{1, 2, 3, \ldots}$.
If $f$ is a labeling of $X$, then the image of
a given element $x \in X$ under $f$ is called the
\defn{label} of $x$ (in $f$).
\end{dfn}

\begin{dfn}
If $p$ is an element of $\ive{N}$,
and if $S$ is a nonempty subset of $\ive{N}$,
then an element \defn{$\min_p S$} of $S$ is defined as
follows:
If there exists an element of $S$ that is greater
or equal to $p$, then we set $\min_p S$ to be the
smallest such element;
otherwise, we set $\min_p S$ to be the smallest
element of $S$.

(Roughly speaking, $\min_p S$ is the first element
of $S$ you encounter when traversing the elements
$p, p+1, \ldots, n, 1, 2, \ldots, p-1$.)
\end{dfn}

\begin{dfn} \label{def.mlqs.down}
Let $U$ and $V$ be two subsets of $\ive{N}$ satisfying
$\abs{U} \leq \abs{V}$.
Let $f$ be a labeling of $U$.
Let $k$ be a positive integer.

Then, a labeling \defn{$f \downarrow_{V, k}$} of $V$ is defined
as follows (recursively over $\abs{U}$):

\begin{itemize}
 \item If $U = \varnothing$, then $f \downarrow_{V, k}$ sends
       each element of $V$ to $k$.
 \item Otherwise, pick an element $u$ of $U$ having minimum
       label (i.e., having minimum $f \tup{u}$).
       Let $v = \min_u V$.
       Let $g$ be the restriction of $f$ to the subset
       $U \setminus \set{u}$.
       Let $g' = g \downarrow_{V \setminus \set{v}, k}$.
       Then, the labeling $f \downarrow_{V, k}$ is defined by
       setting
       \[
        \tup{f \downarrow_{V, k}} \tup{s}
        =
        \begin{cases}
         f \tup{u}, & \text{ if } s = u; \\
         g \tup{s}, & \text{ if } s \neq u
        \end{cases}
        \qquad \text{ for all } s \in V .
       \]
       Note that this does not depend on the choice of $u$
       (according to Proposition~\ref{prop.mlqs.down.indep}).
\end{itemize}

\end{dfn}

\begin{prop} \label{prop.mlqs.down.indep}
The definition of $f \downarrow_{V, k}$ given above does not
depend on the choice of $u$.
\end{prop}

\begin{proof}[Proof of Proposition~\ref{prop.mlqs.down.indep}.]
Let $U, V, f, k$ be as in Definition~\ref{def.mlqs.down}.
Let $u_1$ and $u_2$ be two elements of $U$ having minimum
label (in $f$).
We must prove that the labeling
$f \downarrow_{V, k}$ obtained by taking $u = u_1$ in
Definition~\ref{def.mlqs.down}
is identical with the labeling
$f \downarrow_{V, k}$ obtained by taking $u = u_2$ in
Definition~\ref{def.mlqs.down}.

......
\end{proof}

Next, we shall define the notion of an
\textit{MLQ} (short for \textit{multiline queue}).
This notion corresponds to what is called a
``discrete MLQ'' in \cite[\S 2.2]{AasLin17}; we omit the
word ``discrete'', as we will not encounter any other kind
of MLQs.

\begin{dfn} \label{def.mlqs.mlq}
Fix a nonnegative integer $n$,
and an $n$-tuple $\mm = \tup{m_1, m_2, \ldots, m_n}$ of nonnegative
integers.
An \defn{MLQ} of type $\mm$ is
an $\tup{n+1}$-tuple $\tup{S_0, S_1, \ldots, S_n}$ of subsets of
$\ive{N}$
such that each $i \in \set{0, 1, \ldots, n}$ satisfies
$\abs{S_i} = m_1 + m_2 + \cdots + m_i$.
(In particular, $S_0 = \varnothing$.)

We shall represent an MLQ $\tup{S_0, S_1, \ldots, S_n}$ as an
$n \times N$-array, each row of which has some of its boxes
marked. Namely, for each $i \in \set{1, 2, \ldots, n}$ and
each $j \in S_i$, we mark the $j$-th box in row $i$.
(The rows and the columns are indexed as usual for a matrix.)
\end{dfn}

\begin{dfn} \label{def.mlqs.can-lab}
Let $n$ and $\mm$ be as in Definition~\ref{def.mlqs.mlq}.
Let $\MLQ = \tup{S_0, S_1, \ldots, S_n}$ be an MLQ of type $\mm$.

Then, for each $i \in \set{0, 1, \ldots, n}$, we can
define a labeling $f_i$ of the set $S_i$ as follows (by
recursion over $i$):
\begin{itemize}
 \item The labeling $f_0$ on $S_0$ is defined in the
       obvious way (since $S_0 = \varnothing$).
 \item If $f_{i-1}$ is defined for some $i \geq 1$,
       then we define $f_i$ by
       $f_i = f_{i-1} \downarrow_{S_i, i}$.
\end{itemize}
These labelings $f_0, f_1, \ldots, f_n$ are called the
\defn{canonical labelings} of the MLQ $\MLQ$.

The pair $\tup{S_n, f_n}$ is called the
\defn{bottom row} of the MLQ $\MLQ$.
\end{dfn}

\begin{dfn} \label{def.mlq.sw}
Let $n$ and $\mm$ be as in Definition~\ref{def.mlqs.mlq}.
Let $\MLQ = \tup{S_0, S_1, \ldots, S_n}$ be an MLQ of type $\mm$.

The \defn{spectral weight} $\wt\tup{\MLQ}$
of $\MLQ$ is defined to be the
monomial
\[
 \prod_{i=0}^{n-1} \prod_{s \in S_i} x_s
\]
in the indeterminates $x_1, x_2, \ldots, x_N$.
\end{dfn}


\subsection{Flagged Schur functions}

Let $\lambda$ be a partition of length $m$ and $b = (b_1 \leq b_2 \leq \cdots \leq b_m)$.
Let
\[
h_d(k) = \sum_{i_1 \leq \cdots \leq i_d \leq k} x_{i_1} \cdots x_{i_d}
\]
be the complete homogeneous symmetric function of degree $d$ in the variables $x_1, \dotsc, x_k$.
A \defn{flagged Schur function} is
\[
\fs_{\lambda}(b; \xx) = \det\big[ h_{\lambda_i - i + j}(b_i) \bigr]_{i,j=1}^m.
\]
We can also express the flagged Schur function combinatorially using $\mcF_{\lambda}(b)$, the semistandard tableaux of shape $\lambda$ such that the max entry in row $i$ is at most $b_i$.
From~\cite{Wachs85}, we have
\[
\fs_{\lambda}(b; \xx) = \sum_{T \in \mcF_{\lambda}(b)} x^T.
\]








%=====================================================================
\section{Main result}
\label{sec:result}


Let
\[
  \mcM_k(b; \xx) = \sum_M \wt(M),
\]
where we sum over all MLQs with final positions $b$ and shape $k$.

\begin{thm} \label{thm.decreasing}
Fix a nonnegative integer $n$,
and an $n$-tuple $\mm = \tup{m_1, m_2, \ldots, m_n}$ of positive
integers.

Fix a subset $S = \set{b_1 < b_2 < \cdots < b_n}$ of $\ive{N}$.
% Let $b$ be the sequence $\tup{b_1, b_2, \ldots, b_n}$.
Let $f : S \to \set{1, 2, \ldots, n}$ be the unique
map that is weakly decreasing on $S$ and satisfies
$\abs{f^{-1} \tup{i}} = m_i$ for all $i$.

Let $\mcM \tup{S, f}$ be the set of all MLQs of type $\mm$
with bottom row $\tup{S, f}$.

We have
\[
\sum_{\MLQ \in \mcM \tup{S, f}} \wt\tup{\MLQ}
= \det \tup{
             \tup{ h_{\gamma_i - i + j} \tup{x_1, x_2, \ldots, x_{b_i}}
                 }_{1 \leq i \leq n, \  1 \leq j \leq n}
           },
\]
where $\gamma_i$ is the number of all
$k \in \set{1, 2, \ldots, n-1}$ satisfying
$m_{k + 1} + m_{k + 2} + \cdots + m_n < i$.
\end{thm}

\begin{proof}
For ordered $b$, the proof that there is no wrapping is the same as in~\cite{AasLin17} when all entries of $b$ are distinct.
Hence the only possibility for lattice paths to intersect is when the two paths have the same color.

If two paths have the same color and intersect, separate them into a non-intersecting lattice path by shifting up 1 step the entire path and all other paths above it.

Thus, by the Lindstr\"om--Gessel--Viennot Lemma, we have
\[
\det\big[ h_{j-k_{n+1-i}(S)'}(b_{n+1-i}) \bigr]_{i,j=1}^n.
\]
Thus the claim follows.
\end{proof}

\begin{comment}
 We hope to have
\[
\sum_{\MLQ \in \mcM \tup{S, f}} \wt\tup{\MLQ}
= \fs_{\lambda}(b; \xx),
\]
where $\lambda$ is the partition $\tup{1^{m_1}, 2^{m_2}, 3^{m_3}, \ldots}$
(in multiplicity notation), or at least some partition.

But this doesn't seem to work on the nose, since either
the partition increases or the flags decrease.

Maybe ghost entries, so a more general notion of flagged
Schur functions.
\end{comment}


\begin{thm}
Consider the action of the Young subgroup $S_{\mu}$.
Then, we have
\[
\mcM_{\mu}(b; \xx) = \mcM_k(b; \xx).
\]
\end{thm}

\Travis{This theorem is not the correct statement.}

\begin{proof}
The only wrapping the can occur is within the action of a particular Young subgroup.
This is similar to the proof that there is no wrapping in~\cite{AasLin17}.
\end{proof}

\section{The projection formula}

For the homogenous case, where all spectral parameters equal, it is well known that the number of multiline queues with a given bottom row is proportional to the stationary distribution of a certain Markov chain, the multi-species TASEP on a ring, at the state corresponding to that bottom row. The TASEP has a property of merging classes that translates to a statement about multiline queues, usually called the {\it projection principle}. Somewhat surprisingly, this principle has a generalisation to our case with non-equal spectral parameters. We now state this generalisation.

\begin{thm}
  Let $u$ be a word of type $\mbf{m}$, and let $v$ be the word obtained from $u$ by merging classes $k$ and $k+1$.
  Then $\sum_{u'} \wt{u} = e_{m_1+\dots+m_k}(x) \wt{v}$,
  where $u'$ runs over all words gotten by permuting $k$ and $k+1$ in $u$.
\end{thm}

This theorem will be a corollary of Theorem [??].


\noindent
{\bf Example.}
Looking at queues, we get \\
\noindent

$[13234] = x_1x_2x_3^2x_4(x_1^2+x_1x_4+x_1x_5+x_4x_5+x_5^2)$

$[13245] = x_1x_2x_3^2x_4(x_1^2+x_1x_4+x_1x_5+x_4^2+x_4x_5+x_5^2)(x_1x_2x_3+x_1x_2x_5+x_1x_3x_5+x_2x_3x_5)$.

$[14235] = x_1x_2x_3^2x_4^2(x_1^3x_2+x_1^3x_3+x_1^3x_5+x_1^2x_2x_3+x_1^2x_2x_4+2x_1^2x_2x_5+x_1^2x_3x_4+2x_1^2x_3x_5+x_1^2x_4x_5+x_1^2x_5^2+x_1x_2x_3x_5+x_1x_2x_4x_5+2x_1x_2x_5^2+x_1x_3x_4x_5+2x_1x_3x_5^2+x_1x_4x_5^2+x_1x_5^3+x_2x_3x_5^2+x_2x_4x_5^2+x_2x_5^3+x_3x_4x_5^2+x_3x_5^3)
$

(We have factored the expressions for readability only.)

The projection formula states that $[13234] e_3(x) = [13245] + [14235]$, as we can check directly in this case.


\section{Generalized queues}

\begin{figure}
\label{fig_generalized_queue}
\begin{tikzpicture}
  \node at  (1,2){1};
  \node at  (2,2){3};
  \node at  (3,2){2};
  \node at  (4,2){3};
  \node at  (5,2){3};
  \node at  (6,2){2};
  \node at  (7,2){2};
  \node at  (8,2){3};
  \node at  (9,2){1};
  \node at (10,2){2};
  \node at (11,2){1};

  \node at                     (1,1){4};
  \node[circle,draw=black] at  (2,1){1};
  \node at                     (3,1){4};
  \node[circle,draw=black] at  (4,1){1};
  \node[circle,draw=black] at  (5,1){2};
  \node at                     (6,1){3};
  \node at                     (7,1){4};
  \node[circle,draw=black] at  (8,1){2};
  \node at                     (9,1){3};
  \node[circle,draw=black] at (10,1){1};
  \node at                    (11,1){4};
\end{tikzpicture}
\caption{A generalized queue showing that $v = Q_S(u)$ where $S = \{2,4,5,8,10\}$, $u = 13233223121$ and $v =41412342314$. It has weight $x_2x_4x_5x_8x_{10}$.}
\end{figure}

In \cite{AAMP} a more general notion of queue was introduced. For any subset $S$ of $[N]$ we define a function $Q_S$ on words $u$ over $\{1, 2, \dots \}^<$ of length $N$, as follows.
Let $\mbf{m}$ be the type of $u$ and let $k$ be such that $m_1 + \dots + m_k \leq |S| < m_1 + \dots + m_{k+1}$. We will mostly be interested in cases where both inequalities are strict.

Think of two rows of length $N$ where the top row is labeled $u$ and the bottom row has boxes at the positions in $S$. We define a labeling $Q_S(u)$ of the bottom row as follows. Go through all particles of class $i$ for $i = 1, \dots, k-1$ and do the labeling as before. Now, go through all particles of class $i$, starting with the highest class $r$ and going \emph{downwards} to class $k+1$. For a particle of class $i$ in this case, find the first nonlabeled \emph{nonboxed} position weakly cyclically to the \emph{left}, and label it $i+1$.
After this, label all boxed non-labeled positions $k$ and all non-boxed non-labeled positions $k+1$.
The weight of the queue is simply $\prod_{i\in S} x_i$.
See Figure \ref{fig_generalized_queue} for an example.

\begin{dfn} A $(\sigma, \mbf{m})$-\defn{multiline queue} is an $r \times N$ array with $m_1 + \dots + m_{\sigma_i}$ boxes in row $i$ for $i = 1, \dots, r$. \end{dfn}

By thinking of each row in a $(\sigma, \mbf{m})$-MLQ as a generalized queue, we get a labeling of the MLQ.


\begin{thm}
  Let $u$ be a word of type $\mbf{m}$. The sum of the weights of all $(\sigma, \mbf{m})$-MLQ's whose bottom row is labeled $u$ does not depend on the permutation $\sigma$.
\end{thm}

We show this by showing that the weight is the same for any permutation $\sigma$ and for any adjacent transposition $\sigma s_k$ of $\sigma$.

Consider two consecutive rows in some MLQ. Suppose the set of positions of the boxes in the first row is $S_1$ and for the second row is $S_2$.

For such a pair $(S_1, S_2)$ we will associate another pair $(T_1, T_2)$ with $|T_1| = |S_2|$ and $|T_2| = |S_1|$ such that $Q_{S_2}Q_{S_1} = Q_{T_2}Q_{T_1}$ considered as functions from the labeling above $S_1$ or $T_1$ to the labeling on the row of $S_2$ or $T_2$. Furthermore, $(S_1, S_2)$ will have the same weight as $(T_1, T_2)$, that is, $\prod_{i\in S_1} x_i \prod_{i\in S_2} x_i = \prod_{i\in T_1} x_i \prod_{i\in T_2} x_i$.


%We proceed to define $(T_1, T_2)$ from $(S_1, S_2)$. To do this, put $S_1$ on top of $S_2$ and break the resulting two-row configuration into \emph{balanced blocks} (which are sets of consecutive columns).

%One conceptual way to look at this is to draw an infinite periodic Dyck path, as follows. Start at $(0, 0)$ and look at the first column in the two-row configuration. If there are boxes in both or in none of the rows, take a step to the right, to $(1, 0)$. If there is a box only in the upper row, go up, to $(1, 1)$, and if there is a box only in the bottom row, then go down to $(1, -1)$. Extend this path to the right and to the left (by interpreting indices modulo $N$). This path will either be going up on average or down on average (if it stays level on average, then $|S_1| = |S_2|$, and our bijection can be taken to be the identity map).

%Say that the path goes up on average. Then a balanced block corresponds to finding a local minimum of the path and finding the first time exactly to the left of this intersection where the path passes first.

%Suppose $|S_1| > |S_2|$. Then the boxes to be moved (straight) down are those that satisfy the following property. Look at a box in the upper row that does not have a box below it. Start a counter at +1. Now go left (cyclically), and add +1 for every box in the upper row and -1 for every box in the bottom row. If the counter never reaches $0$, the initial box has the property.

Suppose $|S_1| > |S_2|$. The case $|S_2| > |S_1|$ is analogous. We get $(T_1, T_2)$ from $(S_1, S_2)$ by taking letting $T_1 = S_1 \setminus C$ and $T_2 = S_2 \cup C$, where $C$ is the set of \emph{unbalanced boxes} in $S_1$ with respect to $(S_1, S_2)$.

One possible definition of unbalanced box: a box on the upper row is unbalanced if, looking at any (cyclic) interval of length $r$ whose left end point is the box, the number of boxes on the top row is always strictly larger than the number of boxes on the bottom row. (Better definition in terms of Dyck paths.) The number of unbalanced boxes is $|S_1| - |S_2|$.


{\bf Example.} 

$(S_1, S_2)$:

\scalebox{0.4}{
\begin{tikzpicture}
  \node[circle, draw=black] at (01, 2){\phantom{c}};
  \node[circle, draw=black] at (02, 2){\phantom{c}};
  \node[circle, draw=black] at (03, 2){\phantom{c}};
  \node[circle, draw=black] at (06, 2){\phantom{c}};
  \node[circle, draw=black] at (07, 2){\phantom{c}};
  \node[circle, draw=black] at (08, 2){\phantom{c}};
  \node[circle, draw=black] at (09, 2){\phantom{c}};
  \node[circle, draw=black] at (11, 2){\phantom{c}};
  \node[circle, draw=black] at (13, 2){\phantom{c}};
  \node[circle, draw=black] at (18, 2){\phantom{c}};
  \node[circle, draw=black] at (19, 2){\phantom{c}};
  \node[circle, draw=black] at (22, 2){\phantom{c}};
  \node[circle, draw=black] at (23, 2){\phantom{c}};
  \node[circle, draw=black] at (24, 2){\phantom{c}};
  \node[circle, draw=black] at (25, 2){\phantom{c}};
  \node[circle, draw=black] at (26, 2){\phantom{c}};

  \node[circle, draw=black] at (02, 1){\phantom{c}};
  \node[circle, draw=black] at (04, 1){\phantom{c}};
  \node[circle, draw=black] at (05, 1){\phantom{c}};
  \node[circle, draw=black] at (06, 1){\phantom{c}};
  \node[circle, draw=black] at (08, 1){\phantom{c}};
  \node[circle, draw=black] at (10, 1){\phantom{c}};
  \node[circle, draw=black] at (11, 1){\phantom{c}};
  \node[circle, draw=black] at (12, 1){\phantom{c}};
  \node[circle, draw=black] at (14, 1){\phantom{c}};
  \node[circle, draw=black] at (16, 1){\phantom{c}};
  \node[circle, draw=black] at (17, 1){\phantom{c}};
  \node[circle, draw=black] at (19, 1){\phantom{c}};
  \node[circle, draw=black] at (20, 1){\phantom{c}};
  \node[circle, draw=black] at (21, 2){u};
  \node[circle, draw=black] at (22, 1){\phantom{c}};
  \node[circle, draw=black] at (24, 1){\phantom{c}};
  \node[circle, draw=black] at (26, 1){\phantom{c}};
\end{tikzpicture}
}

\vspace{2cm}

$(T_1,T_2)$:

\scalebox{0.4}{
\begin{tikzpicture}
  \node[circle, draw=black] at (01, 2){\phantom{c}};
  \node[circle, draw=black] at (02, 2){\phantom{c}};
  \node[circle, draw=black] at (03, 2){\phantom{c}};
  \node[circle, draw=black] at (06, 2){\phantom{c}};
  \node[circle, draw=black] at (07, 2){\phantom{c}};
  \node[circle, draw=black] at (08, 2){\phantom{c}};
  \node[circle, draw=black] at (09, 2){\phantom{c}};
  \node[circle, draw=black] at (11, 2){\phantom{c}};
  \node[circle, draw=black] at (13, 2){\phantom{c}};
  \node[circle, draw=black] at (18, 2){\phantom{c}};
  \node[circle, draw=black] at (19, 2){\phantom{c}};
  \node[circle, draw=black] at (22, 2){\phantom{c}};
  \node[circle, draw=black] at (23, 2){\phantom{c}};
  \node[circle, draw=black] at (24, 2){\phantom{c}};
  \node[circle, draw=black] at (25, 2){\phantom{c}};
  \node[circle, draw=black] at (26, 2){\phantom{c}};

  \node[circle, draw=black] at (02, 1){\phantom{c}};
  \node[circle, draw=black] at (04, 1){\phantom{c}};
  \node[circle, draw=black] at (05, 1){\phantom{c}};
  \node[circle, draw=black] at (06, 1){\phantom{c}};
  \node[circle, draw=black] at (08, 1){\phantom{c}};
  \node[circle, draw=black] at (10, 1){\phantom{c}};
  \node[circle, draw=black] at (11, 1){\phantom{c}};
  \node[circle, draw=black] at (12, 1){\phantom{c}};
  \node[circle, draw=black] at (14, 1){\phantom{c}};
  \node[circle, draw=black] at (16, 1){\phantom{c}};
  \node[circle, draw=black] at (17, 1){\phantom{c}};
  \node[circle, draw=black] at (19, 1){\phantom{c}};
  \node[circle, draw=black] at (20, 1){\phantom{c}};
  \node[circle, draw=black] at (21, 1){u};
  \node[circle, draw=black] at (22, 1){\phantom{c}};
  \node[circle, draw=black] at (24, 1){\phantom{c}};
  \node[circle, draw=black] at (26, 1){\phantom{c}};
\end{tikzpicture}
}



\begin{lemma}
  The map $(S_1, S_2) \to (T_1, T_2)$ is a weight-preserving bijection such that $Q_{S_2}Q_{S_1} = Q_{T_2}Q_{T_1}$ (as functions on words/labelings).
\end{lemma}

Note that since $Q_{S_1}$ et.c. are defined only in terms of relative comparisons, to prove that $Q_{S_2}Q_{S_1}$ and $Q_{T_2}Q_{T_1}$ are equal, it is enough to show that they agree on all words from $\{1, 2\} ^N$. However, it is also enough (but seemingly wasteful) to show that they agree on all permutations of $[N]$. This is what we will do since this turns out to be easier to induct on.




{\bf Example}

To compute $[135452]$ we count $(1,1,1,1,2)$-MLQ's, such as the following one.

\begin{tikzpicture}\node at (0, 4){2};\node at (1, 4){2};\node at (2, 4){2};\node[circle, draw=black] at (3, 4){1};\node at (4, 4){2};\node at (5, 4){2};\node at (0, 3){3};\node at (1, 3){3};\node[circle, draw=black] at (2, 3){2};\node at (3, 3){3};\node at (4, 3){3};\node[circle, draw=black] at (5, 3){1};\node[circle, draw=black] at (0, 2){1};\node[circle, draw=black] at (1, 2){3};\node at (2, 2){4};\node at (3, 2){4};\node[circle, draw=black] at (4, 2){2};\node at (5, 2){4};\node[circle, draw=black] at (0, 1){1};\node[circle, draw=black] at (1, 1){3};\node at (2, 1){5};\node[circle, draw=black] at (3, 1){4};\node at (4, 1){5};\node[circle, draw=black] at (5, 1){2};\end{tikzpicture}

To get a word $136452$ or $135462$ we can add a new row on top, with $5$ boxes. The total weight of these additions is $e_5(x_1, \dots, x_6)$. Here is an example of such an addition.

\begin{tikzpicture}\node[circle, draw=black] at (0, 5){1};\node[circle, draw=black] at (1, 5){1};\node[circle, draw=black] at (2, 5){1};\node at (3, 5){2};\node[circle, draw=black] at (4, 5){1};\node[circle, draw=black] at (5, 5){1};\node at (0, 4){2};\node at (1, 4){2};\node at (2, 4){3};\node[circle, draw=black] at (3, 4){1};\node at (4, 4){2};\node at (5, 4){2};\node at (0, 3){3};\node at (1, 3){4};\node[circle, draw=black] at (2, 3){2};\node at (3, 3){3};\node at (4, 3){3};\node[circle, draw=black] at (5, 3){1};\node[circle, draw=black] at (0, 2){1};\node[circle, draw=black] at (1, 2){3};\node at (2, 2){4};\node at (3, 2){4};\node[circle, draw=black] at (4, 2){2};\node at (5, 2){5};\node[circle, draw=black] at (0, 1){1};\node[circle, draw=black] at (1, 1){3};\node at (2, 1){5};\node[circle, draw=black] at (3, 1){4};\node at (4, 1){6};\node[circle, draw=black] at (5, 1){2};\end{tikzpicture}

By the theorem, such queues are in 1-1 correspondence with ordinary queues counting, in this case, $[135462]$. By applying the bijection $4$ times to bring the top row to the bottom, we get the corresponding ordinary queue, as follows.\\

\noindent Swap rows 1 and 2:

\begin{tikzpicture}\node at (0, 5){2};\node at (1, 5){2};\node at (2, 5){2};\node[circle, draw=black] at (3, 5){1};\node at (4, 5){2};\node at (5, 5){2};\node[circle, draw=black] at (0, 4){2};\node[circle, draw=black] at (1, 4){2};\node at (2, 4){3};\node[circle, draw=black] at (3, 4){1};\node[circle, draw=black] at (4, 4){2};\node[circle, draw=black] at (5, 4){2};\node at (0, 3){3};\node at (1, 3){4};\node[circle, draw=black] at (2, 3){2};\node at (3, 3){3};\node at (4, 3){3};\node[circle, draw=black] at (5, 3){1};\node[circle, draw=black] at (0, 2){1};\node[circle, draw=black] at (1, 2){3};\node at (2, 2){4};\node at (3, 2){4};\node[circle, draw=black] at (4, 2){2};\node at (5, 2){5};\node[circle, draw=black] at (0, 1){1};\node[circle, draw=black] at (1, 1){3};\node at (2, 1){5};\node[circle, draw=black] at (3, 1){4};\node at (4, 1){6};\node[circle, draw=black] at (5, 1){2};\end{tikzpicture} \\

\noindent Swap rows 2 and 3:

\begin{tikzpicture}\node at (0, 5){2};\node at (1, 5){2};\node at (2, 5){2};\node[circle, draw=black] at (3, 5){1};\node at (4, 5){2};\node at (5, 5){2};\node at (0, 4){3};\node[circle, draw=black] at (1, 4){2};\node at (2, 4){3};\node at (3, 4){3};\node at (4, 4){3};\node[circle, draw=black] at (5, 4){1};\node[circle, draw=black] at (0, 3){3};\node at (1, 3){4};\node[circle, draw=black] at (2, 3){2};\node[circle, draw=black] at (3, 3){3};\node[circle, draw=black] at (4, 3){3};\node[circle, draw=black] at (5, 3){1};\node[circle, draw=black] at (0, 2){1};\node[circle, draw=black] at (1, 2){3};\node at (2, 2){4};\node at (3, 2){4};\node[circle, draw=black] at (4, 2){2};\node at (5, 2){5};\node[circle, draw=black] at (0, 1){1};\node[circle, draw=black] at (1, 1){3};\node at (2, 1){5};\node[circle, draw=black] at (3, 1){4};\node at (4, 1){6};\node[circle, draw=black] at (5, 1){2};\end{tikzpicture} \\

\noindent Swap rows 3 and 4:

\begin{tikzpicture}\node at (0, 5){2};\node at (1, 5){2};\node at (2, 5){2};\node[circle, draw=black] at (3, 5){1};\node at (4, 5){2};\node at (5, 5){2};\node at (0, 4){3};\node[circle, draw=black] at (1, 4){2};\node at (2, 4){3};\node at (3, 4){3};\node at (4, 4){3};\node[circle, draw=black] at (5, 4){1};\node[circle, draw=black] at (0, 3){3};\node at (1, 3){4};\node at (2, 3){4};\node at (3, 3){4};\node[circle, draw=black] at (4, 3){2};\node[circle, draw=black] at (5, 3){1};\node[circle, draw=black] at (0, 2){1};\node[circle, draw=black] at (1, 2){3};\node[circle, draw=black] at (2, 2){4};\node[circle, draw=black] at (3, 2){4};\node[circle, draw=black] at (4, 2){2};\node at (5, 2){5};\node[circle, draw=black] at (0, 1){1};\node[circle, draw=black] at (1, 1){3};\node at (2, 1){5};\node[circle, draw=black] at (3, 1){4};\node at (4, 1){6};\node[circle, draw=black] at (5, 1){2};\end{tikzpicture} \\

\noindent Swap rows 4 and 5

\begin{tikzpicture}\node at (0, 5){2};\node at (1, 5){2};\node at (2, 5){2};\node[circle, draw=black] at (3, 5){1};\node at (4, 5){2};\node at (5, 5){2};\node at (0, 4){3};\node[circle, draw=black] at (1, 4){2};\node at (2, 4){3};\node at (3, 4){3};\node at (4, 4){3};\node[circle, draw=black] at (5, 4){1};\node[circle, draw=black] at (0, 3){3};\node at (1, 3){4};\node at (2, 3){4};\node at (3, 3){4};\node[circle, draw=black] at (4, 3){2};\node[circle, draw=black] at (5, 3){1};\node[circle, draw=black] at (0, 2){1};\node[circle, draw=black] at (1, 2){3};\node at (2, 2){5};\node[circle, draw=black] at (3, 2){4};\node[circle, draw=black] at (4, 2){2};\node at (5, 2){5};\node[circle, draw=black] at (0, 1){1};\node[circle, draw=black] at (1, 1){3};\node[circle, draw=black] at (2, 1){5};\node[circle, draw=black] at (3, 1){4};\node at (4, 1){6};\node[circle, draw=black] at (5, 1){2};\end{tikzpicture} \\



\bibliographystyle{alpha}
\bibliography{queue}{}
\begin{thebibliography}{ABX}
\bibitem{AAMP} Chikashi Arita, Arvind Ayyer, Kirone Mallick and Sylvain Prolhac, Recursive structures in the multispecies TASEP, J. Phys. A 44, 335004 (2011).
\end{thebibliography}
\end{document}
