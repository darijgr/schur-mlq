\documentclass[reqno]{amsart}
\usepackage{setspace,tikz,xcolor,mathrsfs,listings,multicol}
\usepackage{amssymb}
%\usepackage{amsmath}
%\usepackage{amsthm}
\usepackage{rotating}
\usepackage[vcentermath]{youngtab}
%\usepackage{subcaption}
%\usepackage{fullpage}
\usepackage{enumerate}
\usepackage[all,cmtip]{xy}
\usetikzlibrary{arrows,matrix}
\usepackage{comment}
\usepackage{color}
\usepackage{ifthen}
\usepackage[sc]{mathpazo}
\usepackage[T1]{fontenc}
\usepackage{needspace}
\usepackage{tabls}
%\usepackage[margin=1.25in]{geometry}
%\onehalfspacing

\usepackage[colorlinks=true, pdfstartview=FitV, linkcolor=blue, citecolor=blue, urlcolor=blue]{hyperref}

% use these commands for typesetting doi and arXiv references in the bibliography
\newcommand{\doi}[1]{\href{http://dx.doi.org/#1}{\texttt{doi:#1}}}
\newcommand{\arxiv}[1]{\href{http://arxiv.org/abs/#1}{\texttt{arXiv:#1}}}

\newcommand{\iso}{\cong}
%\newcommand{\qedbox}{\rule{2mm}{2mm}}
%\renewcommand{\qedsymbol}{\qedbox}
\newcommand{\qbinom}[3]{\genfrac{[}{]}{0pt}{}{#1}{#2}_{#3}}
\newcommand{\absval}[1]{\left\lvert #1 \right\rvert}
\newcommand{\case}[1]{\vspace{12pt}\noindent\underline{#1}:}
\newcommand{\fs}{\mathcal{S}} % flagged Schur function
\newcommand{\mbf}{\mathbf}

\DeclareMathOperator{\supp}{supp} % Support
\DeclareMathOperator{\wt}{wt} % weight
\DeclareMathOperator{\pr}{pr} % promotion
\DeclareMathOperator{\id}{id} % identity
\DeclareMathOperator{\ch}{ch} % character
\DeclareMathOperator{\gr}{gr} % grading

\newcommand{\xx}{\mathbf{x}}
\newcommand{\mm}{\mathbf{m}}
\newcommand{\MLQ}{\mathbf{S}}

\newcommand{\mcA}{\mathcal{A}}
\newcommand{\mcF}{\mathcal{F}}
\newcommand{\mcM}{\mathcal{M}}

\newcommand{\ZZ}{\mathbb{Z}}
\newcommand{\QQ}{\mathbb{Q}}
\newcommand{\RR}{\mathbb{R}}
\newcommand{\CC}{\mathbb{C}}

\newcommand{\bze}{\overline{0}}
\newcommand{\bon}{\overline{1}}
\newcommand{\btw}{\overline{2}}
\newcommand{\bth}{\overline{3}}
\newcommand{\bfo}{\overline{4}}
\newcommand{\bfive}{\overline{5}}
\newcommand{\bsix}{\overline{6}}
\newcommand{\bseven}{\overline{7}}
\newcommand{\beight}{\overline{8}}
\newcommand{\bi}{\overline\imath}
\newcommand{\bk}{\overline{k}}
\newcommand{\brr}{\overline{r}}
\newcommand{\bn}{\overline{n}}
\newcommand{\ellbar}{\overline{\ell}}

\let\sumnonlimits\sum
\let\prodnonlimits\prod
\let\cupnonlimits\bigcup
\let\capnonlimits\bigcap
\renewcommand{\sum}{\sumnonlimits\limits}
\renewcommand{\prod}{\prodnonlimits\limits}
\renewcommand{\bigcup}{\cupnonlimits\limits}
\renewcommand{\bigcap}{\capnonlimits\limits}

\newenvironment{verlong}{}{}
\newenvironment{vershort}{}{}
\newenvironment{noncompile}{}{}
\excludecomment{verlong}
\includecomment{vershort}
\excludecomment{noncompile}
\newcommand{\rev}{\operatorname{rev}}
\newcommand{\conncomp}{\operatorname{conncomp}}
\newcommand{\NN}{\mathbb{N}}
\newcommand{\powset}[2][]{\ifthenelse{\equal{#2}{}}{\mathcal{P}\left(#1\right)}{\mathcal{P}_{#1}\left(#2\right)}}
% $\powset[k]{S}$ stands for the set of all $k$-element subsets of
% $S$. The argument $k$ is optional, and if not provided, the result
% is the whole powerset of $S$.
\newcommand{\set}[1]{\left\{ #1 \right\}}
% $\set{...}$ yields $\left\{ ... \right\}$.
\newcommand{\abs}[1]{\left| #1 \right|}
% $\abs{...}$ yields $\left| ... \right|$.
\newcommand{\tup}[1]{\left( #1 \right)}
% $\tup{...}$ yields $\left( ... \right)$.
\newcommand{\ive}[1]{\left[ #1 \right]}
% $\ive{...}$ yields $\left[ ... \right]$.
\newcommand{\verts}[1]{\operatorname{V}\left( #1 \right)}
% $\verts{...}$ yields $\operatorname{V}\left( ... \right)$.
\newcommand{\edges}[1]{\operatorname{E}\left( #1 \right)}
% $\edges{...}$ yields $\operatorname{E}\left( ... \right)$.
\newcommand{\arcs}[1]{\operatorname{A}\left( #1 \right)}
% $\arcs{...}$ yields $\operatorname{A}\left( ... \right)$.
\newcommand{\underbrack}[2]{\underbrace{#1}_{\substack{#2}}}
% $\underbrack{...1}{...2}$ yields
% $\underbrace{...1}_{\substack{...2}}$. This is useful for doing
% local rewriting transformations on mathematical expressions with
% justifications.
\newcommand{\mlnode}[1]{\node[circle, draw=black] at (#1){\phantom{c}};}

% Dark red emphasis
\definecolor{darkred}{rgb}{0.7,0,0} % darkred color
\newcommand{\defn}[1]{{\color{darkred}\emph{#1}}} % emphasis of a definition


%% For typesetting code listings                                                
\usepackage{listings}
\lstdefinelanguage{Sage}[]{Python}
{morekeywords={False,sage,True},sensitive=true}
\lstset{
  frame=single,
  showtabs=False,
  showspaces=False,
  showstringspaces=False,
  commentstyle={\ttfamily\color{dgreencolor}},
  keywordstyle={\ttfamily\color{dbluecolor}\bfseries},
  stringstyle={\ttfamily\color{dgraycolor}\bfseries},
  language=Sage,
  basicstyle={\footnotesize\ttfamily},
  aboveskip=0.75em,
  belowskip=0.75em,
  xleftmargin=.15in,
}
\definecolor{dblackcolor}{rgb}{0.0,0.0,0.0}
\definecolor{dbluecolor}{rgb}{0.01,0.02,0.7}
\definecolor{dgreencolor}{rgb}{0.2,0.4,0.0}
\definecolor{dgraycolor}{rgb}{0.30,0.3,0.30}
\newcommand{\dblue}{\color{dbluecolor}\bf}
\newcommand{\dred}{\color{dredcolor}\bf}
\newcommand{\dblack}{\color{dblackcolor}\bf}

\usepackage{xparse}

\makeatletter
% \specialmergetwolists{<coupler>}{<list1>}{<list2>}{<return macro>}
% \specialmergetwolists*{<coupler>}{<listcmd1>}{<listcmd2>}{<return macro>}
\protected\def\specialmergetwolists{%
  \begingroup
  \@ifstar{\def\cnta{1}\@specialmergetwolists}
    {\def\cnta{0}\@specialmergetwolists}%
}
\def\@specialmergetwolists#1#2#3#4{%
  \def\tempa##1##2{%
    \edef##2{%
      \ifnum\cnta=\@ne\else\expandafter\@firstoftwo\fi
      \unexpanded\expandafter{##1}%
    }%
  }%
  \tempa{#2}\tempb\tempa{#3}\tempa
  \def\cnta{0}\def#4{}%
  \foreach \x in \tempb{%
    \xdef\cnta{\the\numexpr\cnta+1}%
    \gdef\cntb{0}%
    \foreach \y in \tempa{%
      \xdef\cntb{\the\numexpr\cntb+1}%
      \ifnum\cntb=\cnta\relax
        \xdef#4{#4\ifx#4\empty\else,\fi\x#1\y}%
        \breakforeach
      \fi
    }%
  }%
  \endgroup
}
\makeatother

\theoremstyle{plain}
\newtheorem{thm}{Theorem}[section]
\newtheorem{lemma}[thm]{Lemma}
\newtheorem{conj}[thm]{Conjecture}
\newtheorem{prop}[thm]{Proposition}
\newtheorem{cor}[thm]{Corollary}
\theoremstyle{definition}
\newtheorem{dfn}[thm]{Definition}
\newtheorem{example}[thm]{Example}
\newtheorem{remark}[thm]{Remark}
\numberwithin{equation}{section}
%\numberwithin{figure}{section}
%\numberwithin{table}{section}
%\setcounter{section}{-1}

% For breaking equations across multiple pages
% \allowdisplaybreaks[1]

\usepackage[colorinlistoftodos]{todonotes}
\newcommand{\darij}[1]{\todo[size=\tiny,color=red!30]{#1 \\ \hfill --- Darij}}
\newcommand{\Darij}[1]{\todo[size=\tiny,inline,color=red!30]{#1
      \\ \hfill --- Darij}}
\newcommand{\travis}[1]{\todo[size=\tiny,color=blue!30]{#1 \\ \hfill --- Travis}}
\newcommand{\Travis}[1]{\todo[size=\tiny,inline,color=blue!30]{#1
      \\ \hfill --- Travis}}



%%%%%%%%%%%%%%%%%%%%%%%%%%%%%%%%%%%%%%%%

\begin{document}
\title[MLQs]{Multiline queues with special parameters}

\author[E.~Aas]{Erik Aas}
\address[E. Aas]{Department of Mathematics, Pennsylvania State University, ??, ??}
\email{eaas@kth.se}

\author[D.~Grinberg]{Darij Grinberg}
\address[D. Grinberg]{School of Mathematics, University of Minnesota, 206 Church St. SE, Minneapolis, MN 55455}
\email{darij.grinberg@gmail.com}
\urladdr{http://www.cip.ifi.lmu.de/~grinberg/}

\author[T.~Scrimshaw]{Travis Scrimshaw}
\address[T. Scrimshaw]{School of Mathematics and Physics, University of Queensland, St. Lucia, QLD 4072, Australia}
\email{tcscrims@gmail.com}
\urladdr{https://sites.google.com/view/tscrim/home}

\keywords{multiline queue, flagged Schur function}
\subjclass[2010]{05E10, 17B37}

%\thanks{TS was partially supported by the National Science Foundation RTG grant NSF/DMS-1148634.}

\begin{abstract}
We prove the spectral parameter version of the MLQs conjecture.
\end{abstract}

\maketitle



%=====================================================================
\section{Introduction}
\label{sec:introduction}

To be written last.






%=====================================================================
\section{Background}
\label{sec:background}

\subsection{Multiline queues}

Fix a positive integer $N$.
Let $\ive{N}$ denote the set $\set{1, 2, \ldots, N}$.

%\begin{dfn} \label{def.mlqs.mlq}
Fix a nonnegative integer $n$.
A \defn{multiline queue (MLQ)} is an $(n+1)$-tuple $(S_0, S_1, \dotsc, S_n)$ such that $S_0 = \emptyset$ and, for all $1 \leq i \leq n$, we have $S_i \subseteq \ive{N}$ and $\lvert S_{i-1} \rvert < \lvert S_i \rvert$.
We shall represent an MLQ $\tup{S_0, S_1, \ldots, S_n}$ as an
$n \times N$-array, each row of which has some of its boxes
marked. Namely, for each $i \in \set{1, 2, \ldots, n}$ and
each $j \in S_i$, we mark the $j$-th box in row $i$.
(The rows and the columns are indexed as usual for a matrix.)
%\end{dfn}

For example, $\tup{\varnothing, \set{2, 4}, \set{1, 4, 5}}$
is an MLQ for $n = 2$ and $N = 6$; we draw it as follows:
\scalebox{0.4}{
\begin{tikzpicture}
  \mlnode{2, 2}
  \mlnode{4, 2}
  \mlnode{1, 1}
  \mlnode{4, 1}
  \mlnode{5, 1}
\end{tikzpicture}
}

\begin{remark}
Our notion of an MLQ corresponds to what is called a ``discrete MLQ'' in~\cite[\S 2.2]{AasLin17}.
We omit the word ``discrete'', as we will not encounter any other kind of MLQs.
\end{remark}

Let $\MLQ = (S_0, \dotsc, S_n)$ be an MLQ.
The \defn{type} of $\MLQ$ is the  $n$-tuple $\mm = \tup{m_1, m_2, \ldots, m_n}$ of
positive
% positive, not just nonnegative, since we required the S_i to strictly grow in size
integers such that $\abs{S_i} = m_1 + m_2 + \cdots + m_i$.
(In particular, $S_0 = \varnothing$.)

In order to define the labeling of an MLQ, we need to setup some notation.

If $p$ is an element of $\ive{N}$,
and if $S$ is a nonempty subset of $\ive{N}$,
then an element $\min_p S$ of $S$ is defined as follows:
If there exists an element of $S$ that is greater
or equal to $p$, then we set $\min_p S$ to be the
smallest such element;
otherwise, we set $\min_p S$ to be the smallest
element of $S$.
(Roughly speaking, $\min_p S$ is the first element
of $S$ you encounter when traversing the elements
$p, p+1, \ldots, n, 1, 2, \ldots, p-1$.)

Let $U$ and $V$ be two subsets of $\ive{N}$ satisfying
$\abs{U} \leq \abs{V}$.
Let $f \colon U \to \ive{n}$ be any map; we shall
refer to it as a ``labeling'', and we shall refer an
image $f \tup{u}$ as the ``label'' of $u$ in $f$.
Let $k$ be a positive integer.
We define the map $f \downarrow_{V, k} : V \to \ive{n}$
as follows (recursively over $\abs{U}$):

\begin{itemize}
 \item If $U = \varnothing$, then $f \downarrow_{V, k}$ sends
       each element of $V$ to $k$.
 \item Otherwise, pick an element $u$ of $U$ having minimum
       label (i.e., having minimum $f \tup{u}$).
       Let $v = \min_u V$.
       Let $g$ be the restriction of $f$ to the subset
       $U \setminus \set{u}$.
       Let $g' = g \downarrow_{V \setminus \set{v}, k}$.
       Then, the labeling $f \downarrow_{V, k}$ is defined by
       setting
       \[
        \tup{f \downarrow_{V, k}} \tup{s}
        =
        \begin{cases}
         f \tup{u}, & \text{ if } s = v; \\
         g \tup{s}, & \text{ if } s \neq v
        \end{cases}
        \qquad \text{ for all } s \in V .
       \]
\end{itemize}

\darij{Remember to fix the spelling of ``labelling''.}

\begin{prop} \label{prop.mlqs.down.indep}
Let $U$ and $V$ be two subsets of $\ive{N}$ satisfying $\abs{U} \leq \abs{V}$.
Let $f \colon U \to \ive{n}$.
Let $k$ be a positive integer.
The definition of $f \downarrow_{V, k}$ given above does not
depend on the choice of $u$.
\end{prop}

\begin{proof}
We can replace this definition by an equivalent one, which
clearly does not depend on any choices:

For any $p \in \ive{N}$, we define a subset
$\operatorname{arc}\ive{p : s}$ of $\ive{N}$ by
\[
\operatorname{arc}\ive{p : s}
=
\begin{cases}
\set{p, p+1, \ldots, s}, & \text{ if } p \leq s; \\
\set{p, p+1, \ldots, N, 1, 2, \ldots, s}, & \text{ if } p > s
\end{cases} .
\]

For any $0 \leq i \leq n$, we let $f^{-1} \tup{\leq i}$ be
the subset
$f^{-1} \tup{\ive{i}} = \set{u \in U \mid f \tup{u} \leq i}$
of $U$.

Let $s \in V$.
If there exists some $i \in \ive{n}$ and some
$p \in \ive{N}$ such that the set
$\operatorname{arc}\ive{p : s}$
contains at least as many elements of $f^{-1} \tup{\leq i}$
as it contains elements of $V$
(that is, it satisfies
$\abs{\operatorname{arc}\ive{p : s} \cap f^{-1} \tup{\leq i}}
\geq \abs{\operatorname{arc}\ive{p : s} \cap V}$),
then we pick the smallest $i$
for which this holds,
and we set $\tup{f \downarrow_{V, k}}\tup{s} = i$.
Otherwise, we set
$\tup{f \downarrow_{V, k}}\tup{s} = k$.

[The condition
``$\abs{\operatorname{arc}\ive{p : s} \cap f^{-1} \tup{\leq i}}
\geq \abs{\operatorname{arc}\ive{p : s} \cap V}$
for some $p \in \ive{N}$''
can also be restated as follows:
If we define a circular word on $N$ letters whose $q$-th letter
is
\[
\begin{cases}
(, & \text{ if } q \in f^{-1} \tup{\leq i} \setminus V; \\
), & \text{ if } q \in V \setminus f^{-1} \tup{\leq i}; \\
0, & \text{ otherwise}
\end{cases}
,
\]
and if we match parentheses as usual for a circular word
(this means, in particular, that the last $($ can match the
first $)$),
then the $s$-th letter is either $0$ or a matched parenthesis.
This is strongly reminiscent of the Remmel-Shimozono proof
of the LR rule, and suggests that something coplactic is
going on.]

......
\end{proof}

\begin{dfn}
The \defn{labeling} of $\MLQ$ is the tuple $(f_0, f_1, \dotsc, f_n)$ given recursively as follows.
Define $f_0 \colon S_0 \to \ive{n}$ in the obvious way (since $S_0 = \varnothing$) and then define $f_i$ by $f_i = f_{i-1} \downarrow_{S_i, i}$.

We define the \defn{bottom row labels} of an MLQ $\MLQ$ with $S_n = (b_1 < b_2 < \cdots < b_n)$ to be $\tup{f_n(b_1), f_n(b_2), \dotsc, f_n(b_n)}$.
\end{dfn}

\begin{dfn} \label{def.mlq.sw}
Let $\MLQ = \tup{S_0, S_1, \ldots, S_n}$ be an MLQ of type $\mm$.
The \defn{spectral weight} $\wt\tup{\MLQ}$
of $\MLQ$ is defined to be the monomial
\[
 \prod_{i=0}^{n-1} \prod_{s \in S_i} x_s
\]
in the indeterminates $x_1, x_2, \ldots, x_N$.
\end{dfn}


%\subsection{Flagged Schur functions}
%
%Let $\lambda$ be a partition of length $m$ and $b = (b_1 \leq b_2 \leq \cdots \leq b_m)$.
%Let
%\[
%h_d(k) = \sum_{i_1 \leq \cdots \leq i_d \leq k} x_{i_1} \cdots x_{i_d}
%\]
%be the complete homogeneous symmetric function of degree $d$ in the variables $x_1, \dotsc, x_k$.
%A \defn{flagged Schur function} is
%\[
%\fs_{\lambda}(b; \xx) = \det\big[ h_{\lambda_i - i + j}(b_i) \bigr]_{i,j=1}^m.
%\]
%We can also express the flagged Schur function combinatorially using $\mcF_{\lambda}(b)$, the semistandard tableaux of shape $\lambda$ such that the max entry in row $i$ is at most $b_i$.
%From~\cite{Wachs85}, we have
%\[
%\fs_{\lambda}(b; \xx) = \sum_{T \in \mcF_{\lambda}(b)} x^T.
%\]








%=====================================================================
\section{Main result}
\label{sec:result}


Fix a subset $B = \set{b_1 < b_2 < \cdots < b_n}$ of $\ive{N}$.
Let $\mcM \tup{B, \mm}$ be the set of all MLQs $\MLQ = (S_0, \dotsc, S_n)$ of type $\mm$
with $S_n = X$ and bottom row labels
\[
(\underbrace{n, \dotsc, n}_{m_n},\, \dotsc,\, \underbrace{2, \dotsc, 2}_{m_2},\, \underbrace{1, \dotsc, 1}_{m_1}).
\]

\begin{thm} \label{thm.decreasing}
Fix a nonnegative integer $n$,
Fix a subset $B = \set{b_1 < b_2 < \cdots < b_n}$ of $\ive{N}$.
Fix an $n$-tuple $\mm = \tup{m_1, m_2, \ldots, m_n}$ of positive integers.

We have
\[
\sum_{\MLQ \in \mcM \tup{B, f}} \wt\tup{\MLQ}
= \det \tup{
             \tup{ h_{\gamma_i - i + j} \tup{x_1, x_2, \ldots, x_{b_i}}
                 }_{1 \leq i \leq n, \  1 \leq j \leq n}
           },
\]
where $\gamma_i$ is the number of all
$k \in \set{1, 2, \ldots, n-1}$ satisfying
$m_{k + 1} + m_{k + 2} + \cdots + m_n < i$.
\end{thm}

\begin{proof}
For ordered $b$, the proof that there is no wrapping is the same as in~\cite{AasLin17} when all entries of $b$ are distinct.
Hence the only possibility for lattice paths to intersect is when the two paths have the same color.

If two paths have the same color and intersect, separate them into a non-intersecting lattice path by shifting up 1 step the entire path and all other paths above it.

Thus, by the Lindstr\"om--Gessel--Viennot Lemma, we have
\[
\det\big[ h_{j-k_{n+1-i}(S)'}(b_{n+1-i}) \bigr]_{i,j=1}^n.
\]
Thus the claim follows.
\end{proof}

\begin{comment}
 We hope to have
\[
\sum_{\MLQ \in \mcM \tup{S, f}} \wt\tup{\MLQ}
= \fs_{\lambda}(b; \xx),
\]
where $\lambda$ is the partition $\tup{1^{m_1}, 2^{m_2}, 3^{m_3}, \ldots}$
(in multiplicity notation), or at least some partition.

But this doesn't seem to work on the nose, since either
the partition increases or the flags decrease.

Maybe ghost entries, so a more general notion of flagged
Schur functions.
\end{comment}


\begin{thm}
Consider the action of the Young subgroup $S_{\mu}$.
Then, we have
\[
\mcM_{\mu}(b; \xx) = \mcM_k(b; \xx).
\]
\end{thm}

\Travis{This theorem is not the correct statement.}

\begin{proof}
The only wrapping the can occur is within the action of a particular Young subgroup.
This is similar to the proof that there is no wrapping in~\cite{AasLin17}.
\end{proof}

\section{The projection formula}

For the homogenous case, where all spectral parameters equal, it is well known that the number of multiline queues with a given bottom row is proportional to the stationary distribution of a certain Markov chain, the multi-species TASEP on a ring, at the state corresponding to that bottom row. The TASEP has a property of merging classes that translates to a statement about multiline queues, usually called the {\it projection principle}. Somewhat surprisingly, this principle has a generalisation to our case with non-equal spectral parameters. We now state this generalisation.

\begin{thm}
  Let $u$ be a word of type $\mbf{m}$, and let $v$ be the word obtained from $u$ by merging classes $k$ and $k+1$.
  Then $\sum_{u'} \wt{u} = e_{m_1+\dots+m_k}(x) \wt{v}$,
  where $u'$ runs over all words gotten by permuting $k$ and $k+1$ in $u$.
\end{thm}

This theorem will be a corollary of Theorem [??].


\begin{example}
Looking at queues, we get
\begin{align*}
[13234] & = x_1 x_2 x_3^2 x_4 (x_1^2 + x_1 x_4 + x_1 x_5 + x_4 x_5 + x_5^2)
\\ [13245] & = x_1 x_2 x_3^2 x_4 (x_1^2 + x_1x_4 + x_1x_5 + x_4^2 + x_4x_5 + x_5^2)
\\ & \hspace{20pt} \times (x_1x_2x_3 + x_1x_2x_5+x_1x_3x_5+x_2x_3x_5)
\\ [14235] & = x_1x_2x_3^2x_4^2 (x_1^3x_2 + x_1^3x_3 + x_1^3x_5 + x_1^2x_2x_3 + x_1^2x_2x_4 + 2x_1^2x_2x_5 + x_1^2x_3x_4
\\ & \hspace{55pt} + 2x_1^2x_3x_5 + x_1^2x_4x_5 + x_1^2x_5^2 + x_1x_2x_3x_5 + x_1x_2x_4x_5 + 2x_1x_2x_5^2
\\ & \hspace{55pt} + x_1x_3x_4x_5 + 2x_1x_3x_5^2 + x_1x_4x_5^2 + x_1x_5^3 + x_2x_3x_5^2 + x_2x_4x_5^2
\\ & \hspace{55pt} + x_2x_5^3 + x_3x_4x_5^2 + x_3x_5^3)
\end{align*}
(We have factored the expressions for readability only.)

The projection formula states that $[13234] e_3(x) = [13245] + [14235]$, as we can check directly in this case.
\end{example}


\section{Generalized queues}

\begin{figure}
\label{fig_generalized_queue}
\begin{tikzpicture}
  \node at  (1,2){1};
  \node at  (2,2){3};
  \node at  (3,2){2};
  \node at  (4,2){3};
  \node at  (5,2){3};
  \node at  (6,2){2};
  \node at  (7,2){2};
  \node at  (8,2){3};
  \node at  (9,2){1};
  \node at (10,2){2};
  \node at (11,2){1};

  \node at                     (1,1){4};
  \node[circle,draw=black] at  (2,1){1};
  \node at                     (3,1){4};
  \node[circle,draw=black] at  (4,1){1};
  \node[circle,draw=black] at  (5,1){2};
  \node at                     (6,1){3};
  \node at                     (7,1){4};
  \node[circle,draw=black] at  (8,1){2};
  \node at                     (9,1){3};
  \node[circle,draw=black] at (10,1){1};
  \node at                    (11,1){4};
\end{tikzpicture}
\caption{A generalized queue showing that $v = Q_S(u)$ where $S = \{2,4,5,8,10\}$, $u = 13233223121$ and $v =41412342314$. It has weight $x_2x_4x_5x_8x_{10}$.}
\end{figure}

In~\cite{AAMP} a more general notion of queue was introduced. For any subset $S$ of $[N]$ we define a function $Q_S$ on words $u$ over $\{1, 2, \dots \}^<$ of length $N$, as follows.
Let $\mbf{m}$ be the type of $u$ and let $k$ be such that $m_1 + \dots + m_k \leq |S| < m_1 + \dots + m_{k+1}$. We will mostly be interested in cases where both inequalities are strict.

Think of two rows of length $N$ where the top row is labeled $u$ and the bottom row has boxes at the positions in $S$. We define a labeling $Q_S(u)$ of the bottom row as follows. Go through all particles of class $i$ for $i = 1, \dots, k-1$ and do the labeling as before. Now, go through all particles of class $i$, starting with the highest class $r$ and going \emph{downwards} to class $k+1$. For a particle of class $i$ in this case, find the first nonlabeled \emph{nonboxed} position weakly cyclically to the \emph{left}, and label it $i+1$.
After this, label all boxed non-labeled positions $k$ and all non-boxed non-labeled positions $k+1$.
The weight of the queue is simply $\prod_{i\in S} x_i$.
See Figure~\ref{fig_generalized_queue} for an example.

\begin{dfn} A $(\sigma, \mbf{m})$-\defn{multiline queue} is an $r \times N$ array with $m_1 + \dots + m_{\sigma_i}$ boxes in row $i$ for $i = 1, \dots, r$. \end{dfn}

By thinking of each row in a $(\sigma, \mbf{m})$-MLQ as a generalized queue, we get a labeling of the MLQ.


\begin{thm}
  Let $u$ be a word of type $\mbf{m}$. The sum of the weights of all $(\sigma, \mbf{m})$-MLQ's whose bottom row is labeled $u$ does not depend on the permutation $\sigma$.
\end{thm}

We show this by showing that the weight is the same for any permutation $\sigma$ and for any adjacent transposition $\sigma s_k$ of $\sigma$.

Consider two consecutive rows in some MLQ. Suppose the set of positions of the boxes in the first row is $S_1$ and for the second row is $S_2$.

For such a pair $(S_1, S_2)$ we will associate another pair $(T_1, T_2)$ with $|T_1| = |S_2|$ and $|T_2| = |S_1|$ such that $Q_{S_2}Q_{S_1} = Q_{T_2}Q_{T_1}$ considered as functions from the labeling above $S_1$ or $T_1$ to the labeling on the row of $S_2$ or $T_2$. Furthermore, $(S_1, S_2)$ will have the same weight as $(T_1, T_2)$, that is, $\prod_{i\in S_1} x_i \prod_{i\in S_2} x_i = \prod_{i\in T_1} x_i \prod_{i\in T_2} x_i$.

We now describe how to get $(T_1, T_2)$ from $(S_1, S_2)$. Think of $(S_1,S_2)$ as two rows on top of each other, and write these as a (cyclic) word $w_1\dots w_N$ over the alphabet $\{u, d, t, n\}$. Here $u$ refers to a column where the Upper row has a box, while the bottom does not, $t$ is a column with two boxes, $n$ is a column with no boxes and $d$ is a column with one box in the bottom row.

Suppose $|S_1| > |S_2|$ (the case $|S_1| < |S_2|$ is analogous, and the case $|S_1| = |S_2|$ is trivial). Then any such word can be uniquely written as an alternating concatentation $\dots UBUB\dots$ where $U$ refers to a word consisting of $u$'s only, and $B$ is a \emph{balanced} word. A word is balanced if, for each $k$, the first $k$ columns in the word has at least as many boxes in the upper row as in the bottom row. [TODO: description in terms of Dyck paths]

Now $(T_1, T_2)$ is obtained from $(S_1, S_2)$ by replacing all $u$'s in $U$ blocks in $(S_1,S_2)$ with $d$'s.

{\bf Example.} 

$(S_1, S_2)$:

\scalebox{0.4}{
\begin{tikzpicture}
  \node[circle, draw=black] at (01, 2){\phantom{c}};
  \node[circle, draw=black] at (02, 2){\phantom{c}};
  \node[circle, draw=black] at (03, 2){\phantom{c}};
  \node[circle, draw=black] at (06, 2){\phantom{c}};
  \node[circle, draw=black] at (07, 2){\phantom{c}};
  \node[circle, draw=black] at (08, 2){\phantom{c}};
  \node[circle, draw=black] at (09, 2){\phantom{c}};
  \node[circle, draw=black] at (11, 2){\phantom{c}};
  \node[circle, draw=black] at (13, 2){\phantom{c}};
  \node[circle, draw=black] at (18, 2){\phantom{c}};
  \node[circle, draw=black] at (19, 2){\phantom{c}};
  \node[circle, draw=black] at (22, 2){\phantom{c}};
  \node[circle, draw=black] at (23, 2){\phantom{c}};
  \node[circle, draw=black] at (24, 2){\phantom{c}};
  \node[circle, draw=black] at (25, 2){\phantom{c}};
  \node[circle, draw=black] at (26, 2){\phantom{c}};

  \node[circle, draw=black] at (02, 1){\phantom{c}};
  \node[circle, draw=black] at (04, 1){\phantom{c}};
  \node[circle, draw=black] at (05, 1){\phantom{c}};
  \node[circle, draw=black] at (06, 1){\phantom{c}};
  \node[circle, draw=black] at (08, 1){\phantom{c}};
  \node[circle, draw=black] at (10, 1){\phantom{c}};
  \node[circle, draw=black] at (11, 1){\phantom{c}};
  \node[circle, draw=black] at (12, 1){\phantom{c}};
  \node[circle, draw=black] at (14, 1){\phantom{c}};
  \node[circle, draw=black] at (16, 1){\phantom{c}};
  \node[circle, draw=black] at (17, 1){\phantom{c}};
  \node[circle, draw=black] at (19, 1){\phantom{c}};
  \node[circle, draw=black] at (20, 1){\phantom{c}};
  \node[circle, draw=black] at (21, 2){u};
  \node[circle, draw=black] at (22, 1){\phantom{c}};
  \node[circle, draw=black] at (24, 1){\phantom{c}};
  \node[circle, draw=black] at (26, 1){\phantom{c}};
\end{tikzpicture}
}

\vspace{2cm}

$(T_1,T_2)$:

\scalebox{0.4}{
\begin{tikzpicture}
  \node[circle, draw=black] at (01, 2){\phantom{c}};
  \node[circle, draw=black] at (02, 2){\phantom{c}};
  \node[circle, draw=black] at (03, 2){\phantom{c}};
  \node[circle, draw=black] at (06, 2){\phantom{c}};
  \node[circle, draw=black] at (07, 2){\phantom{c}};
  \node[circle, draw=black] at (08, 2){\phantom{c}};
  \node[circle, draw=black] at (09, 2){\phantom{c}};
  \node[circle, draw=black] at (11, 2){\phantom{c}};
  \node[circle, draw=black] at (13, 2){\phantom{c}};
  \node[circle, draw=black] at (18, 2){\phantom{c}};
  \node[circle, draw=black] at (19, 2){\phantom{c}};
  \node[circle, draw=black] at (22, 2){\phantom{c}};
  \node[circle, draw=black] at (23, 2){\phantom{c}};
  \node[circle, draw=black] at (24, 2){\phantom{c}};
  \node[circle, draw=black] at (25, 2){\phantom{c}};
  \node[circle, draw=black] at (26, 2){\phantom{c}};

  \node[circle, draw=black] at (02, 1){\phantom{c}};
  \node[circle, draw=black] at (04, 1){\phantom{c}};
  \node[circle, draw=black] at (05, 1){\phantom{c}};
  \node[circle, draw=black] at (06, 1){\phantom{c}};
  \node[circle, draw=black] at (08, 1){\phantom{c}};
  \node[circle, draw=black] at (10, 1){\phantom{c}};
  \node[circle, draw=black] at (11, 1){\phantom{c}};
  \node[circle, draw=black] at (12, 1){\phantom{c}};
  \node[circle, draw=black] at (14, 1){\phantom{c}};
  \node[circle, draw=black] at (16, 1){\phantom{c}};
  \node[circle, draw=black] at (17, 1){\phantom{c}};
  \node[circle, draw=black] at (19, 1){\phantom{c}};
  \node[circle, draw=black] at (20, 1){\phantom{c}};
  \node[circle, draw=black] at (21, 1){u};
  \node[circle, draw=black] at (22, 1){\phantom{c}};
  \node[circle, draw=black] at (24, 1){\phantom{c}};
  \node[circle, draw=black] at (26, 1){\phantom{c}};
\end{tikzpicture}
}



\begin{lemma}
  The map $(S_1, S_2) \to (T_1, T_2)$ is a weight-preserving bijection such that $Q_{S_2}Q_{S_1} = Q_{T_2}Q_{T_1}$ (as functions on words/labelings).
\end{lemma}

\begin{proof} (TODO: this is a sketch)

It is enough to prove this lemma for input words that are permutations of $[N]$\footnote{It is also enough to prove it for all words in $\{1,2\}^N$.}.
We write $r = |S_1| > s = |S_2|$.
Let $u$ be a permutation of $[N]$, $v_S = Q_{S_2}Q_{S_1}(u)$ and $v_T = Q_{T_2}Q_{T_1}(u)$.

\begin{enumerate}
  \item The particles labeled $\{1, \dots, r\}$ are in the same positions in $v_S$ and $v_T$.
  \item The particles labeled $\{s+1, s+2, \dots, N\}$ are in the same positions in $v_S$ and $v_T$.
  \item The remaining particles $\{r+1, r+2, \dots, s-1\}$ are in the same positions in $v_S$ and $v_T$

Note that the claims $(1)$ and $(2)$ are equivalent, and that the bottom row is labeled by $(3)$ only in non-balanced parts of the word.
\end{enumerate}



\end{proof}



{\bf Example}

To compute $[135452]$ we count $(1,1,1,1,2)$-MLQ's, such as the following one.

\begin{tikzpicture}\node at (0, 4){2};\node at (1, 4){2};\node at (2, 4){2};\node[circle, draw=black] at (3, 4){1};\node at (4, 4){2};\node at (5, 4){2};\node at (0, 3){3};\node at (1, 3){3};\node[circle, draw=black] at (2, 3){2};\node at (3, 3){3};\node at (4, 3){3};\node[circle, draw=black] at (5, 3){1};\node[circle, draw=black] at (0, 2){1};\node[circle, draw=black] at (1, 2){3};\node at (2, 2){4};\node at (3, 2){4};\node[circle, draw=black] at (4, 2){2};\node at (5, 2){4};\node[circle, draw=black] at (0, 1){1};\node[circle, draw=black] at (1, 1){3};\node at (2, 1){5};\node[circle, draw=black] at (3, 1){4};\node at (4, 1){5};\node[circle, draw=black] at (5, 1){2};\end{tikzpicture}

To get a word $136452$ or $135462$ we can add a new row on top, with $5$ boxes. The total weight of these additions is $e_5(x_1, \dots, x_6)$. Here is an example of such an addition.

\begin{tikzpicture}\node[circle, draw=black] at (0, 5){1};\node[circle, draw=black] at (1, 5){1};\node[circle, draw=black] at (2, 5){1};\node at (3, 5){2};\node[circle, draw=black] at (4, 5){1};\node[circle, draw=black] at (5, 5){1};\node at (0, 4){2};\node at (1, 4){2};\node at (2, 4){3};\node[circle, draw=black] at (3, 4){1};\node at (4, 4){2};\node at (5, 4){2};\node at (0, 3){3};\node at (1, 3){4};\node[circle, draw=black] at (2, 3){2};\node at (3, 3){3};\node at (4, 3){3};\node[circle, draw=black] at (5, 3){1};\node[circle, draw=black] at (0, 2){1};\node[circle, draw=black] at (1, 2){3};\node at (2, 2){4};\node at (3, 2){4};\node[circle, draw=black] at (4, 2){2};\node at (5, 2){5};\node[circle, draw=black] at (0, 1){1};\node[circle, draw=black] at (1, 1){3};\node at (2, 1){5};\node[circle, draw=black] at (3, 1){4};\node at (4, 1){6};\node[circle, draw=black] at (5, 1){2};\end{tikzpicture}

By the theorem, such queues are in 1-1 correspondence with ordinary queues counting, in this case, $[135462]$. By applying the bijection $4$ times to bring the top row to the bottom, we get the corresponding ordinary queue, as follows.\\

\noindent Swap rows 1 and 2:

\begin{tikzpicture}\node at (0, 5){2};\node at (1, 5){2};\node at (2, 5){2};\node[circle, draw=black] at (3, 5){1};\node at (4, 5){2};\node at (5, 5){2};\node[circle, draw=black] at (0, 4){2};\node[circle, draw=black] at (1, 4){2};\node at (2, 4){3};\node[circle, draw=black] at (3, 4){1};\node[circle, draw=black] at (4, 4){2};\node[circle, draw=black] at (5, 4){2};\node at (0, 3){3};\node at (1, 3){4};\node[circle, draw=black] at (2, 3){2};\node at (3, 3){3};\node at (4, 3){3};\node[circle, draw=black] at (5, 3){1};\node[circle, draw=black] at (0, 2){1};\node[circle, draw=black] at (1, 2){3};\node at (2, 2){4};\node at (3, 2){4};\node[circle, draw=black] at (4, 2){2};\node at (5, 2){5};\node[circle, draw=black] at (0, 1){1};\node[circle, draw=black] at (1, 1){3};\node at (2, 1){5};\node[circle, draw=black] at (3, 1){4};\node at (4, 1){6};\node[circle, draw=black] at (5, 1){2};\end{tikzpicture} \\

\noindent Swap rows 2 and 3:

\begin{tikzpicture}\node at (0, 5){2};\node at (1, 5){2};\node at (2, 5){2};\node[circle, draw=black] at (3, 5){1};\node at (4, 5){2};\node at (5, 5){2};\node at (0, 4){3};\node[circle, draw=black] at (1, 4){2};\node at (2, 4){3};\node at (3, 4){3};\node at (4, 4){3};\node[circle, draw=black] at (5, 4){1};\node[circle, draw=black] at (0, 3){3};\node at (1, 3){4};\node[circle, draw=black] at (2, 3){2};\node[circle, draw=black] at (3, 3){3};\node[circle, draw=black] at (4, 3){3};\node[circle, draw=black] at (5, 3){1};\node[circle, draw=black] at (0, 2){1};\node[circle, draw=black] at (1, 2){3};\node at (2, 2){4};\node at (3, 2){4};\node[circle, draw=black] at (4, 2){2};\node at (5, 2){5};\node[circle, draw=black] at (0, 1){1};\node[circle, draw=black] at (1, 1){3};\node at (2, 1){5};\node[circle, draw=black] at (3, 1){4};\node at (4, 1){6};\node[circle, draw=black] at (5, 1){2};\end{tikzpicture} \\

\noindent Swap rows 3 and 4:

\begin{tikzpicture}\node at (0, 5){2};\node at (1, 5){2};\node at (2, 5){2};\node[circle, draw=black] at (3, 5){1};\node at (4, 5){2};\node at (5, 5){2};\node at (0, 4){3};\node[circle, draw=black] at (1, 4){2};\node at (2, 4){3};\node at (3, 4){3};\node at (4, 4){3};\node[circle, draw=black] at (5, 4){1};\node[circle, draw=black] at (0, 3){3};\node at (1, 3){4};\node at (2, 3){4};\node at (3, 3){4};\node[circle, draw=black] at (4, 3){2};\node[circle, draw=black] at (5, 3){1};\node[circle, draw=black] at (0, 2){1};\node[circle, draw=black] at (1, 2){3};\node[circle, draw=black] at (2, 2){4};\node[circle, draw=black] at (3, 2){4};\node[circle, draw=black] at (4, 2){2};\node at (5, 2){5};\node[circle, draw=black] at (0, 1){1};\node[circle, draw=black] at (1, 1){3};\node at (2, 1){5};\node[circle, draw=black] at (3, 1){4};\node at (4, 1){6};\node[circle, draw=black] at (5, 1){2};\end{tikzpicture} \\

\noindent Swap rows 4 and 5

\begin{tikzpicture}\node at (0, 5){2};\node at (1, 5){2};\node at (2, 5){2};\node[circle, draw=black] at (3, 5){1};\node at (4, 5){2};\node at (5, 5){2};\node at (0, 4){3};\node[circle, draw=black] at (1, 4){2};\node at (2, 4){3};\node at (3, 4){3};\node at (4, 4){3};\node[circle, draw=black] at (5, 4){1};\node[circle, draw=black] at (0, 3){3};\node at (1, 3){4};\node at (2, 3){4};\node at (3, 3){4};\node[circle, draw=black] at (4, 3){2};\node[circle, draw=black] at (5, 3){1};\node[circle, draw=black] at (0, 2){1};\node[circle, draw=black] at (1, 2){3};\node at (2, 2){5};\node[circle, draw=black] at (3, 2){4};\node[circle, draw=black] at (4, 2){2};\node at (5, 2){5};\node[circle, draw=black] at (0, 1){1};\node[circle, draw=black] at (1, 1){3};\node[circle, draw=black] at (2, 1){5};\node[circle, draw=black] at (3, 1){4};\node at (4, 1){6};\node[circle, draw=black] at (5, 1){2};\end{tikzpicture} \\



\bibliographystyle{alpha}
\bibliography{queue}{}
%\begin{thebibliography}{ABX}
%\bibitem{AAMP} Chikashi Arita, Arvind Ayyer, Kirone Mallick and Sylvain Prolhac, Recursive structures in the multispecies TASEP, J. Phys. A 44, 335004 (2011).
%\end{thebibliography}
\end{document}
