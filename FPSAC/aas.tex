\documentclass[submission]{FPSAC2018}
\usepackage{xcolor,listings,multicol}
\usepackage{rotating}
\usepackage[vcentermath]{youngtab}
\usepackage{enumerate}
\usepackage[all,cmtip]{xy}
\usepackage{tikz}
\usetikzlibrary{arrows,matrix}
\usepackage{comment}
\usepackage{color}
%\usepackage{subcaption}

%\usepackage[colorlinks=true, pdfstartview=FitV, linkcolor=blue, citecolor=blue, urlcolor=blue]{hyperref}

% use these commands for typesetting doi and arXiv references in the bibliography
%\newcommand{\doi}[1]{\href{http://dx.doi.org/#1}{\texttt{doi:#1}}}
\newcommand{\arxiv}[1]{\href{http://arxiv.org/abs/#1}{\texttt{arXiv:#1}}}

% For easier display of e-mail
\newcommand{\email}[1]{\href{mailto:#1}{#1}}

\newcommand{\iso}{\cong}
%\newcommand{\qedbox}{\rule{2mm}{2mm}}
%\renewcommand{\qedsymbol}{\qedbox}
\newcommand{\qbinom}[3]{\genfrac{[}{]}{0pt}{}{#1}{#2}_{#3}}
\newcommand{\absval}[1]{\left\lvert #1 \right\rvert}
\newcommand{\case}[1]{\vspace{12pt}\noindent\underline{#1}:}
\newcommand{\fs}{\mathcal{S}} % flagged Schur function
\newcommand{\mbf}{\mathbf}
\newcommand{\0}{\phantom{c}}

\DeclareMathOperator{\supp}{supp} % Support
\DeclareMathOperator{\inter}{int} % queuing interval
\DeclareMathOperator{\wt}{wt} % weight
\DeclareMathOperator{\pr}{pr} % promotion
\DeclareMathOperator{\id}{id} % identity
\DeclareMathOperator{\ch}{ch} % character
\DeclareMathOperator{\gr}{gr} % grading

\newcommand{\xx}{\mathbf{x}}
\newcommand{\mm}{\mathbf{m}}
\newcommand{\MLQ}{\mathbf{S}}

\newcommand{\mcA}{\mathcal{A}}
\newcommand{\mcF}{\mathcal{F}}
\newcommand{\mcM}{\mathcal{M}}
\newcommand{\mcW}{\mathcal{W}}

\newcommand{\ZZ}{\mathbb{Z}}
\newcommand{\QQ}{\mathbb{Q}}
\newcommand{\RR}{\mathbb{R}}
\newcommand{\CC}{\mathbb{C}}

\newcommand{\bze}{\overline{0}}
\newcommand{\bon}{\overline{1}}
\newcommand{\btw}{\overline{2}}
\newcommand{\bth}{\overline{3}}
\newcommand{\bfo}{\overline{4}}
\newcommand{\bfive}{\overline{5}}
\newcommand{\bsix}{\overline{6}}
\newcommand{\bseven}{\overline{7}}
\newcommand{\beight}{\overline{8}}
\newcommand{\bi}{\overline\imath}
\newcommand{\bk}{\overline{k}}
\newcommand{\brr}{\overline{r}}
\newcommand{\bn}{\overline{n}}
\newcommand{\ellbar}{\overline{\ell}}

\let\sumnonlimits\sum
\let\prodnonlimits\prod
\let\cupnonlimits\bigcup
\let\capnonlimits\bigcap
\renewcommand{\sum}{\sumnonlimits\limits}
\renewcommand{\prod}{\prodnonlimits\limits}
\renewcommand{\bigcup}{\cupnonlimits\limits}
\renewcommand{\bigcap}{\capnonlimits\limits}

\newenvironment{verlong}{}{}
\newenvironment{vershort}{}{}
\newenvironment{noncompile}{}{}
\excludecomment{verlong}
\includecomment{vershort}
\excludecomment{noncompile}
\newcommand{\rev}{\operatorname{rev}}
\newcommand{\conncomp}{\operatorname{conncomp}}
\newcommand{\NN}{\mathbb{N}}
\newcommand{\powset}[2][]{\ifthenelse{\equal{#2}{}}{\mathcal{P}\left(#1\right)}{\mathcal{P}_{#1}\left(#2\right)}}
% $\powset[k]{S}$ stands for the set of all $k$-element subsets of
% $S$. The argument $k$ is optional, and if not provided, the result
% is the whole powerset of $S$.
\newcommand{\set}[1]{\left\{ #1 \right\}}
% $\set{...}$ yields $\left\{ ... \right\}$.
\newcommand{\abs}[1]{\left| #1 \right|}
% $\abs{...}$ yields $\left| ... \right|$.
\newcommand{\tup}[1]{\left( #1 \right)}
% $\tup{...}$ yields $\left( ... \right)$.
\newcommand{\ive}[1]{\left[ #1 \right]}
% $\ive{...}$ yields $\left[ ... \right]$.
\newcommand{\verts}[1]{\operatorname{V}\left( #1 \right)}
% $\verts{...}$ yields $\operatorname{V}\left( ... \right)$.
\newcommand{\edges}[1]{\operatorname{E}\left( #1 \right)}
% $\edges{...}$ yields $\operatorname{E}\left( ... \right)$.
\newcommand{\arcs}[1]{\operatorname{A}\left( #1 \right)}
% $\arcs{...}$ yields $\operatorname{A}\left( ... \right)$.
\newcommand{\underbrack}[2]{\underbrace{#1}_{\substack{#2}}}
% $\underbrack{...1}{...2}$ yields
% $\underbrace{...1}_{\substack{...2}}$. This is useful for doing
% local rewriting transformations on mathematical expressions with
% justifications.
\newcommand{\mlnode}[1]{\node[circle, draw=black] at (#1){\phantom{c}};}

% Dark red emphasis
\definecolor{darkred}{rgb}{0.7,0,0} % darkred color
\newcommand{\defn}[1]{{\color{darkred}\emph{#1}}} % emphasis of a definition

%% For typesetting code listings                                                
\usepackage{listings}
\lstdefinelanguage{Sage}[]{Python}
{morekeywords={False,sage,True},sensitive=true}
\lstset{
  frame=single,
  showtabs=False,
  showspaces=False,
  showstringspaces=False,
  commentstyle={\ttfamily\color{dgreencolor}},
  keywordstyle={\ttfamily\color{dbluecolor}\bfseries},
  stringstyle={\ttfamily\color{dgraycolor}\bfseries},
  language=Sage,
  basicstyle={\footnotesize\ttfamily},
  aboveskip=0.75em,
  belowskip=0.75em,
  xleftmargin=.15in,
}
\definecolor{dblackcolor}{rgb}{0.0,0.0,0.0}
\definecolor{dbluecolor}{rgb}{0.01,0.02,0.7}
\definecolor{dgreencolor}{rgb}{0.2,0.4,0.0}
\definecolor{dgraycolor}{rgb}{0.30,0.3,0.30}
\newcommand{\dblue}{\color{dbluecolor}\bf}
\newcommand{\dred}{\color{dredcolor}\bf}
\newcommand{\dblack}{\color{dblackcolor}\bf}

\usepackage{xparse}

\makeatletter
% \specialmergetwolists{<coupler>}{<list1>}{<list2>}{<return macro>}
% \specialmergetwolists*{<coupler>}{<listcmd1>}{<listcmd2>}{<return macro>}
\protected\def\specialmergetwolists{%
  \begingroup
  \@ifstar{\def\cnta{1}\@specialmergetwolists}
    {\def\cnta{0}\@specialmergetwolists}%
}
\def\@specialmergetwolists#1#2#3#4{%
  \def\tempa##1##2{%
    \edef##2{%
      \ifnum\cnta=\@ne\else\expandafter\@firstoftwo\fi
      \unexpanded\expandafter{##1}%
    }%
  }%
  \tempa{#2}\tempb\tempa{#3}\tempa
  \def\cnta{0}\def#4{}%
  \foreach \x in \tempb{%
    \xdef\cnta{\the\numexpr\cnta+1}%
    \gdef\cntb{0}%
    \foreach \y in \tempa{%
      \xdef\cntb{\the\numexpr\cntb+1}%
      \ifnum\cntb=\cnta\relax
        \xdef#4{#4\ifx#4\empty\else,\fi\x#1\y}%
        \breakforeach
      \fi
    }%
  }%
  \endgroup
}
\makeatother

\theoremstyle{plain}
\newtheorem{thm}{Theorem}[section]
\newtheorem{lemma}[thm]{Lemma}
\newtheorem{conj}[thm]{Conjecture}
\newtheorem{prop}[thm]{Proposition}
\newtheorem{cor}[thm]{Corollary}
\theoremstyle{definition}
\newtheorem{dfn}[thm]{Definition}
\newtheorem{example}[thm]{Example}
\newtheorem{remark}[thm]{Remark}
\numberwithin{equation}{section}
%\numberwithin{figure}{section}
%\numberwithin{table}{section}
%\setcounter{section}{-1}

% For breaking equations across multiple pages
% \allowdisplaybreaks[1]

\usepackage[colorinlistoftodos]{todonotes}
\newcommand{\erik}[1]{\todo[size=\tiny,color=green!30]{#1 \\ \hfill --- Erik}}
\newcommand{\Erik}[1]{\todo[size=\tiny,inline,color=green!30]{#1
      \\ \hfill --- Erik}}
\newcommand{\darij}[1]{\todo[size=\tiny,color=red!30]{#1 \\ \hfill --- Darij}}
\newcommand{\Darij}[1]{\todo[size=\tiny,inline,color=red!30]{#1
      \\ \hfill --- Darij}}
\newcommand{\travis}[1]{\todo[size=\tiny,color=blue!30]{#1 \\ \hfill --- Travis}}
\newcommand{\Travis}[1]{\todo[size=\tiny,inline,color=blue!30]{#1
      \\ \hfill --- Travis}}

%%%%%%%%%%%%%%%%%%%%%%%%%%%%%%%%%%%%%%%%


\title[MLQs with special parameters]{Multiline queues with special parameters}

\author[E.~Aas]{Erik Aas\thanks{\email{eaas@kth.se}}\addressmark{1}
\and Darij Grinberg\thanks{\email{darij.grinberg@gmail.com}}\addressmark{2}
\and Travis Scrimshaw\thanks{\email{tcscrims@gmail.com}}\addressmark{3}}

\address{\addressmark{1}Department of Mathematics, Pennsylvania State University, McAllister Building, State College, PA 116802, USA
\\ \addressmark{2}School of Mathematics, University of Minnesota, 206 Church St. SE, Minneapolis, MN 55455, USA
\\ \addressmark{3}School of Mathematics and Physics, University of Queensland, St. Lucia, QLD 4072, Australia}

%\urladdr{http://www.cip.ifi.lmu.de/~grinberg/}
%\urladdr{https://sites.google.com/view/tscrim/home}

%% put the date of submission here
\received{\today}

%% leave this blank until submitting a revised version
%\revised{}

\abstract{
We prove the spectral parameter version of the MLQs conjecture.
}

%% put your French abstract here, or comment this out if you don't have one
%\resume{\lipsum[2]}


\keywords{multiline queue, flagged Schur function, TASEP}
%\subjclass[2010]{05E10, 17B37}

\usepackage[backend=bibtex]{biblatex}
\addbibresource{queue.bib}

\begin{document}

\maketitle

%=====================================================================
\section{Introduction}
\label{sec:introduction}

To be written last.

\Erik{Mention Gorodentsev's book.}









%=====================================================================
\section{Background}
\label{sec:background}

For a positive integer $N$, let $\ive{N}$ denote the set $\set{1, 2, \ldots, N}$.

%%%%%%%%%%
\subsection{Words and queues}

Fix a positive integer $n$.
Let $\mcW_n$ be the set of words $u = u_1 \dotsm u_n$ over the ordered alphabet $\mcA := \{1 < 2 < 3 < \cdots \}$.
We will consider the indices of words to be taken modulo $n$. 
The \defn{type} $\mm$ of a word $u$ is the vector $(m_1, m_2, \ldots)$ such that $u$ has $m_i$ occurrences of the letter $i$.
We sometimes refer to $u_i = t$ as a \defn{particle at $i$ of class $t$}.
Sometimes it is useful to consider only \defn{standard} words $u$ whose type $\mm$ is such that for some $r$, $m_i \neq 0$ for $1 \leq i \leq r$ and $m_i = 0$ for $i > r$.
We call $r$ the \defn{number of classes} in $u$.
We can \defn{merge} two adjacent classes $i,i+1$ in a standard word $u$ to obtain a new standard word as follows: first replace all occurrences of $i+1$ in $u$ by $i$, then replace all occurrences of $j$ in $u$ by $j-1$, for each $j > i$.

We define an \defn{$r$-queue} $q$ to be any subset of $[n]$ of size $r$. When $r$ is clear, we will simply call $q$ a \defn{queue}.
We think of $q$ as a function from $\mcW_n$ to itself as follows.
Let $\mm$ be the type of $u$.
Let $p_i = m_1 + m_2 + \cdots + m_i$.
There exists a unique $t$ such that
$
p_{t-1} \leq r < p_t.
$
The output word $v = q(u)$ will have type $(m_1, \dots, m_{t-1}, r-p_{t-1}, p_{t+1}-r, m_{t+2}, m_{t+3}, \ldots)$.
Note that $p_{t+1} - r = m_{t+1} + (p_t - r)$.
We think of this as splitting the class $t$ into two new classes $t$ and $t+1$. The following algorithm computes $v = q(u)$. In the start no letter of $v$ is set.

\begin{description}
\item[Phase I]
  Go through all $i$ such that $u_i > t$ in any order such that larger letters precede smaller ones.
  When considering a site $i$, find the first $j$ weakly to the left (cyclically) of $i$ such that $j \notin q$ and $v_j$ is not set. Then set $v_j = u_i + 1$. We call $\inter[j,i]$ the \defn{queueing interval} of $i$ with respect to $q$ and $u$.

\item[Phase II]
  Go through all $i$ such that $u_i < t$ in any order such that smaller letters precede larger ones.
  When considering a site $i$, find the first $j$ weakly to the right of $i$ such that $j \in q$ and $v_j$ is not set. Then set $v_j = u_i$. Similarly, $\inter[i,j]$ is called the \defn{queueing interval} of $i$ with respect to $q$ and $u$.

\item[Phase III]
  At this point, there are $m_t$ unset values $v_i$. For such $i$, set $v_i = t$ for $i \in q$ and $v_i = t+1$ for $i\notin q$.
\end{description}

\begin{example}
We consider the $4$-queue $q = \{1, 4, 8, 9\}$ and consider the word $u = 346613321$. Thus, the type of $u$ is $\mm = (2, 1, 3, 1, 0, 2, 0, \ldots)$ with $p_2 = 3$ and $p_3 = 6$, and so $t = 3$. To compute $q(w)$:
\[
\begin{tikzpicture}[scale=1.3]
\node (i1) at (1,1) {$3$};
\node (i2) at (2,1) {$4$};
\node (i3) at (3,1) {$6$};
\node (i4) at (4,1) {$6$};
\node (i5) at (5,1) {$1$};
\node (i6) at (6,1) {$3$};
\node (i7) at (7,1) {$3$};
\node (i8) at (8,1) {$2$};
\node (i9) at (9,1) {$1$};
\node (t1) at (1,-1) {$2$};
\node (t2) at (2,-1) {$6$};
\node (t3) at (3,-1) {$6$};
\node (t4) at (4,-1) {$3$};
\node (t5) at (5,-1) {$4$};
\node (t6) at (6,-1) {$4$};
\node (t7) at (7,-1) {$4$};
\node (t8) at (8,-1) {$1$};
\node (t9) at (9,-1) {$1$};
\node[circle,draw=black] (q1) at (1,0) {};
\node[circle,draw=black] (q2) at (4,0) {};
\node[circle,draw=black] (q3) at (8,0) {};
\node[circle,draw=black] (q4) at (9,0) {};
\draw[->,red] (i4) -- (q2);
\draw[->,red] (q2) -- (2,0) -- (t2);
\draw[->,red] (i3) -- (t3);
\draw[->,red] (i2) -- (2,0.1) -- (q1);
\draw[->,red] (q1) -- (0,0);
\draw[>->,red] (10,0) -- (q4);
\draw[->,red] (q4) -- (q3);
\draw[->,red] (q3) -- (7,0) -- (t7);
\draw[->,blue] (i5) -- (5,0.1) .. controls (7.5,0.1) .. (q3);
\draw[->,blue] (q3) -- (t8);
\draw[->,blue] (i9) -- (q4);
\draw[->,blue] (q4) -- (t9);
\draw[->,blue] (i8) -- (8,0.2) .. controls (9.5,0.2) .. (10,0.1);
\draw[>->,blue] (0,0.1) -- (q1);
\draw[->,blue] (q1) -- (t1);
\end{tikzpicture}
\]
where the arrows in red are Phase I and the blue are Phase II. Hence, we have $q(346613321) = 266344411$, which has type $(2,1,1,3,0,2,0,\ldots)$.
\end{example}

Since Phase I only deals with $j \notin q$, and Phase II only with $j\in q$, these two phases commute.
We illustrate the situation $v = q(u)$ with a $2 \times n$ array where the first row is the word $u$, and second row has a circle labelled $v_j$ for $j \in q$ or a square labelled $v_j$ for $j \notin q$ in position $j$.
See Figure~1\travis{This needs to be copied over and become an explicit reference} for an example.

Consider a pair $k, k+1 \pmod{n}$ of consecutive columns.
For $s > t$ the \defn{$s$-flow} from $k+1$ to $k$ is the number of $i$ such that $u_i=s$, and whose queueing interval $\inter[j,i]$ contains both $k$ and $k+1$.
Similarly, for $s < t$, the \defn{$s$-flow} from $k$ to $k+1$ is the number of $i$ such that $u_i = s$, and whose queueing interval $\inter[i,j]$ contains both $k$ and $k+1$.

There is an obvious duality in the definition of the labelling process above.
Due to annoying notation, we formulate it as the following lemma (whose proof is straightforward).

\begin{lemma}[Duality]
  Let $q$ be a queue and $u$ be a standard word with $r$ classes. Define a new word $v$ by letting $v_i = r + 1 - u_{n+1-i}$ and a new queue $q'$ by letting $i \in q' \Leftrightarrow n+1-i \notin q$.
 Then $q(u)_i = r + 2 - q'(v)_{n+1-i}$.
\end{lemma}

%\begin{remark} (Excision)
%  Fix a queue $q$ and a word $u$, and let $i,j$ be columns. Suppose that the $s$-flow from $j+1$ to $j$ equals the $s$-flow from $i+1$ to $i$ for each $s > t$, and that the $s$-flow from $i$ to $i+1$ equals the $s$-flow from $j$ to $j+1$ for each $s < t$. Then we have $q_{|\inter[i,j]^c}(u_{|\inter[i,j]^c}) = (q(u))_{|\inter[i,j]^c}$. Here, for a (cyclic) word $u$, we let $u_{|\inter[i,j]^c}$ denote the (cyclic) word gotten from simply removing the closed (cyclic) interval $\inter[i,j]$, and similarly for $q_{|\inter[i,j]^c}$.
%\end{remark}

\begin{lemma}[Monotonicity]
\label{le:mono}
  For any $t \in \ZZ_{\geq 1}$, let $f_t \colon \{1,2, \ldots\} \to \{1,2\}$ be given by $f_t(x) = 1$ for $x \leq t$ and $f_t(x) = 2$ for $x > t$. Let $q$ be a queue, $u$ be any word, and $i,j\in[n]$. Then the comparison $q(f_t(u))_i \leq q(f_t(u))_j$ has the same result for all $t$.
%and it is the same result as q(u)_i \leq q(u)_j
\end{lemma}

Lemma~\ref{le:mono} tells us that $q$, considered as a function on words, is completely determined by its values $q(u)$ on words $u\in\{1,2\}^n$.

%%%%%%%%%%
\subsection{Multiline queues}

%\begin{dfn} \label{def.mlqs.mlq}
Fix a nonnegative integer $n$.
A \defn{multiline queue (MLQ)} is an $(n+1)$-tuple $(S_0, S_1, \dotsc, S_n)$ such that $S_0 = \emptyset$ and, for all $1 \leq i \leq n$, we have $S_i \subseteq \ive{N}$ and $\lvert S_{i-1} \rvert < \lvert S_i \rvert$.
We shall represent an MLQ $\tup{S_0, S_1, \ldots, S_n}$ as an
$n \times N$-array, each row of which has some of its boxes
marked. Namely, for each $i \in \set{1, 2, \ldots, n}$ and
each $j \in S_i$, we mark the $j$-th box in row $i$.
(The rows and the columns are indexed as usual for a matrix.)
%\end{dfn}

For example, $\tup{\varnothing, \set{2, 4}, \set{1, 4, 5}}$
is an MLQ for $n = 2$ and $N = 6$; we draw it as follows:
\[
\begin{tikzpicture}[scale=0.7, every node/.style={scale=0.7}]
  \mlnode{2, 2}
  \mlnode{4, 2}
  \mlnode{1, 1}
  \mlnode{4, 1}
  \mlnode{5, 1}
\end{tikzpicture}
\]

\begin{remark}
Our notion of an MLQ corresponds to what is called a ``discrete MLQ'' in~\cite[\S 2.2]{AasLin17}.
We omit the word ``discrete'', as we will not encounter any other kind of MLQs.
\end{remark}

Let $\MLQ = (S_0, \dotsc, S_n)$ be an MLQ.
The \defn{type} of $\MLQ$ is the  $n$-tuple $\mm = \tup{m_1, m_2, \ldots, m_n}$ of
positive
% positive, not just nonnegative, since we required the S_i to strictly grow in size
integers such that $\abs{S_i} = m_1 + m_2 + \cdots + m_i$.
(In particular, $S_0 = \varnothing$.)

In order to define the labeling of an MLQ, we need to setup some notation.

If $p$ is an element of $\ive{N}$,
and if $S$ is a nonempty subset of $\ive{N}$,
then an element $\min_p S$ of $S$ is defined as follows:
If there exists an element of $S$ that is greater
or equal to $p$, then we set $\min_p S$ to be the
smallest such element;
otherwise, we set $\min_p S$ to be the smallest
element of $S$.
(Roughly speaking, $\min_p S$ is the first element
of $S$ you encounter when traversing the elements
$p, p+1, \ldots, n, 1, 2, \ldots, p-1$.)

Let $U$ and $V$ be two subsets of $\ive{N}$ satisfying
$\abs{U} \leq \abs{V}$.
Let $f \colon U \to \ive{n}$ be any map; we shall
refer to it as a ``labeling'', and we shall refer an
image $f \tup{u}$ as the ``label'' of $u$ in $f$.
Let $k$ be a positive integer.
We define the map $f \downarrow_{V, k} : V \to \ive{n}$
as follows (recursively over $\abs{U}$):

\begin{itemize}
 \item If $U = \varnothing$, then $f \downarrow_{V, k}$ sends
       each element of $V$ to $k$.
 \item Otherwise, pick an element $u$ of $U$ having minimum
       label (i.e., having minimum $f \tup{u}$).
       Let $v = \min_u V$.
       Let $g$ be the restriction of $f$ to the subset
       $U \setminus \set{u}$.
       Let $g' = g \downarrow_{V \setminus \set{v}, k}$.
       Then, the labeling $f \downarrow_{V, k}$ is defined by
       setting
       \[
        \tup{f \downarrow_{V, k}} \tup{s}
        =
        \begin{cases}
         f \tup{u}, & \text{ if } s = v; \\
         g \tup{s}, & \text{ if } s \neq v
        \end{cases}
        \qquad \text{ for all } s \in V .
       \]
\end{itemize}

\darij{Remember to fix the spelling of ``labelling''.}

\begin{prop} \label{prop.mlqs.down.indep}
Let $U$ and $V$ be two subsets of $\ive{N}$ satisfying $\abs{U} \leq \abs{V}$.
Let $f \colon U \to \ive{n}$.
Let $k$ be a positive integer.
The definition of $f \downarrow_{V, k}$ given above does not
depend on the choice of $u$.
\end{prop}

\begin{proof}
We can replace this definition by an equivalent one, which
clearly does not depend on any choices:

For any $p \in \ive{N}$, we define a subset
$\operatorname{arc}\ive{p : s}$ of $\ive{N}$ by
\[
\operatorname{arc}\ive{p : s}
=
\begin{cases}
\set{p, p+1, \ldots, s}, & \text{ if } p \leq s; \\
\set{p, p+1, \ldots, N, 1, 2, \ldots, s}, & \text{ if } p > s
\end{cases} .
\]

For any $0 \leq i \leq n$, we let $f^{-1} \tup{\leq i}$ be
the subset
$f^{-1} \tup{\ive{i}} = \set{u \in U \mid f \tup{u} \leq i}$
of $U$.

Let $s \in V$.
If there exists some $i \in \ive{n}$ and some
$p \in \ive{N}$ such that the set
$\operatorname{arc}\ive{p : s}$
contains at least as many elements of $f^{-1} \tup{\leq i}$
as it contains elements of $V$.
(that is, it satisfies
\[
\abs{\operatorname{arc}\ive{p : s} \cap f^{-1} \tup{\leq i}}
\geq \abs{\operatorname{arc}\ive{p : s} \cap V}
\]
),
then we pick the smallest $i$ for which this holds, and we set $\tup{f \downarrow_{V, k}}\tup{s} = i$.
Otherwise, we set $\tup{f \downarrow_{V, k}}\tup{s} = k$.

[The condition
``$\abs{\operatorname{arc}\ive{p : s} \cap f^{-1} \tup{\leq i}}
\geq \abs{\operatorname{arc}\ive{p : s} \cap V}$
for some $p \in \ive{N}$''
can also be restated as follows:
If we define a circular word on $N$ letters whose $q$-th letter
is
\[
\begin{cases}
(, & \text{ if } q \in f^{-1} \tup{\leq i} \setminus V, \\
), & \text{ if } q \in V \setminus f^{-1} \tup{\leq i}, \\
0, & \text{ otherwise},
\end{cases}
\]
and if we match parentheses as usual for a circular word
(this means, in particular, that the last $($ can match the
first $)$),
then the $s$-th letter is either $0$ or a matched parenthesis.
This is strongly reminiscent of the Remmel-Shimozono proof
of the LR rule, and suggests that something coplactic is
going on.]
\Travis{It is not surprising to me that something coplactic is running around in the background because of the $R$-matrix interpretation by Kuniba and co. Coplactic operators are crystal operators and $R$-matrices are crystal isomorphisms. We might be able to get the proof for free by appealing to that perspective.}

......
\end{proof}

\begin{dfn}
The \defn{labeling} of $\MLQ$ is the tuple $(f_0, f_1, \dotsc, f_n)$ given recursively as follows.
Define $f_0 \colon S_0 \to \ive{n}$ in the obvious way (since $S_0 = \varnothing$) and then define $f_i$ by $f_i = f_{i-1} \downarrow_{S_i, i}$.

We define the \defn{bottom row labels} of an MLQ $\MLQ$ with $S_n = (b_1 < b_2 < \cdots < b_n)$ to be $\tup{f_n(b_1), f_n(b_2), \dotsc, f_n(b_n)}$.
\end{dfn}

\begin{dfn} \label{def.mlq.sw}
Let $\MLQ = \tup{S_0, S_1, \ldots, S_n}$ be an MLQ of type $\mm$.
The \defn{spectral weight} $\wt\tup{\MLQ}$
of $\MLQ$ is defined to be the monomial
\[
 \prod_{i=0}^{n-1} \prod_{s \in S_i} x_s
\]
in the indeterminates $x_1, x_2, \ldots, x_N$.
\end{dfn}


%\subsection{Flagged Schur functions}
%
%Let $\lambda$ be a partition of length $m$ and $b = (b_1 \leq b_2 \leq \cdots \leq b_m)$.
%Let
%\[
%h_d(k) = \sum_{i_1 \leq \cdots \leq i_d \leq k} x_{i_1} \cdots x_{i_d}
%\]
%be the complete homogeneous symmetric function of degree $d$ in the variables $x_1, \dotsc, x_k$.
%A \defn{flagged Schur function} is
%\[
%\fs_{\lambda}(b; \xx) = \det\big[ h_{\lambda_i - i + j}(b_i) \bigr]_{i,j=1}^m.
%\]
%We can also express the flagged Schur function combinatorially using $\mcF_{\lambda}(b)$, the semistandard tableaux of shape $\lambda$ such that the max entry in row $i$ is at most $b_i$.
%From~\cite{Wachs85}, we have
%\[
%\fs_{\lambda}(b; \xx) = \sum_{T \in \mcF_{\lambda}(b)} x^T.
%\]

%=====================================================================
\section{Main result}
\label{sec:result}


Fix a subset $B = \set{b_1 < b_2 < \cdots < b_n}$ of $\ive{N}$.
Let $\mcM \tup{B, \mm}$ be the set of all MLQs $\MLQ = (S_0, \dotsc, S_n)$ of type $\mm$
with $S_n = X$ and bottom row labels
\[
(\underbrace{n, \dotsc, n}_{m_n},\, \dotsc,\, \underbrace{2, \dotsc, 2}_{m_2},\, \underbrace{1, \dotsc, 1}_{m_1}).
\]

For a standard word $u$ of type $\mm = (m_1, \dotsc, m_r)$, we define the \defn{amplitude} 
\[
  [u] = \sum_{(q_1, \dotsc, q_{r-1})} \prod_{i=1} ^{r-1} \prod_{j \in q_i} x_j,
\]
where we sum over all sequences $q_1, \dots, q_{r-1}$ of queues such that
\begin{enumerate}
\item $q_i$ is a $(m_1+\dots+m_i)$-queue, and
\item $u = q_{r-1}(\cdots q_2(q_1(1\dotsm 1)) \cdots)$.
\end{enumerate}

Such a sequence $(q_1, \dotsc, q_{r-1}$ is called an (ordinary) \defn{multiline queue} (MLQ) of type $\mm$.

The following result is proved in~\cite{AAMP} in the case $x_1 = \cdots = x_r$. \erik{our proof is different[...]}

\begin{thm}
\label{thm:permutation}
  For any permutation $\sigma$ of $[r-1]$, we have 
\[
  [u] = \sum_{(q_1, \dots, q_{r-1})} \prod_{i=1} ^{r-1} \prod_{j\in q_i} x_j,
\]
where we sum over all sequences $q_1, \dotsc, q_{r-1}$ of queues such that
\begin{enumerate}
\item $q_i$ is a $(m_1 + \cdots + m_{\sigma_i})$-queue, and
\item $u = q_{r-1}(\cdots q_2(q_1(1\dotsm 1)) \cdots )$.
\end{enumerate}
\end{thm}

\begin{cor}
  Let $\mm$ be a standard type and $\mm'$ be the result of merging the classes $i$ and $i+1$ in $\mm$. For any word $v$ of type $m$, we have
\[
  [v] = e_{m_1+\dots+m_i}(x_1, \dots, x_n) \sum_u [u],
\]
where we sum over all $u$ of type $\mm$ such that merging classes $i$ and $i+1$ in $u$ yields $v$.
\end{cor}

\begin{proof}
  Note that if $u = q_{r-1}(\dots q_2(q_1(1 \dots 1)) \dots)$ where $q_1$ is a $m_1+\dots+m_i$-queue, then $v = q_{r-1}(\dots q_2(1\dots 1)\dots)$ is the result of merging classes $i$ and $i+1$ in $u$. The result follows from Theorem~\ref{thm:permutation}.
\end{proof}

\begin{thm}
\label{thm:determinant_form}
  Let $1 \leq b_1 < \dots < b_k \leq n$ be integers and $v_1v_2 \dots v_k$ be a weakly decreasing (non-cyclic) standard word of length $k$ with $r-1$ classes. Define a word $u$ of length $n$ as follows,
  $u_i = \left\{
%\begin{array}{cc}
%  v_k & \textrm{ if }i = b_k\textrm{ for some }k, \\
%  r & \textrm{ otherwise}.
%\end{array}
 \right.$

Then $[u] = \det(h_{i-j-1+\gamma_j}(x_1, \dots, x_{b_j}))_{1\leq i,j\leq n}$, where $\gamma_i$ is the number of distinct letters in $v_1\dots v_j$. We define $h_d(\dots) = 0$ for $d < 0$.
\end{thm}

\begin{figure}
\[
\begin{tikzpicture}[xscale=1.5]
  \node[circle,draw=black] at  (1, 0){3};
  \node[circle,draw=black] at  (2, 0){3};
  \node[circle,draw=black] at  (4, 0){2};
  \node[circle,draw=black] at  (6, 0){2};
  \node[circle,draw=black] at  (7, 0){2};
  \node[circle,draw=black] at  (9, 0){1};
  \node[circle,draw=black] at  (3, 1){2};
  \node[circle,draw=black] at  (4, 1){2};
  \node[circle,draw=black] at  (6, 1){2};
  \node[circle,draw=black] at  (8, 1){1};
  \node[circle,draw=black] at  (7, 2){1};
\end{tikzpicture}
\]
\[
\begin{tikzpicture}[xscale=1.5]
  \draw[densely dotted] (1, 1) -- (9, 1);
  \draw[densely dotted] (1, 2) -- (9, 2);
  \draw[densely dotted] (1, 3) -- (9, 3);
  \draw[densely dotted] (1, 4) -- (9, 4);
  \draw[densely dotted] (1, 5) -- (9, 5);
  \draw[densely dotted] (1, 6) -- (9, 6);

  \draw[densely dotted] (1, 6) -- (1, 1);
  \draw[densely dotted] (2, 6) -- (2, 1);
  \draw[densely dotted] (3, 6) -- (3, 1);
  \draw[densely dotted] (4, 6) -- (4, 1);
  \draw[densely dotted] (5, 6) -- (5, 1);
  \draw[densely dotted] (6, 6) -- (6, 1);
  \draw[densely dotted] (7, 6) -- (7, 1);
  \draw[densely dotted] (8, 6) -- (8, 1);
  \draw[densely dotted] (9, 6) -- (9, 1);

  \draw[draw=red] (1,2)--(2,2);
  \draw[draw=red] (1,3)--(3,3)--(3,2)--(4,2);
  \draw[draw=red] (1,4)--(4,4)--(4,3)--(6,3);
  \draw[draw=red] (1,5)--(6,5)--(6,4)--(7,4);
  \draw[draw=red] (1,6)--(7,6)--(7,5)--(8,5)--(8,4)--(9,4);

  \node at (1,1){\textbullet};
  \node at (1,2){\textbullet};
  \node at (1,3){\textbullet};
  \node at (1,4){\textbullet};
  \node at (1,5){\textbullet};
  \node at (1,6){\textbullet};
  \node at (2,2){\textbullet};
  \node at (4,2){\textbullet};
  \node at (6,3){\textbullet};
  \node at (7,4){\textbullet};
  \node at (9,4){\textbullet};
\end{tikzpicture}
\]
\caption{Here, $n = 14, r = 3, v = 3211$.}
\end{figure}

Now, fix a sequence $b_1 < b_2 < \dots < b_k$, and for a permutation $v$ of $[k]$ let $u(v)$ be the corresponding word as defined in Lemma BLAH. Furthermore let $S \subseteq [k-1]$ be such that $i\in S \Rightarrow i+1 \notin S$. In [AL], a formula for the amplitude $\left[ u\bigl( (\prod_{i\in S} s_i) w_0 \bigr) \right]$ is conjectured, where $w_0$ is the reverse permutation on $[k]$; $w_0 = k(k-1)\dots 1$.

Let $\varphi(S) = \left[ ( \prod_{i\in S} s_i ) w_0 \right]$, and $\psi(T) = \sum_{S \subseteq T} \varphi(S)$. By Lemma Y, $\psi(T) = \dots$. (Apply M\"obius inversion and be done.)









%=====================================================================
\section{Concluding (starting?) remarks}

It seems our three lemmas effectively compute $C(u)$ for $u \in \{1,2\}^n$. Can this be done in a more transparent manner (pairs of labelled periodic Dyck paths come to mind)?

\vspace{10cm} 

\begin{figure}
\[
\begin{tikzpicture}
  \node at (0, 3){1};
  \node at (1, 3){2};
  \node at (2, 3){2};
  \node at (3, 3){2};
  \node at (4, 3){1};
  \node at (5, 3){2};
  \node at (6, 3){2};
  \node at (7, 3){2};

  \node[circle, draw=black] at (0,2){\0};
  \node[circle, draw=black] at (1,2){\0};
  \node[circle, draw=black] at (2,2){\0};
  \node[circle, draw=black] at (3,2){\0};
  \node[circle, draw=black] at (4,2){\0};
  \node[rectangle, draw=black] at (5,2){\0};
  \node[rectangle, draw=black] at (6,2){\0};
  \node[circle, draw=black] at (7,2){\0};

  \node at (0,5){\textrm{unbalanced}};
  \draw (3.5,1)--(3.5,3);
  \node at (5,5){\textrm{balanced}};
  \node at (0,4){i};
  \node at (7,4){j};
\end{tikzpicture}\]
\caption{A typical situation in Lemma~\ref{le:UB} (when there is no balanced block contained in $\inter(i,j)$). Here $k=3$. In the proof, we consider the case when the circles at $i$ and $i+1$ are labelled $(1,1)$ and when they are labelled $(1,2)$ separately.}
\end{figure}

\begin{figure}
\[
\begin{tikzpicture}{scale=0.9}
  \node at (0,2){3};
  \node at (1,2){2};
  \node at (2,2){1};
  \node at (3,2){3};
  \node at (4,2){2};
  \node at (5,2){2};
  \node at (6,2){1};

  \node[circle,draw=black]    at (0, 1){1};
  \node[rectangle,draw=black] at (1, 1){3};
  \node[circle,draw=black]    at (2, 1){1};
  \node[rectangle,draw=black] at (3, 1){4};
  \node[circle,draw=black]    at (4, 1){2};
  \node[circle,draw=black]    at (5, 1){2};
  \node[rectangle,draw=black] at (6, 1){4};
\end{tikzpicture}
\]
\caption{Here, $n = 7$, $r=4$, $q = \{1,3,5,6\}$ and $u = 3213221$ then $v = q(u) = 1314224$.}
\end{figure}

\begin{figure}
\[
\begin{tikzpicture}
  \node at (0,6){i};
  \node at (5,6){j};
  \node at(-2,5){u:};
  \node at(-2,4){$q_1$:};
  \node at(-1,5){$\dots$};
  \node at(-1,4){$\dots$};
  \node at (6,5){$\dots$};
  \node at (6,4){$\dots$};
  \node at (0,5){1};
  \node at (1,5){2};
  \node at (2,5){1};
  \node at (3,5){1};
  \node at (4,5){2};
  \node at (5,5){2};
  \node[rectangle,draw=black] at (0, 4){3};
  \node[rectangle,draw=black] at (1, 4){3};
  \node[circle,draw=black]    at (2, 4){1};
  \node[circle,draw=black]    at (3, 4){1};
  \node[rectangle,draw=black] at (4, 4){3};
  \node[circle,draw=black]    at (5, 4){1};

  \node at(-2,2){$u_{i\leftrightarrow j}$:};
  \node at(-2,1){$q_1$:};
  \node at(-1,2){$\dots$};
  \node at(-1,1){$\dots$};
  \node at (6,2){$\dots$};
  \node at (6,1){$\dots$};
  \node at (0,2){2};
  \node at (1,2){2};
  \node at (2,2){1};
  \node at (3,2){1};
  \node at (4,2){2};
  \node at (5,2){1};
  \node[rectangle,draw=black] at (0, 1){3};
  \node[rectangle,draw=black] at (1, 1){3};
  \node[circle,draw=black]    at (2, 1){1};
  \node[circle,draw=black]    at (3, 1){1};
  \node[rectangle,draw=black] at (4, 1){3};
  \node[circle,draw=black]    at (5, 1){1};
\end{tikzpicture}
\]
\caption{A BB-swap.}
\end{figure}

\begin{figure}
\[
\begin{tikzpicture}
\draw (-0.5,6)--(-0.5,3);
\draw (5.5,6)--(5.5,3);
  \node at(-2,6){2};
  \node at(-1,6){2};
  \node at (0,6){1};
  \node at (1,6){1};
  \node at (2,6){2};
  \node at (3,6){2};
  \node at (4,6){2};
  \node at (5,6){2};
  \node at (6,6){1};
  \node at (7,6){2};
  \node[circle,draw=black] at (0, 5){\0};
  \node[rectangle,draw=black] at (1, 5){\0};
  \node[circle,draw=black] at (2, 5){\0};
  \node[rectangle, draw=black] at (3, 5){\0};
  \node[circle,draw=black] at (4, 5){\0};
  \node[rectangle,draw=black] at (5, 5){\0};
  \node[rectangle,draw=black] at (0, 4){\0};
  \node[circle, draw=black] at (1, 4){\0};
  \node[rectangle,draw=black] at (2, 4){\0};
  \node[rectangle,draw=black] at (3, 4){\0};
  \node[circle,draw=black] at (4, 4){\0};
  \node[circle, draw=black] at (5, 4){\0};
  \node[circle,draw=black] at (-1, 5){\0};
  \node[circle,draw=black] at (-2, 5){\0};
  \node at (-3,5){\dots};
  \node[circle,draw=black] at (6, 5){\0};
  \node[circle,draw=black] at (7, 5){\0};
  \node at (8,5){\dots};
\end{tikzpicture}
\]
\[
\begin{tikzpicture}
  \node at(-2,6){2};
  \node at(-1,6){2};
  \node at (0,6){1};
  \node at (1,6){1};
  \node at (2,6){2};
  \node at (3,6){2};
  \node at (4,6){2};
  \node at (5,6){2};
  \node at (6,6){1};
  \node at (7,6){2};

\draw (-0.5,6)--(-0.5,3);
\draw (5.5,6)--(5.5,3);
  \node[circle,draw=black] at (0, 5){\0};
  \node[rectangle,draw=black] at (1, 5){\0};
  \node[circle,draw=black] at (2, 5){\0};
  \node[rectangle, draw=black] at (3, 5){\0};
  \node[circle,draw=black] at (4, 5){\0};
  \node[rectangle,draw=black] at (5, 5){\0};

  \node[rectangle,draw=black] at (0, 4){\0};
  \node[circle, draw=black] at (1, 4){\0};
  \node[rectangle,draw=black] at (2, 4){\0};
  \node[rectangle,draw=black] at (3, 4){\0};
  \node[circle,draw=black] at (4, 4){\0};
  \node[circle, draw=black] at (5, 4){\0};
  
  \node[rectangle,draw=black] at (-1, 5){\0};
  \node[rectangle,draw=black] at (-2, 5){\0};
  \node at (-3,5){$\dotsm$};

  \node[rectangle,draw=black] at (6, 5){\0};
  \node[rectangle,draw=black] at (7, 5){\0};
  \node at (8,5){$\dotsm$};
\end{tikzpicture}
\]
\caption{A balanced block in $C_1$, together with an input word $u = \dotsm 2211222212 \dotsm$.}
\end{figure}









%=====================================================================
\section{The projection formula}

For the homogenous case, where all spectral parameters equal, it is well known that the number of multiline queues with a given bottom row is proportional to the stationary distribution of a certain Markov chain, the multi-species TASEP on a ring, at the state corresponding to that bottom row. The TASEP has a property of merging classes that translates to a statement about multiline queues, usually called the {\it projection principle}. Somewhat surprisingly, this principle has a generalisation to our case with non-equal spectral parameters. We now state this generalisation.

\begin{thm}
  Let $u$ be a word of type $\mm$, and let $v$ be the word obtained from $u$ by merging classes $k$ and $k+1$.
  Then $\sum_{u'} \wt{u} = e_{m_1+\dots+m_k}(x) \wt{v}$,
  where $u'$ runs over all words gotten by permuting $k$ and $k+1$ in $u$.
\end{thm}

This theorem will be a corollary of Theorem [??].


\begin{example}
Looking at queues, we get
\begin{align*}
[13234] & = x_1 x_2 x_3^2 x_4 (x_1^2 + x_1 x_4 + x_1 x_5 + x_4 x_5 + x_5^2)
\\ [13245] & = x_1 x_2 x_3^2 x_4 (x_1^2 + x_1x_4 + x_1x_5 + x_4^2 + x_4x_5 + x_5^2)
\\ & \hspace{20pt} \times (x_1x_2x_3 + x_1x_2x_5+x_1x_3x_5+x_2x_3x_5)
\\ [14235] & = x_1x_2x_3^2x_4^2 (x_1^3x_2 + x_1^3x_3 + x_1^3x_5 + x_1^2x_2x_3 + x_1^2x_2x_4 + 2x_1^2x_2x_5 + x_1^2x_3x_4
\\ & \hspace{55pt} + 2x_1^2x_3x_5 + x_1^2x_4x_5 + x_1^2x_5^2 + x_1x_2x_3x_5 + x_1x_2x_4x_5 + 2x_1x_2x_5^2
\\ & \hspace{55pt} + x_1x_3x_4x_5 + 2x_1x_3x_5^2 + x_1x_4x_5^2 + x_1x_5^3 + x_2x_3x_5^2 + x_2x_4x_5^2
\\ & \hspace{55pt} + x_2x_5^3 + x_3x_4x_5^2 + x_3x_5^3)
\end{align*}
(We have factored the expressions for readability only.)

The projection formula states that $[13234] e_3(x) = [13245] + [14235]$, as we can check directly in this case.
\end{example}


\begin{figure}
\label{fig_generalized_queue}
\[
\begin{tikzpicture}
  \node at  (1,2){1};
  \node at  (2,2){3};
  \node at  (3,2){2};
  \node at  (4,2){3};
  \node at  (5,2){3};
  \node at  (6,2){2};
  \node at  (7,2){2};
  \node at  (8,2){3};
  \node at  (9,2){1};
  \node at (10,2){2};
  \node at (11,2){1};

  \node at                     (1,1){4};
  \node[circle,draw=black] at  (2,1){1};
  \node at                     (3,1){4};
  \node[circle,draw=black] at  (4,1){1};
  \node[circle,draw=black] at  (5,1){2};
  \node at                     (6,1){3};
  \node at                     (7,1){4};
  \node[circle,draw=black] at  (8,1){2};
  \node at                     (9,1){3};
  \node[circle,draw=black] at (10,1){1};
  \node at                    (11,1){4};
\end{tikzpicture}
\]
\caption{A generalized queue showing that $v = Q_S(u)$ where $S = \{2,4,5,8,10\}$, $u = 13233223121$ and $v =41412342314$. It has weight $x_2x_4x_5x_8x_{10}$.}
\end{figure}


{\bf Example.} 

$(S_1, S_2)$:

\scalebox{0.4}{
\begin{tikzpicture}
  \node[circle, draw=black] at (01, 2){\phantom{c}};
  \node[circle, draw=black] at (02, 2){\phantom{c}};
  \node[circle, draw=black] at (03, 2){\phantom{c}};
  \node[circle, draw=black] at (06, 2){\phantom{c}};
  \node[circle, draw=black] at (07, 2){\phantom{c}};
  \node[circle, draw=black] at (08, 2){\phantom{c}};
  \node[circle, draw=black] at (09, 2){\phantom{c}};
  \node[circle, draw=black] at (11, 2){\phantom{c}};
  \node[circle, draw=black] at (13, 2){\phantom{c}};
  \node[circle, draw=black] at (18, 2){\phantom{c}};
  \node[circle, draw=black] at (19, 2){\phantom{c}};
  \node[circle, draw=black] at (22, 2){\phantom{c}};
  \node[circle, draw=black] at (23, 2){\phantom{c}};
  \node[circle, draw=black] at (24, 2){\phantom{c}};
  \node[circle, draw=black] at (25, 2){\phantom{c}};
  \node[circle, draw=black] at (26, 2){\phantom{c}};

  \node[circle, draw=black] at (02, 1){\phantom{c}};
  \node[circle, draw=black] at (04, 1){\phantom{c}};
  \node[circle, draw=black] at (05, 1){\phantom{c}};
  \node[circle, draw=black] at (06, 1){\phantom{c}};
  \node[circle, draw=black] at (08, 1){\phantom{c}};
  \node[circle, draw=black] at (10, 1){\phantom{c}};
  \node[circle, draw=black] at (11, 1){\phantom{c}};
  \node[circle, draw=black] at (12, 1){\phantom{c}};
  \node[circle, draw=black] at (14, 1){\phantom{c}};
  \node[circle, draw=black] at (16, 1){\phantom{c}};
  \node[circle, draw=black] at (17, 1){\phantom{c}};
  \node[circle, draw=black] at (19, 1){\phantom{c}};
  \node[circle, draw=black] at (20, 1){\phantom{c}};
  \node[circle, draw=black] at (21, 2){u};
  \node[circle, draw=black] at (22, 1){\phantom{c}};
  \node[circle, draw=black] at (24, 1){\phantom{c}};
  \node[circle, draw=black] at (26, 1){\phantom{c}};
\end{tikzpicture}
}

\vspace{2cm}

$(T_1,T_2)$:

\scalebox{0.4}{
\begin{tikzpicture}
  \node[circle, draw=black] at (01, 2){\phantom{c}};
  \node[circle, draw=black] at (02, 2){\phantom{c}};
  \node[circle, draw=black] at (03, 2){\phantom{c}};
  \node[circle, draw=black] at (06, 2){\phantom{c}};
  \node[circle, draw=black] at (07, 2){\phantom{c}};
  \node[circle, draw=black] at (08, 2){\phantom{c}};
  \node[circle, draw=black] at (09, 2){\phantom{c}};
  \node[circle, draw=black] at (11, 2){\phantom{c}};
  \node[circle, draw=black] at (13, 2){\phantom{c}};
  \node[circle, draw=black] at (18, 2){\phantom{c}};
  \node[circle, draw=black] at (19, 2){\phantom{c}};
  \node[circle, draw=black] at (22, 2){\phantom{c}};
  \node[circle, draw=black] at (23, 2){\phantom{c}};
  \node[circle, draw=black] at (24, 2){\phantom{c}};
  \node[circle, draw=black] at (25, 2){\phantom{c}};
  \node[circle, draw=black] at (26, 2){\phantom{c}};

  \node[circle, draw=black] at (02, 1){\phantom{c}};
  \node[circle, draw=black] at (04, 1){\phantom{c}};
  \node[circle, draw=black] at (05, 1){\phantom{c}};
  \node[circle, draw=black] at (06, 1){\phantom{c}};
  \node[circle, draw=black] at (08, 1){\phantom{c}};
  \node[circle, draw=black] at (10, 1){\phantom{c}};
  \node[circle, draw=black] at (11, 1){\phantom{c}};
  \node[circle, draw=black] at (12, 1){\phantom{c}};
  \node[circle, draw=black] at (14, 1){\phantom{c}};
  \node[circle, draw=black] at (16, 1){\phantom{c}};
  \node[circle, draw=black] at (17, 1){\phantom{c}};
  \node[circle, draw=black] at (19, 1){\phantom{c}};
  \node[circle, draw=black] at (20, 1){\phantom{c}};
  \node[circle, draw=black] at (21, 1){u};
  \node[circle, draw=black] at (22, 1){\phantom{c}};
  \node[circle, draw=black] at (24, 1){\phantom{c}};
  \node[circle, draw=black] at (26, 1){\phantom{c}};
\end{tikzpicture}
}



\begin{example}
To compute $[135452]$ we count $(1,1,1,1,2)$-MLQ's, such as the following one.
\[
\begin{tikzpicture}\node at (0, 4){2};\node at (1, 4){2};\node at (2, 4){2};\node[circle, draw=black] at (3, 4){1};\node at (4, 4){2};\node at (5, 4){2};\node at (0, 3){3};\node at (1, 3){3};\node[circle, draw=black] at (2, 3){2};\node at (3, 3){3};\node at (4, 3){3};\node[circle, draw=black] at (5, 3){1};\node[circle, draw=black] at (0, 2){1};\node[circle, draw=black] at (1, 2){3};\node at (2, 2){4};\node at (3, 2){4};\node[circle, draw=black] at (4, 2){2};\node at (5, 2){4};\node[circle, draw=black] at (0, 1){1};\node[circle, draw=black] at (1, 1){3};\node at (2, 1){5};\node[circle, draw=black] at (3, 1){4};\node at (4, 1){5};\node[circle, draw=black] at (5, 1){2};\end{tikzpicture}
\]
To get a word $136452$ or $135462$ we can add a new row on top, with $5$ boxes. The total weight of these additions is $e_5(x_1, \dots, x_6)$. Here is an example of such an addition.
\[
\begin{tikzpicture}\node[circle, draw=black] at (0, 5){1};\node[circle, draw=black] at (1, 5){1};\node[circle, draw=black] at (2, 5){1};\node at (3, 5){2};\node[circle, draw=black] at (4, 5){1};\node[circle, draw=black] at (5, 5){1};\node at (0, 4){2};\node at (1, 4){2};\node at (2, 4){3};\node[circle, draw=black] at (3, 4){1};\node at (4, 4){2};\node at (5, 4){2};\node at (0, 3){3};\node at (1, 3){4};\node[circle, draw=black] at (2, 3){2};\node at (3, 3){3};\node at (4, 3){3};\node[circle, draw=black] at (5, 3){1};\node[circle, draw=black] at (0, 2){1};\node[circle, draw=black] at (1, 2){3};\node at (2, 2){4};\node at (3, 2){4};\node[circle, draw=black] at (4, 2){2};\node at (5, 2){5};\node[circle, draw=black] at (0, 1){1};\node[circle, draw=black] at (1, 1){3};\node at (2, 1){5};\node[circle, draw=black] at (3, 1){4};\node at (4, 1){6};\node[circle, draw=black] at (5, 1){2};\end{tikzpicture}
\]
By the theorem, such queues are in 1-1 correspondence with ordinary queues counting, in this case, $[135462]$. By applying the bijection $4$ times to bring the top row to the bottom, we get the corresponding ordinary queue, as follows.\\

\noindent Swap rows 1 and 2:
\[
\begin{tikzpicture}\node at (0, 5){2};\node at (1, 5){2};\node at (2, 5){2};\node[circle, draw=black] at (3, 5){1};\node at (4, 5){2};\node at (5, 5){2};\node[circle, draw=black] at (0, 4){2};\node[circle, draw=black] at (1, 4){2};\node at (2, 4){3};\node[circle, draw=black] at (3, 4){1};\node[circle, draw=black] at (4, 4){2};\node[circle, draw=black] at (5, 4){2};\node at (0, 3){3};\node at (1, 3){4};\node[circle, draw=black] at (2, 3){2};\node at (3, 3){3};\node at (4, 3){3};\node[circle, draw=black] at (5, 3){1};\node[circle, draw=black] at (0, 2){1};\node[circle, draw=black] at (1, 2){3};\node at (2, 2){4};\node at (3, 2){4};\node[circle, draw=black] at (4, 2){2};\node at (5, 2){5};\node[circle, draw=black] at (0, 1){1};\node[circle, draw=black] at (1, 1){3};\node at (2, 1){5};\node[circle, draw=black] at (3, 1){4};\node at (4, 1){6};\node[circle, draw=black] at (5, 1){2};\end{tikzpicture}
\]
Swap rows 2 and 3:
\[
\begin{tikzpicture}\node at (0, 5){2};\node at (1, 5){2};\node at (2, 5){2};\node[circle, draw=black] at (3, 5){1};\node at (4, 5){2};\node at (5, 5){2};\node at (0, 4){3};\node[circle, draw=black] at (1, 4){2};\node at (2, 4){3};\node at (3, 4){3};\node at (4, 4){3};\node[circle, draw=black] at (5, 4){1};\node[circle, draw=black] at (0, 3){3};\node at (1, 3){4};\node[circle, draw=black] at (2, 3){2};\node[circle, draw=black] at (3, 3){3};\node[circle, draw=black] at (4, 3){3};\node[circle, draw=black] at (5, 3){1};\node[circle, draw=black] at (0, 2){1};\node[circle, draw=black] at (1, 2){3};\node at (2, 2){4};\node at (3, 2){4};\node[circle, draw=black] at (4, 2){2};\node at (5, 2){5};\node[circle, draw=black] at (0, 1){1};\node[circle, draw=black] at (1, 1){3};\node at (2, 1){5};\node[circle, draw=black] at (3, 1){4};\node at (4, 1){6};\node[circle, draw=black] at (5, 1){2};\end{tikzpicture}
\]
Swap rows 3 and 4:
\[
\begin{tikzpicture}\node at (0, 5){2};\node at (1, 5){2};\node at (2, 5){2};\node[circle, draw=black] at (3, 5){1};\node at (4, 5){2};\node at (5, 5){2};\node at (0, 4){3};\node[circle, draw=black] at (1, 4){2};\node at (2, 4){3};\node at (3, 4){3};\node at (4, 4){3};\node[circle, draw=black] at (5, 4){1};\node[circle, draw=black] at (0, 3){3};\node at (1, 3){4};\node at (2, 3){4};\node at (3, 3){4};\node[circle, draw=black] at (4, 3){2};\node[circle, draw=black] at (5, 3){1};\node[circle, draw=black] at (0, 2){1};\node[circle, draw=black] at (1, 2){3};\node[circle, draw=black] at (2, 2){4};\node[circle, draw=black] at (3, 2){4};\node[circle, draw=black] at (4, 2){2};\node at (5, 2){5};\node[circle, draw=black] at (0, 1){1};\node[circle, draw=black] at (1, 1){3};\node at (2, 1){5};\node[circle, draw=black] at (3, 1){4};\node at (4, 1){6};\node[circle, draw=black] at (5, 1){2};\end{tikzpicture}
\]
Swap rows 4 and 5
\[
\begin{tikzpicture}\node at (0, 5){2};\node at (1, 5){2};\node at (2, 5){2};\node[circle, draw=black] at (3, 5){1};\node at (4, 5){2};\node at (5, 5){2};\node at (0, 4){3};\node[circle, draw=black] at (1, 4){2};\node at (2, 4){3};\node at (3, 4){3};\node at (4, 4){3};\node[circle, draw=black] at (5, 4){1};\node[circle, draw=black] at (0, 3){3};\node at (1, 3){4};\node at (2, 3){4};\node at (3, 3){4};\node[circle, draw=black] at (4, 3){2};\node[circle, draw=black] at (5, 3){1};\node[circle, draw=black] at (0, 2){1};\node[circle, draw=black] at (1, 2){3};\node at (2, 2){5};\node[circle, draw=black] at (3, 2){4};\node[circle, draw=black] at (4, 2){2};\node at (5, 2){5};\node[circle, draw=black] at (0, 1){1};\node[circle, draw=black] at (1, 1){3};\node[circle, draw=black] at (2, 1){5};\node[circle, draw=black] at (3, 1){4};\node at (4, 1){6};\node[circle, draw=black] at (5, 1){2};\end{tikzpicture}
\]
\end{example}

%% if you use biblatex then this generates the bibliography
%% if you use some other method then remove this and do it your own way
\printbibliography

%\bibliographystyle{alpha}
%\bibliography{queue}{}
%\begin{thebibliography}{ABX}
%\bibitem{AAMP} Chikashi Arita, Arvind Ayyer, Kirone Mallick and Sylvain Prolhac, Recursive structures in the multispecies TASEP, J. Phys. A 44, 335004 (2011).
%\end{thebibliography}
\end{document}
