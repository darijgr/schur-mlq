\documentclass[reqno]{amsart}
\usepackage{setspace,tikz,xcolor,mathrsfs,listings,multicol}
\usepackage{amssymb}
\usepackage{rotating}
\usepackage[vcentermath]{youngtab}
\usepackage{enumerate}
\usepackage[all,cmtip]{xy}
\usetikzlibrary{arrows,matrix}
\usepackage{comment}
\usepackage{color}
\usepackage[sc]{mathpazo}
\usepackage[T1]{fontenc}
\usepackage{needspace}
\usepackage{tabls}
%\usepackage{amsmath}
%\usepackage{amsthm}
%\usepackage{subcaption}
%\usepackage{fullpage}
%\usepackage[margin=1.25in]{geometry}
%\onehalfspacing

\usepackage[colorlinks=true, pdfstartview=FitV, linkcolor=blue, citecolor=blue, urlcolor=blue]{hyperref}

% use these commands for typesetting doi and arXiv references in the bibliography
\newcommand{\doi}[1]{\href{https://doi.org/#1}{\texttt{doi:#1}}}
\newcommand{\arxiv}[1]{\href{http://arxiv.org/abs/#1}{\texttt{arXiv:#1}}}

\newcommand{\iso}{\cong}
%\newcommand{\qedbox}{\rule{2mm}{2mm}}
%\renewcommand{\qedsymbol}{\qedbox}
\newcommand{\qbinom}[3]{\genfrac{[}{]}{0pt}{}{#1}{#2}_{#3}}
\newcommand{\absval}[1]{\left\lvert #1 \right\rvert}
\newcommand{\case}[1]{\vspace{12pt}\noindent\underline{#1}:}
\newcommand{\fs}{\mathcal{S}} % flagged Schur function
\newcommand{\mbf}{\mathbf}
\newcommand{\0}{\phantom{c}}
\newcommand{\swt}[1]{\left\langle #1 \right\rangle} % Spectral weight or amplitude
\newcommand{\merge}[1]{\vee_{#1}} % merge
\newcommand{\SymGp}[1]{\mathfrak{S}_{#1}} % symmetric group
\newcommand{\std}[1]{\widetilde{#1}} % standardization

\DeclareMathOperator{\supp}{supp} % Support
\DeclareMathOperator{\inter}{int} % queuing interval
\DeclareMathOperator{\wt}{wt} % weight
\DeclareMathOperator{\pr}{pr} % promotion
\DeclareMathOperator{\id}{id} % identity
\DeclareMathOperator{\ch}{ch} % character
\DeclareMathOperator{\gr}{gr} % grading

\newcommand{\xx}{\mathbf{x}}
\newcommand{\mm}{\mathbf{m}}
\newcommand{\nn}{\mathbf{n}}
\newcommand{\qq}{\mathbf{q}}
\newcommand{\MLQ}{\mathbf{S}}

\newcommand{\mcA}{\mathcal{A}}
\newcommand{\mcF}{\mathcal{F}}
\newcommand{\mcM}{\mathcal{M}}
\newcommand{\mcW}{\mathcal{W}}
\newcommand{\mcI}{\mathcal{I}}

\newcommand{\ZZ}{\mathbb{Z}}
\newcommand{\QQ}{\mathbb{Q}}
\newcommand{\RR}{\mathbb{R}}
\newcommand{\CC}{\mathbb{C}}

\newcommand{\bze}{\overline{0}}
\newcommand{\bon}{\overline{1}}
\newcommand{\btw}{\overline{2}}
\newcommand{\bth}{\overline{3}}
\newcommand{\bfo}{\overline{4}}
\newcommand{\bfive}{\overline{5}}
\newcommand{\bsix}{\overline{6}}
\newcommand{\bseven}{\overline{7}}
\newcommand{\beight}{\overline{8}}
\newcommand{\bi}{\overline\imath}
\newcommand{\bk}{\overline{k}}
\newcommand{\brr}{\overline{r}}
\newcommand{\bn}{\overline{n}}
\newcommand{\ellbar}{\overline{\ell}}

\newcommand{\fraks}{\mathfrak{s}}

\let\sumnonlimits\sum
\let\prodnonlimits\prod
\let\cupnonlimits\bigcup
\let\capnonlimits\bigcap
\renewcommand{\sum}{\sumnonlimits\limits}
\renewcommand{\prod}{\prodnonlimits\limits}
\renewcommand{\bigcup}{\cupnonlimits\limits}
\renewcommand{\bigcap}{\capnonlimits\limits}

\newenvironment{verlong}{}{}
\newenvironment{vershort}{}{}
\newenvironment{noncompile}{}{}
\excludecomment{verlong}
\includecomment{vershort}
\excludecomment{noncompile}
\newcommand{\rev}{\operatorname{rev}}
\newcommand{\conncomp}{\operatorname{conncomp}}
\newcommand{\core}{\operatorname{core}}
\newcommand{\col}{\operatorname{col}}
\newcommand{\colw}{\operatorname{colw}}
\newcommand{\word}{\operatorname{word}}
\newcommand{\NN}{\mathbb{N}}
\newcommand{\powset}[2][]{\ifthenelse{\equal{#2}{}}{\mathcal{P}\left(#1\right)}{\mathcal{P}_{#1}\left(#2\right)}}
% $\powset[k]{S}$ stands for the set of all $k$-element subsets of
% $S$. The argument $k$ is optional, and if not provided, the result
% is the whole powerset of $S$.
\newcommand{\set}[1]{\left\{ #1 \right\}}
% $\set{...}$ yields $\left\{ ... \right\}$.
\newcommand{\abs}[1]{\left| #1 \right|}
% $\abs{...}$ yields $\left| ... \right|$.
\newcommand{\tup}[1]{\left( #1 \right)}
% $\tup{...}$ yields $\left( ... \right)$.
\newcommand{\ive}[1]{\left[ #1 \right]}
% $\ive{...}$ yields $\left[ ... \right]$.
\newcommand{\verts}[1]{\operatorname{V}\left( #1 \right)}
% $\verts{...}$ yields $\operatorname{V}\left( ... \right)$.
\newcommand{\edges}[1]{\operatorname{E}\left( #1 \right)}
% $\edges{...}$ yields $\operatorname{E}\left( ... \right)$.
\newcommand{\arcs}[1]{\operatorname{A}\left( #1 \right)}
% $\arcs{...}$ yields $\operatorname{A}\left( ... \right)$.
\newcommand{\underbrack}[2]{\underbrace{#1}_{\substack{#2}}}
% $\underbrack{...1}{...2}$ yields
% $\underbrace{...1}_{\substack{...2}}$. This is useful for doing
% local rewriting transformations on mathematical expressions with
% justifications.
\newcommand{\mlnode}[1]{\node[circle, draw=black] at (#1){\phantom{c}};}

% Dark red emphasis
\definecolor{darkred}{rgb}{0.7,0,0} % darkred color
\newcommand{\defn}[1]{{\color{darkred}\emph{#1}}} % emphasis of a definition

%% For typesetting code listings                                                
\usepackage{listings}
\lstdefinelanguage{Sage}[]{Python}
{morekeywords={False,sage,True},sensitive=true}
\lstset{
  frame=single,
  showtabs=False,
  showspaces=False,
  showstringspaces=False,
  commentstyle={\ttfamily\color{dgreencolor}},
  keywordstyle={\ttfamily\color{dbluecolor}\bfseries},
  stringstyle={\ttfamily\color{dgraycolor}\bfseries},
  language=Sage,
  basicstyle={\footnotesize\ttfamily},
  aboveskip=0.75em,
  belowskip=0.75em,
  xleftmargin=.15in,
}
\definecolor{dblackcolor}{rgb}{0.0,0.0,0.0}
\definecolor{dbluecolor}{rgb}{0.01,0.02,0.7}
\definecolor{dgreencolor}{rgb}{0.2,0.4,0.0}
\definecolor{dgraycolor}{rgb}{0.30,0.3,0.30}
\newcommand{\dblue}{\color{dbluecolor}\bf}
\newcommand{\dred}{\color{dredcolor}\bf}
\newcommand{\dblack}{\color{dblackcolor}\bf}

\usepackage{xparse}

\makeatletter
% \specialmergetwolists{<coupler>}{<list1>}{<list2>}{<return macro>}
% \specialmergetwolists*{<coupler>}{<listcmd1>}{<listcmd2>}{<return macro>}
\protected\def\specialmergetwolists{%
  \begingroup
  \@ifstar{\def\cnta{1}\@specialmergetwolists}
    {\def\cnta{0}\@specialmergetwolists}%
}
\def\@specialmergetwolists#1#2#3#4{%
  \def\tempa##1##2{%
    \edef##2{%
      \ifnum\cnta=\@ne\else\expandafter\@firstoftwo\fi
      \unexpanded\expandafter{##1}%
    }%
  }%
  \tempa{#2}\tempb\tempa{#3}\tempa
  \def\cnta{0}\def#4{}%
  \foreach \x in \tempb{%
    \xdef\cnta{\the\numexpr\cnta+1}%
    \gdef\cntb{0}%
    \foreach \y in \tempa{%
      \xdef\cntb{\the\numexpr\cntb+1}%
      \ifnum\cntb=\cnta\relax
        \xdef#4{#4\ifx#4\empty\else,\fi\x#1\y}%
        \breakforeach
      \fi
    }%
  }%
  \endgroup
}
\makeatother

\theoremstyle{plain}
\newtheorem{thm}{Theorem}[section]
\newtheorem{lemma}[thm]{Lemma}
\newtheorem{conj}[thm]{Conjecture}
\newtheorem{prop}[thm]{Proposition}
\newtheorem{cor}[thm]{Corollary}
\theoremstyle{definition}
\newtheorem{dfn}[thm]{Definition}
\newtheorem{example}[thm]{Example}
\newtheorem{remark}[thm]{Remark}
\numberwithin{equation}{section}
%\numberwithin{figure}{section}
%\numberwithin{table}{section}
%\setcounter{section}{-1}

% For breaking equations across multiple pages
% \allowdisplaybreaks[1]

\usepackage[colorinlistoftodos]{todonotes}
\newcommand{\erik}[1]{\todo[size=\tiny,color=green!30]{#1 \\ \hfill --- Erik}}
\newcommand{\Erik}[1]{\todo[size=\tiny,inline,color=green!30]{#1
      \\ \hfill --- Erik}}
\newcommand{\darij}[1]{\todo[size=\tiny,color=red!30]{#1 \\ \hfill --- Darij}}
\newcommand{\Darij}[1]{\todo[size=\tiny,inline,color=red!30]{#1
      \\ \hfill --- Darij}}
\newcommand{\travis}[1]{\todo[size=\tiny,color=blue!30]{#1 \\ \hfill --- Travis}}
\newcommand{\Travis}[1]{\todo[size=\tiny,inline,color=blue!30]{#1
      \\ \hfill --- Travis}}

%%%%%%%%%%%%%%%%%%%%%%%%%%%%%%%%%%%%%%%%

\begin{document}
\title[MLQs]{Multiline queues with spectral parameters \\ %
(abstract for AEC 2018)}

\author{\href{http://www.cip.ifi.lmu.de/~grinberg/}{Darij Grinberg}\\
joint work with Erik Aas and \href{https://sites.google.com/view/tscrim/home}{Travis Scrimshaw}}

\date{\today}

% \keywords{multiline queue, TASEP, R-matrix, symmetric function}
% \subjclass[2010]{
% 60C05,  % Combinatorial probability
% 05A19,  % Combinatorial identities, bijective combinatorics
% 16T25,  % Yang--Baxter equations
% 05E05}  % Symmetric functions

% \thanks{TS was partially supported by the Australian Research Council DP170102648 and the National Science Foundation RTG grant DMS-1148634.}

% \begin{abstract}
% Using the description of multiline queues as functions on words, we introduce the notion of a spectral weight of a word by defining a new weighting on multiline queues.
% We show that the spectral weight of a word is invariant under a natural action of the symmetric group, giving a proof of the commutativity conjecture of Arita, Ayyer, Mallick, and Prolhac.
% We give a determinant formula for the spectral weight of a word, which gives a proof of a conjecture of the first author and Linusson.
% \end{abstract}

\maketitle

% %=====================================================================
% \section{Introduction}
% \label{sec:introduction}

% One of the fundamental models of particles moving in a 1-dimensional lattice is the asymmetric simple exclusion process (ASEP), and it has received broad attention in many different variations.
% The earliest known publication of the ASEP was done to model the dynamics of ribosomes along RNA~\cite{MGP68}.
% For statistical mechanics, it is a model for gas particles in a lattice with an induced current, where the exclusion mimics the short-range interactions among the particles.
% Despite admitting very simple descriptions of the particle dynamics, the ASEP has very rich macroscopic behaviors, such as
% \begin{itemize}
% \item boundary-induced phase transitions~\cite{Krug91},
% \item spontaneous symmetry breaking with possibly multiple broken symmetry phases~\cite{AHR98,AHR99,CEM01,EFGM95,EPSZ05,GLEMSS95,PK07},
% \item describing the formations of shocks~\cite{DJLS93,Ferrari92,FF94,FF94II,Liggett76}, and
% \item phase separation and condensation~\cite{EKKM98,JNHWW09,KLMST02,RSS00}.
% \end{itemize}
% We also refer the reader to~\cite{PEM09,Schutz01,SZ95,TJHJ16} and references therein.

% The term exclusion process was coined by Spitzer~\cite{Spitzer70}, where he was focused on an application with Brownian motion with hard-core interactions.
% Moreover, it was~\cite{Spitzer70} that initiated the investigation of exclusion processes using probability theory.
% However, the applications of the ASEP (and its variations) has since spread to other areas, such as
% \begin{itemize}
% \item transportation processes in capillary vessels~\cite{Levitt73} or proteins within the cells along actin filaments~\cite{KNL05},
% \item anistropic conductors known as solid electrolytes~\cite{CL99},
% \item discrete models of traffic flow~\cite{Schad01},
% \item partition growth processes~\cite{Lam15},
% \item random matrix theory~\cite{Johansson00,TW09}, and
% \item moments of Askey--Wilson polynomials~\cite{CW11}.
% \end{itemize}

% If we prohibit the particles from moving backwards, we obtain the totally asymmetric exclusion process (TASEP), a non-equilibrium stochastic process that has its own vast literature.
% For example, we refer the reader to~\cite{AasLin17,AAMP,BE07,BP14,DEHP93,KMO15,KMO16,Liggett99} and references therein.
% In this paper, we consider the TASEP on a ring with $n$ sites and $\ell$ species of particles.
% Thus, we will consider the states to be words $u$ in the alphabet $\{1, \dotsc, \ell\}$ of length $n$, where we take the indices to be $\ZZ / n \ZZ$.
% We will also consider our process to be discrete in time, where our transition map interchanges a pair $u_i u_{i+1}$ with $u_i > u_{i+1}$ to $u_{i+1} u_i$ and is done at a uniform rate.

% The steady state of the TASEP on a ring is known in terms of another process using ordinary multiline queues (MLQs) and applying the Ferrari--Martin (FM) algorithm~\cite{FM06,FM07}.
% This is a generalization of 2-line queues used by Angel~\cite{Angel06} and the work of Ferrari, Fontes, and Kohayakawa~\cite{FFK94}.
% In~\cite{KMO15,KMO16}, the FM algorithm was reformulated in terms of the combinatorial $R$-matrix~\cite{NY97,Shimozono02} and using type $A_{n-1}^{(1)}$ Kirillov--Reshetikhin crystals~\cite{KKMMNN92}.
% This interpretation gives a connection with five-vertex models, corner transfer matrices~\cite{Baxter89}, 3D integrable lattice models, and the tetrahedron equation~\cite{Zam80}, yielding a matrix product formula for the steady state distribution different than~\cite{CdGW15,EFM09,PEM09}.

% In this paper, we introduce a new weighting of MLQs, which is the weight of the MLQ considered as a tensor product of Kirillov--Reshetikhin crystals.
% We also interpret MLQs as functions on words of a fixed length $n$ following~\cite{AAMP}, where it was referred to as the generalized FM algorithm.
% This allows us to define the spectral weight or amplitude of a word $u$ to be the sum over all the weight of all ordinary MLQs $\qq$ such that $u = \qq(1^n)$.
% Furthermore, we introduce the notion of a $\sigma$-twisted MLQ, where $\sigma$ is a permutation, although this is implicitly considered in~\cite{AAMP}.
% Our main result (Theorem~\ref{thm:permutation}) is that for a fixed permutation $\sigma$, the sum of the weights of all $\sigma$-twisted MLQs $\qq_{\sigma}$ such that $u = \qq_{\sigma}(1^n)$ equals the spectral weight of $u$.
% To this end, we construct an action of the symmetric group on MLQs that corresponds, under the usual FM algorithm, to the natural action by letters on words.
% We show that does not change the MLQ as a function on words.
% This action has previously appeared in a number of different guises, such as in Danilov and Koshevoy~\cite{DanilovKoshevoy} (see also~\cite[Ch.~4]{Gorodentsev2}), van Leeuwen~\cite[Lemma~2.3]{vanLeeuwen-dc}, and Lothaire~\cite[Ch.~5, (5.6.3)]{Loth}.
% In the context of Kirillov--Reshetikhin crystals, it can be described as applying a combinatorial $R$-matrix to an MLQ, where the weight remaining invariant is a condition of being a crystal isomorphism.

% As a consequence of this action and specializing our weight parameters to $1$, we obtain a proof of the commutativity conjecture of~\cite{AAMP}.
% However, we note that the interlacing property of~\cite{AAMP} does not generalize to our weighting of MLQs.
% Furthermore, we give a determinant expression for the spectral weight of decreasing words by using the Lindstr\"om--Gessel--Viennot Lemma~\cite{GV85,Lindstrom73}.
% By combining these results, we obtain a proof of~\cite[Conj.~3.10]{AasLin17}, which in turn proves a number of other conjectures in~\cite{AasLin17}.

% We note that our weighting scheme can be extended to multiline process used to determine the steady state distribution of the totally asymmetric zero range process (TARZP) on a ring, where multiple particles can occupy the same site.
% This comes from the fact that the TARZP steady state distribution can also be computed using a tensor product of Kirillov--Reshetikhin crystals (under rank-level duality) using combinatorial $R$-matrices with analogous connections to corner transfer matrices and the tetrahedron equation~\cite{KMO16TARZP,KMO16TARZPII}.
% Thus, we expect that a similar description of $\sigma$-twisted multiline process can be defined such that the weighting is invariant under the action of the combinatorial $R$-matrix.
% Yet it seems unlikely that our weighting is related to the steady state distribution for the inhomogeneous TASEP~\cite{AM13,AL14} or TARZP~\cite{KMO16II}.

% This paper is organized as follows.
% In Section~\ref{sec:background}, we give the necessary background and definitions of MLQs and spectral weight.
% In Section~\ref{sec:result}, we give our main results.
% In Section~\ref{sec:JT_formula}, we show a Jacobi-Trudi like formula for special words $u$, which we use to prove some of the conjectures in~\cite{AasLin17}.
% In Section~\ref{sec:tasep}, we describe the connection between MLQs and the TASEP.
% In Section~\ref{sec:thm_proof}, we give a proof of our main theorem (Theorem~\ref{thm:permutation}).
% In Section~\ref{sec:remarks}, we give some additional remarks about our results.


% \subsection{Acknowledgements}

% We thank Atsuo Kuniba for explaining the results in his papers~\cite{KMO15,KMO16II,KMO16,KMO16TARZP,KMO16TARZPII}.
% We thank Olya Mandelshtam for useful discussions on the inhomogeneous TASEP.
% We thank Jae-Hoon Kwon for pointing out that the $\SymGp{n}$-action on MLQs comes from an $(\mathfrak{sl}_m \oplus \mathfrak{sl}_n)$-action.
% This work benefited from computations using \textsc{SageMath}~\cite{sage,combinat}.







This talk is about the preprint
\href{http://www.cip.ifi.lmu.de/~grinberg/algebra/mlqs.pdf}{\textit{Multiline queues with spectral parameters}}
by Erik Aas, Travis Scrimshaw and myself.

% %=====================================================================
% \section{Background and definitions}
% \label{sec:background}

Fix a positive integer $n$.
For a nonnegative integer $k$, let $\ive{k}$ denote the set $\set{1, 2, \ldots, k}$.
% Let $\SymGp{k}$ denote the symmetric group on $\ive{k}$, and let $s_i \in \SymGp{k}$ be the transposition swapping $i$ with $i+1$.
% Let $w_0 \in \SymGp{k}$ be the longest element: the permutation $k (k-1) \dotsm 321$ (written in one line notation) that reverses the order of all elements.

We shall refer to the elements $1, 2, \ldots, n \in \ZZ / n \ZZ$ as \defn{sites}.
We visualize them as points on a line that ``wraps around'' cyclically; thus, for example, the sites weakly to the right of a site $i$ are $i, i+1, \ldots, n-1, n, 1, 2, 3, \ldots$ (in this order).

% %%%%%%%%%%
% \subsection{Words and queues}

Let $\mcW_n$ be the set of words $u = u_1 \dotsm u_n$ in the ordered alphabet $\mcA := \{1 < 2 < 3 < \cdots \}$.
If $u$ is a word and $k$ is a site, then $u_k$ shall mean $u_g$, where $g$ is the element of $\ive{n}$ having remainder class $k$ in $\ZZ / n \ZZ$.
% Let $\xx := \{x_1, x_2, x_3, \ldots\}$ be commuting indeterminates.

% The \defn{type} of a word $u$ is the vector $\mm = (m_1, m_2, \ldots)$, where $m_i$ is the number of occurrences of $i$ in $u$.
% We sometimes refer to $u_i = t$ as a \defn{particle at $i$ of class $t$}.
% Let $\ell = \max\{i \mid m_i \neq 0 \}$, which we say is the number of \defn{classes} in $u$ or $\mm$.
% A word $u$ or type $\mm$ with $\ell$ classes is \defn{packed} if $m_i \neq 0$ for all $1 \leq i \leq \ell$.
% A word $w$ of type $\mm$ is \defn{standard} if $m_i \leq 1$ for all $i$.

% We \defn{merge} two adjacent classes $i,i+1$ in a word $u$ to obtain a new word by replacing all occurrences of $j$ by $j-1$ in $u$ for each $j = i+1, i+2, \ldots$ in that order.
% We denote the merging of $i$ and $i+1$ in $u$ by $\merge{i} u$.
% Note that $\merge{i} u$ is packed whenever $u$ is packed.
% For $T = \set{t_1 < \cdots < t_k} \subseteq \ive{\ell-1}$, we set $\bigvee_T u := \merge{t_1} \cdots \merge{t_k} u$.
% Similarly, the merging of $i,i+1$ in a type $\mm = (m_1, m_2, \ldots)$ is $\merge{i}(\mm) = (m_1, \dotsc, m_{i-1}, m_i + m_{i+1}, m_{i+2}, \ldots)$.
% These operations interact as one would hope:
% If the type of a word $u$ is $\mm$, then the type of $\merge{i} u$ is $\merge{i}(\mm)$.

% Fix a word $u \in \mcW_n$, and let $\mm = (m_1, m_2, \ldots)$ be the type of $u$.
% For each $i \geq 0$, set
% \begin{equation}
% \label{eq:type_partial_sums}
% p_i(\mm) := m_1 + m_2 + \cdots + m_i.
% \end{equation}
% When $\mm$ is clear, we simply write $p_i$ for this.
% (Thus, $p_0 = 0$ and $p_i = n$ for sufficiently large $i$.)

A \defn{queue} shall mean a subset of $\ive{n}$.
% The \defn{weight} of a queue $q$ is $\wt(q) := \prod_{i \in q} x_i$.

If $q$ is a queue and $u \in \mcW_n$ is a word, then a new word $v = q(u) \in \mcW_n$ is defined as follows:
In the beginning, no letter of $v \in \mcW_n$ is set.
Choose a permutation $\tup{i_1, i_2, \ldots, i_n}$ of $\tup{1, 2, \ldots, n}$
such that $u_{i_1} \leq u_{i_2} \leq \cdots \leq u_{i_n}$.

\begin{description}
\item[Phase I]
  For $i = i_n, i_{n-1}, \ldots, i_{\abs{q}+1}$, do the following.
    Find the first site $j$ weakly to the left (cyclically) of $i$ such that $j \notin q$ and $v_j$ is not set.
    Then set $v_j = u_i + 1$.

\item[Phase II]
  For $i = i_1, i_2, \ldots, i_{\abs{q}}$, do the following.
    Find the first site $j$ weakly to the right (cyclically) of $i$ such that $j \in q$ and $v_j$ is not set.
    Then set $v_j = u_i$.
\end{description}

It is not hard to check that the resulting word $v = q(u)$ does not depend on the choice of permutation $(i_1, i_2, \dotsc, i_n)$,
and that the two phases of the algorithm can be arbitrarily interleaved instead of doing the second after the first.

% \begin{remark}
% \label{rmk:t-splitting}
% Let $u$ and $q$ be as before. Let $r = \abs{q}$.
% There exists a $t \in \ive{\ell}$ such that
% $
% p_{t-1} \leq r \leq p_t.
% $
% The word $v = q(u)$ then has type
% \begin{equation}
% \label{eq:queue_type_change}
% (m_1, \dots, m_{t-1}, r-p_{t-1}, p_{t}-r, m_{t+1}, m_{t+2}, \ldots).
% \end{equation}
% Note that $p_{t} - r = m_{t} + (p_{t-1} - r)$.
% We think of this as splitting the class $t$ into two new classes $t$ and $t+1$. % and so we call $t$ the \defn{splitting class} of $q(u)$.

% For all $i$ processed in Phase~I (resp.\ Phase~II) of the algorithm, we have $u_i \geq t$ (resp.~$u_i \leq t$).
% The queue $q$ can be reconstructed from $v$ and $t$ as the set of all $j \in \ive{n}$ satisfying $v_j \leq t$.
% \end{remark}

\begin{example}
\label{ex:first_queue}
We consider the $4$-queue $q = \{1, 4, 8, 9\}$, and let $u = 346613321$.
% Thus, the type of $u$ is $\mm = (2, 1, 3, 1, 0, 2, 0, \ldots)$ with $p_2 = 3$ and $p_3 = 6$.
% Thus, the $t$ in Remark~\ref{rmk:t-splitting} equals $3$.
To compute $q(u)$, draw the following diagram
(whose upper row shows $u$, whose lower row shows $q(u)$,
and whose middle row represents the set $q$ by balls in the positions of its elements):
\[
\begin{tikzpicture}[>=latex,rounded corners,yscale=1.5,xscale=1.2,baseline=0]
\def\passwidth{3pt};
\node (i1) at (1,1) {$3$};
\node (i2) at (2,1) {$4$};
\node (i3) at (3,1) {$6$};
\node (i4) at (4,1) {$6$};
\node (i5) at (5,1) {$1$};
\node (i6) at (6,1) {$3$};
\node (i7) at (7,1) {$3$};
\node (i8) at (8,1) {$2$};
\node (i9) at (9,1) {$1$};
\node (t1) at (1,-1) {$2$};
\node (t2) at (2,-1) {$7$};
\node (t3) at (3,-1) {$7$};
\node (t4) at (4,-1) {$3$};
\node (t5) at (5,-1) {$4$};
\node (t6) at (6,-1) {$4$};
\node (t7) at (7,-1) {$5$};
\node (t8) at (8,-1) {$1$};
\node (t9) at (9,-1) {$1$};
\node[circle,draw=black] (q1) at (1,0) {};
\node[circle,draw=black] (q2) at (4,0) {};
\node[circle,draw=black] (q3) at (8,0) {};
\node[circle,draw=black] (q4) at (9,0) {};
\draw[->,red] (i4) -- (4,0.5) .. controls (3.8,0.2) and (3.5,0) .. (3.1,0) -- (2,0) -- (t2);
\draw[->,red] (i3) -- (t3);
\draw[->,red] (i2) -- (2,0.15) .. controls (1.8,-0.2) and (1.6,-0.25) .. (1.3,-0.25) -- (0,-0.25);
\draw[>->,red] (10,-0.25) -- (8,-0.25) -- (7,-0.25) -- (t7);
\draw[->,red] (i6) -- (t6);
\draw[->,red] (i7) -- (7,0) -- (5,0) -- (t5);
\draw[white,line width=\passwidth] (5,0.28) -- (7.3,0.28) .. controls (7.7,0.28) and (7.85,0.12) .. (q3);  % To simulate underpass
\draw[->,blue] (i5) -- (5,0.28) -- (7.3,0.28) .. controls (7.7,0.28) and (7.85,0.12) .. (q3);
\draw[white,line width=\passwidth] (q3) -- (t8);  % To simulate underpass
\draw[->,blue] (q3) -- (t8);
\draw[->,blue] (i9) -- (q4);
\draw[white,line width=\passwidth] (q4) -- (t9);  % To simulate underpass
\draw[->,blue] (q4) -- (t9);
\draw[white,line width=\passwidth] (1,0.28) -- (3.3,0.28) .. controls (3.7,0.28) and (3.85,0.12) .. (q2);  % To simulate underpass
\draw[->,blue] (i1) -- (1,0.28) -- (3.3,0.28) .. controls (3.7,0.28) and (3.85,0.12) .. (q2);
\draw[->,blue] (q2) -- (t4);
\draw[white,line width=\passwidth] (i8) -- (8,0.28) -- (10,0.28);  % To simulate underpass
\draw[->,blue] (i8) -- (8,0.28) -- (10,0.28);
\draw[>->,blue] (0,0.28) -- (0.3,0.28) .. controls (0.7,0.28) and (0.85,0.12) .. (q1);
\draw[white,line width=\passwidth] (q1) -- (t1);  % To simulate underpass
\draw[->,blue] (q1) -- (t1);
\end{tikzpicture}
\]
where the paths in red correspond to Phase I and those in blue are from Phase II. Hence, $q(346613321) = 277344511$.
\end{example}

In this talk I will present properties of this action of queues on words, and discuss a generating function obtained from it.

Consider any positive integer $\ell$, any word $u \in \mcW_n$ and any permutation $\sigma$ of $\ive{\ell-1}$. Let $1 \cdots 1 \in \mcW_n$ be the word whose all letters equal $1$. The generating function we are interested in is the polynomial \defn{$\swt{u}_\sigma$} in $n$ variables $x_1, x_2, \ldots, x_n$ defined by
\[
\swt{u}_\sigma = \sum_{\tup{q_1, q_2, \ldots, q_{\ell-1}}} \prod_{i=1}^{\ell-1} \prod_{k \in q_i} x_k ,
\]
where the sum is over all $\tup{\ell-1}$-tuples $\tup{q_1, q_2, \ldots, q_{\ell-1}}$ of queues satisfying $u =  q_{\ell-1}\bigl( \cdots q_2\bigl( q_1(1 \cdots 1) \bigr) \cdots \bigr)$ and $\abs{q_i} = \tup{\# \text{ of letters } \leq \sigma\tup{i} \text{ in }u }$ for each $i\in\ive{\ell-1}$.

Our main result is that this polynomial $\swt{u}_\sigma$ does not depend on $\sigma$, whence it can be denoted by \defn{$\swt{u}$}. The proof of this relies on an action of the symmetric group on tuples of queues, which is closely related to the Lascoux-Sch\"utzenberger action of the symmetric group on words (a.k.a., the Weyl group action on the type-A crystal of words). This can be used to verify a recent conjecture by Arita, Ayyer, Mallick and Prolhac on the totally asymmetric exclusion process (TASEP).

We further give a Jacobi-Trudi-like formula for $\swt{u}$ for a certain class of words $u$; this yields some recent conjectures by Aas and Linusson.

% \bibliographystyle{alpha}
% \bibliography{queue}{}
\end{document}
