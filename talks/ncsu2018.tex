\documentclass{beamer}
\usepackage{amsmath}
\usepackage{amssymb}
\usepackage{array}
\usepackage{setspace}
\usepackage{graphicx}
\usepackage{tikz}
\usetikzlibrary{matrix,arrows,backgrounds,shapes.misc,shapes.geometric,fit}
\usepackage{etex}
\usepackage{amsthm}
\usepackage{color}
\usepackage{wasysym}
\usepackage[all]{xy}
\usepackage{textpos}
\usepackage{ytableau}
\usepackage{stmaryrd}
%\usetikzlibrary{calc,through,backgrounds}
%\CompileMatrices

\definecolor{grau}{rgb}{.5 , .5 , .5}
\definecolor{dunkelgrau}{rgb}{.35 , .35 , .35}
\definecolor{schwarz}{rgb}{0 , 0 , 0}
\definecolor{violet}{RGB}{143,0,255}
\definecolor{forestgreen}{RGB}{34, 100, 34}

\newcommand{\red}{\color{red}}
\newcommand{\grey}{\color{grau}}
\newcommand{\green}{\color{forestgreen}}
\newcommand{\violet}{\color{violet}}
\newcommand{\blue}{\color{blue}}

\newcommand{\bIf}{\textbf{If} }
\newcommand{\bif}{\textbf{if} }
\newcommand{\bthen}{\textbf{then} }

\newcommand{\EE}{{\mathbf{E}}}
\newcommand{\ZZ}{{\mathbb Z}}
\newcommand{\NN}{{\mathbb N}}
\newcommand{\QQ}{{\mathbb Q}}
\newcommand{\kk}{{\mathbf k}}
\newcommand{\OO}{\operatorname {O}}
\newcommand{\Nm}{\operatorname {N}}
\newcommand{\Par}{\operatorname{Par}}
\newcommand{\Comp}{\operatorname{Comp}}
\newcommand{\Stab}{\operatorname {Stab}}
\newcommand{\id}{\operatorname{id}}
\newcommand{\ev}{\operatorname{ev}}
\newcommand{\Sym}{\operatorname{Sym}}
\newcommand{\Lpk}{\operatorname{Lpk}}
\newcommand{\lpk}{\operatorname{lpk}}
\newcommand{\Rpk}{\operatorname{Rpk}}
\newcommand{\rpk}{\operatorname{rpk}}
\newcommand{\Pk}{\operatorname{Pk}}
\newcommand{\Epk}{\operatorname{Epk}}
\newcommand{\epk}{\operatorname{epk}}
\newcommand{\Des}{\operatorname{Des}}
\newcommand{\des}{\operatorname{des}}
\newcommand{\inv}{\operatorname{inv}}
\newcommand{\maj}{\operatorname{maj}}
\newcommand{\Val}{\operatorname{Val}}
\newcommand{\pk}{\operatorname{pk}}
\newcommand{\st}{\operatorname{st}}
\newcommand{\QSym}{\operatorname{QSym}}
\newcommand{\NSym}{\operatorname{NSym}}
\newcommand{\Mat}{\operatorname{M}}
\newcommand{\bk}{\mathbf{k}}
\newcommand{\Nplus}{\mathbb{N}_{+}}
\newcommand\arxiv[1]{\href{http://www.arxiv.org/abs/#1}{\texttt{arXiv:#1}}}
\newcommand{\Orb}{{\mathcal O}}
\newcommand{\GL}{\operatorname {GL}}
\newcommand{\SL}{\operatorname {SL}}
\newcommand{\Or}{\operatorname {O}}
\newcommand{\im}{\operatorname {Im}}
\newcommand{\Iso}{\operatorname {Iso}}
\newcommand{\Adm}{\operatorname{Adm}}
\newcommand{\Supp}{\operatorname{Supp}}
\newcommand{\Powser}{\QQ\left[\left[x_1,x_2,x_3,\ldots\right]\right]}
\newcommand{\rad}{\operatorname {rad}}
\newcommand{\zero}{\mathbf{0}}
\newcommand{\xx}{\mathbf{x}}
\newcommand{\ord}{\operatorname*{ord}}
\newcommand{\bbK}{{\mathbb{K}}}
\newcommand{\whP}{{\widehat{P}}}
%\newcommand{\Trop}{\operatorname*{Trop}}
%\newcommand{\TropZ}{{\operatorname*{Trop}\mathbb{Z}}}
\newcommand{\rato}{\dashrightarrow}
\newcommand{\lcm}{\operatorname*{lcm}}
\newcommand{\lm}{\lambda / \mu}

\newcommand{\fti}[1]{\frametitle{\ \ \ \ \ #1}}
\newenvironment{iframe}[1][]{\begin{frame} \fti{[#1]} \begin{itemize}}{\end{frame}\end{itemize}}

\newcommand{\are}{\ar@{-}}
\newcommand{\arinj}{\ar@{_{(}->}}
\newcommand{\arsurj}{\ar@{->>}}

\newcommand{\set}[1]{\left\{ #1 \right\}}
% $\set{...}$ yields $\left\{ ... \right\}$.
\newcommand{\abs}[1]{\left| #1 \right|}
% $\abs{...}$ yields $\left| ... \right|$.
\newcommand{\tup}[1]{\left( #1 \right)}
% $\tup{...}$ yields $\left( ... \right)$.
\newcommand{\ive}[1]{\left[ #1 \right]}
% $\ive{...}$ yields $\left[ ... \right]$.
\newcommand{\verts}[1]{\operatorname{V}\left( #1 \right)}
% $\verts{...}$ yields $\operatorname{V}\left( ... \right)$.
\newcommand{\edges}[1]{\operatorname{E}\left( #1 \right)}
% $\edges{...}$ yields $\operatorname{E}\left( ... \right)$.
\newcommand{\arcs}[1]{\operatorname{A}\left( #1 \right)}
% $\arcs{...}$ yields $\operatorname{A}\left( ... \right)$.
\newcommand{\underbrack}[2]{\underbrace{#1}_{\substack{#2}}}
% $\underbrack{...1}{...2}$ yields
% $\underbrace{...1}_{\substack{...2}}$. This is useful for doing
% local rewriting transformations on mathematical expressions with
% justifications.

\setbeamertemplate{itemize/enumerate body begin}{\large}
\setbeamertemplate{itemize/enumerate subbody begin}{\large}
\setbeamertemplate{itemize/enumerate subsubbody begin}{\large}

\usepackage{url}%this line and the next are related to hyperlinks
%\usepackage[colorlinks=true, pdfstartview=FitV, linkcolor=blue, citecolor=blue, urlcolor=blur]{hyperref}

\usepackage{color}

%\usetheme{Antibes}
%\usetheme{Bergen}
%\usetheme{Berkeley}
%\usetheme{Berlin}
%\usetheme{Boadilla}
%\usetheme{Copenhagen}
%\usetheme{Darmstadt}
%\usetheme{Dresden}
\usetheme{Frankfurt}
%\usetheme{Goettingen}
%\usetheme{Hannover}
%\usetheme{Ilmenau}
%\usetheme{JuanLesPins}
%\usetheme{Luebeck}
%\usetheme{Madrid}
%\usetheme{Malmoe}
%\usetheme{Marburg}
%\usetheme{Montpellier}
%\usetheme{PaloAlto}
%\usetheme{Pittsburgh}
%\usetheme{Rochester}
%\usetheme{Singapore}
%\usetheme{Szeged}
%\usetheme{Warsaw}

\usefonttheme[onlylarge]{structurebold}
\setbeamerfont*{frametitle}{size=\normalsize,series=\bfseries}
\setbeamertemplate{navigation symbols}{}
\setbeamertemplate{footline}[frame number]
\setbeamertemplate{itemize/enumerate body begin}{}
\setbeamertemplate{itemize/enumerate subbody begin}{\normalsize}
%\setbeamertemplate{section in head/foot shaded}[default][60]
%\setbeamertemplate{subsection in head/foot shaded}[default][60]
\beamersetuncovermixins{\opaqueness<1>{0}}{\opaqueness<2->{15}}

%\usepackage{beamerthemesplit}
\usepackage{epsfig,amsfonts,bbm,mathrsfs}
\usepackage{verbatim} 

% Dark red emphasis
\definecolor{darkred}{rgb}{0.7,0,0} % darkred color
\newcommand{\defn}[1]{{\color{darkred}\emph{#1}}} % emphasis of a definition



\newcommand{\STRUT}{\vrule width 0pt depth 8pt height 0pt}
\newcommand{\ASTRUT}{\vrule width 0pt depth 0pt height 11pt}


\theoremstyle{plain}
\newtheorem{conj}[theorem]{Conjecture}


\setbeamertemplate{headline}{}
%This removes a black stripe from the top of the slides.

\newcommand{\iso}{\cong}
%\newcommand{\qedbox}{\rule{2mm}{2mm}}
%\renewcommand{\qedsymbol}{\qedbox}
\newcommand{\absval}[1]{\left\lvert #1 \right\rvert}
\newcommand{\case}[1]{\vspace{12pt}\noindent\underline{#1}:}
\newcommand{\fs}{\mathcal{S}} % flagged Schur function
\newcommand{\mbf}{\mathbf}
\newcommand{\0}{\phantom{c}}
\newcommand{\swt}[1]{\left\langle #1 \right\rangle} % Spectral weight or amplitude
\renewcommand{\merge}[1]{\vee_{#1}} % merge
\newcommand{\SymGp}[1]{\mathfrak{S}_{#1}} % symmetric group

\DeclareMathOperator{\wt}{wt} % weight
\newcommand{\wtg}{\widetilde{g}}
\newcommand{\sep}{\operatorname{sep}}
\newcommand{\seplist}{\operatorname{seplist}}
\newcommand{\cont}{\operatorname{cont}}
\newcommand{\ircont}{\operatorname{ircont}}
\newcommand{\ceq}{\operatorname{ceq}}
\newcommand{\ttt}{{\mathbf{t}}}

\newcommand*\circled[1]{\tikz[baseline=(char.base)]{
            \node[shape=circle,draw,inner sep=2pt] (char) {#1};}}

\newcommand{\mm}{\mathbf{m}}
\newcommand{\nn}{\mathbf{n}}
\newcommand{\qq}{\mathbf{q}}

\newcommand{\mcA}{\mathcal{A}}
\newcommand{\mcF}{\mathcal{F}}
\newcommand{\mcM}{\mathcal{M}}
\newcommand{\mcW}{\mathcal{W}}
\newcommand{\mcI}{\mathcal{I}}

\newcommand{\bze}{\overline{0}}
\newcommand{\bon}{\overline{1}}
\newcommand{\btw}{\overline{2}}
\newcommand{\bth}{\overline{3}}
\newcommand{\bfo}{\overline{4}}
\newcommand{\bfive}{\overline{5}}
\newcommand{\bsix}{\overline{6}}
\newcommand{\bseven}{\overline{7}}
\newcommand{\beight}{\overline{8}}
\newcommand{\bi}{\overline\imath}
\newcommand{\brr}{\overline{r}}
\newcommand{\bn}{\overline{n}}
\newcommand{\ellbar}{\overline{\ell}}

\newcommand{\fraks}{\mathfrak{s}}

\let\sumnonlimits\sum
\let\prodnonlimits\prod
\let\cupnonlimits\bigcup
\let\capnonlimits\bigcap
\renewcommand{\sum}{\sumnonlimits\limits}
\renewcommand{\prod}{\prodnonlimits\limits}
\renewcommand{\bigcup}{\cupnonlimits\limits}
\renewcommand{\bigcap}{\capnonlimits\limits}

\newcommand{\mlnode}[1]{\node[circle, draw=black] at (#1){\phantom{c}};}

\usepackage{xparse}

\makeatletter
% \specialmergetwolists{<coupler>}{<list1>}{<list2>}{<return macro>}
% \specialmergetwolists*{<coupler>}{<listcmd1>}{<listcmd2>}{<return macro>}
\protected\def\specialmergetwolists{%
  \begingroup
  \@ifstar{\def\cnta{1}\@specialmergetwolists}
    {\def\cnta{0}\@specialmergetwolists}%
}
\def\@specialmergetwolists#1#2#3#4{%
  \def\tempa##1##2{%
    \edef##2{%
      \ifnum\cnta=\@ne\else\expandafter\@firstoftwo\fi
      \unexpanded\expandafter{##1}%
    }%
  }%
  \tempa{#2}\tempb\tempa{#3}\tempa
  \def\cnta{0}\def#4{}%
  \foreach \x in \tempb{%
    \xdef\cnta{\the\numexpr\cnta+1}%
    \gdef\cntb{0}%
    \foreach \y in \tempa{%
      \xdef\cntb{\the\numexpr\cntb+1}%
      \ifnum\cntb=\cnta\relax
        \xdef#4{#4\ifx#4\empty\else,\fi\x#1\y}%
        \breakforeach
      \fi
    }%
  }%
  \endgroup
}
\makeatother

% \theoremstyle{plain}
% \newtheorem{thm}{Theorem}[section]
% \newtheorem{lemma}[thm]{Lemma}
% \newtheorem{conj}[thm]{Conjecture}
% \newtheorem{prop}[thm]{Proposition}
% \newtheorem{cor}[thm]{Corollary}
% \theoremstyle{definition}
% \newtheorem{dfn}[thm]{Definition}
% \newtheorem{example}[thm]{Example}
% \newtheorem{remark}[thm]{Remark}
% \numberwithin{equation}{section}
% %\numberwithin{figure}{section}
% %\numberwithin{table}{section}
% %\setcounter{section}{-1}

% % For breaking equations across multiple pages
% % \allowdisplaybreaks[1]

% \usepackage[colorinlistoftodos]{todonotes}
% \newcommand{\erik}[1]{\todo[size=\tiny,color=green!30]{#1 \\ \hfill --- Erik}}
% \newcommand{\Erik}[1]{\todo[size=\tiny,inline,color=green!30]{#1
      % \\ \hfill --- Erik}}
% \newcommand{\darij}[1]{\todo[size=\tiny,color=red!30]{#1 \\ \hfill --- Darij}}
% \newcommand{\Darij}[1]{\todo[size=\tiny,inline,color=red!30]{#1
      % \\ \hfill --- Darij}}
% \newcommand{\travis}[1]{\todo[size=\tiny,color=blue!30]{#1 \\ \hfill --- Travis}}
% \newcommand{\Travis}[1]{\todo[size=\tiny,inline,color=blue!30]{#1
      % \\ \hfill --- Travis}}

%%%%%%%%%%%%%%%%%%%%%%%%%%%%%%%%%%%%%%%%

\begin{document}
\title[MLQs]{Multiline queues with spectral parameters}

\author{\href{http://www.cip.ifi.lmu.de/~grinberg/}{Darij Grinberg}\\
joint work with Erik Aas and \href{https://sites.google.com/view/tscrim/home}{Travis Scrimshaw}}

\date{15 October 2018 \\ North Carolina State University}

% \keywords{multiline queue, TASEP, R-matrix, symmetric function}
% \subjclass[2010]{
% 60C05,  % Combinatorial probability
% 05A19,  % Combinatorial identities, bijective combinatorics
% 16T25,  % Yang--Baxter equations
% 05E05}  % Symmetric functions

\frame{\titlepage
\textbf{slides: \red \url{http://www.cip.ifi.lmu.de/~grinberg/algebra/ncsu2018.pdf}} \\
\textbf{paper: \red \url{http://www.cip.ifi.lmu.de/~grinberg/algebra/mlqs.pdf}} \\
%\textbf{project: \red \url{https://github.com/darijgr/schur-mlq}}
}

\begin{frame}
\fti{Sites and words}

\begin{itemize}

\item We study a combinatorial algorithm by which \defn{queues} act
      on \defn{words}.  \vspace{0.3pc}

\item Fix a positive integer $n$. \pause

\item For a nonnegative integer $k$, let $\ive{k}$ be the set $\set{1, 2, \ldots, k}$. \pause

\item We shall refer to the elements $1, 2, \ldots, n \in \ZZ / n \ZZ$ as \defn{sites}. \\
Regard them as points on a line that ``wraps around'' cyclically:
\[
{\grey \cdots} \quad {\grey n-1} \quad {\grey n} \quad 1 \quad 2 \quad \cdots \quad n-1 \quad n \quad {\grey 1} \quad {\grey 2} \quad {\grey \cdots}
\]

\pause \item A \defn{word} means a map $\set{\text{sites}} \to \set{\text{positive integers}}$. \\
If $u$ is a word and $i$ is a site, then $u_i := u\tup{i}$.

\pause \item Write $u_1 u_2 \cdots u_n$ for a word $u$ (``one-line notation'').
\pause \\ Example: The word $3 3 1 2 2$ (for $n = 5$) is the map
\[
\begin{array}{cccccccccccccc}
& i & = & {\grey \cdots} & {\grey 4} & {\grey 5} & 1 & 2 & 3 & 4 & 5 & {\grey 1} & {\grey 2} & {\grey \cdots} \\
\mapsto & u_i & = & {\grey \cdots} & {\grey 2} & {\grey 2} & 3 & 3 & 1 & 2 & 2 & {\grey 3} & {\grey 3} & {\grey \cdots}
\end{array}
\]

\end{itemize}
\end{frame}

\begin{frame}
\fti{Queues}
\begin{itemize}

\item A \defn{queue} means a set of sites.

\pause \item Draw a queue $q$ by putting circles on all the sites $i \in q$. 
\pause \\ Example: The queue $\set{2, 5}$ (for $n = 7$) is represented by
\only<1-2>{
\[
{\phantom{\circled{3}} }
\]
}%
\only<3-4>{
\[
{\grey \cdots} \quad {\grey 6} \quad {\grey 7} \quad 1 \quad \circled{2} \quad 3 \quad 4 \quad \circled{5} \quad 6 \quad 7 \quad {\grey 1} \quad {\grey \circled{2}} \quad {\grey \cdots}
\]%
}%
\only<5>{
\[
{\phantom \cdots} \quad {\phantom 6} \quad {\phantom 7} \quad 1 \quad \circled{2} \quad 3 \quad 4 \quad \circled{5} \quad 6 \quad 7 \quad {\phantom 1} \quad {\phantom{\circled{2}}} \quad {\phantom \cdots}
\]
}%

\pause \item We shall omit all the grey parts in the future (i.e., we will draw only one copy of each site).

\end{itemize}
\end{frame}

\begin{frame}
\fti{Action of queues on words, 1: example}
\begin{itemize}

\item Let $q$ be a queue, and $u$ a word. We shall define a word $q(u)$. \pause

\item \textbf{Algorithm:}
\begin{itemize}
\item Draw $u$ on top.
\item Draw $q$ as circles in the middle.
\item Build $q(u)$ letter by letter, as follows...
\end{itemize}

% To compute $q(u)$, draw the following diagram
% (whose upper row shows $u$, whose lower row shows $q(u)$,
% and whose middle row represents the set $q$ by balls in the positions of its elements):
Example: $n = 9$ and $u = 346613321$ and $q = \set{1, 4, 8, 9}$:
\[
\begin{tikzpicture}[>=latex,rounded corners,yscale=1.2,xscale=1,baseline=0]
\def\passwidth{3pt};
\node (i1) at (1,1) {$3$};
\node (i2) at (2,1) {$4$};
\node (i3) at (3,1) {$6$};
\node (i4) at (4,1) {$6$};
\node (i5) at (5,1) {$1$};
\node (i6) at (6,1) {$3$};
\node (i7) at (7,1) {$3$};
\node (i8) at (8,1) {$2$};
\node (i9) at (9,1) {$1$};
\node (tphan) at (10, -1) {\phantom{x}};
\node (tphan2) at (0, -1) {\phantom{x}};
\only<10->{\node (t1) at (1,-1) {$2$};}
\only<4->{\node (t2) at (2,-1) {$7$};}
\only<3->{\node (t3) at (3,-1) {$7$};}
\only<11->{\node (t4) at (4,-1) {$3$};}
\only<7->{\node (t5) at (5,-1) {$4$};}
\only<6->{\node (t6) at (6,-1) {$4$};}
\only<5->{\node (t7) at (7,-1) {$5$};}
\only<8->{\node (t8) at (8,-1) {$1$};}
\only<9->{\node (t9) at (9,-1) {$1$};}
\node[circle,draw=black] (q1) at (1,0) {};
\node[circle,draw=black] (q2) at (4,0) {};
\node[circle,draw=black] (q3) at (8,0) {};
\node[circle,draw=black] (q4) at (9,0) {};
\only<4->{\draw[->,red] (i4) -- (4,0.5) .. controls (3.8,0.2) and (3.5,0) .. (3.1,0) -- (2,0) -- (t2);}
\only<3->{\draw[->,red] (i3) -- (t3);}
\only<5->{\draw[->,red] (i2) -- (2,0.15) .. controls (1.8,-0.2) and (1.6,-0.25) .. (1.3,-0.25) -- (0,-0.25);
\draw[>->,red] (10,-0.25) -- (8,-0.25) -- (7,-0.25) -- (t7);}
\only<6->{\draw[->,red] (i6) -- (t6);}
\only<7->{\draw[->,red] (i7) -- (7,0) -- (5,0) -- (t5);}
\only<8->{\draw[white,line width=\passwidth] (5,0.28) -- (7.3,0.28) .. controls (7.7,0.28) and (7.85,0.12) .. (q3);  % To simulate underpass
\draw[->,blue] (i5) -- (5,0.28) -- (7.3,0.28) .. controls (7.7,0.28) and (7.85,0.12) .. (q3);
\draw[white,line width=\passwidth] (q3) -- (t8);  % To simulate underpass
\draw[->,blue] (q3) -- (t8);}
\only<9->{\draw[->,blue] (i9) -- (q4);
\draw[white,line width=\passwidth] (q4) -- (t9);  % To simulate underpass
\draw[->,blue] (q4) -- (t9);}
\only<11->{\draw[white,line width=\passwidth] (1,0.28) -- (3.3,0.28) .. controls (3.7,0.28) and (3.85,0.12) .. (q2);  % To simulate underpass
\draw[->,blue] (i1) -- (1,0.28) -- (3.3,0.28) .. controls (3.7,0.28) and (3.85,0.12) .. (q2);
\draw[->,blue] (q2) -- (t4);}
\only<10->{\draw[white,line width=\passwidth] (i8) -- (8,0.28) -- (10,0.28);  % To simulate underpass
\draw[->,blue] (i8) -- (8,0.28) -- (10,0.28);
\draw[>->,blue] (0,0.28) -- (0.3,0.28) .. controls (0.7,0.28) and (0.85,0.12) .. (q1);
\draw[white,line width=\passwidth] (q1) -- (t1);  % To simulate underpass
\draw[->,blue] (q1) -- (t1);}
\end{tikzpicture}
\only<1-11>{\vspace{-0.25pc}}
\]
\only<3-7>{%
\textbf{Phase I:}
For each of the \textbf{largest} $n - \abs{q}$ letters of $u$ (in \textbf{decreasing} order),
\begin{itemize}
\item \textbf{drop} this letter \textbf{down} and \textbf{add} $1$ to it;
\item \textbf{move} it left until hitting some unoccupied site $i \notin q$;
\item \textbf{place} it there.
\end{itemize}
}%
\only<8-11>{%
\textbf{Phase II:}
For each of the \textbf{smallest} $\abs{q}$ letters of $u$ (in \textbf{increasing} order),
\begin{itemize}
\item \textbf{drop} this letter down;
\item \textbf{move} it \textbf{right} until hitting some unoccupied site $i \in q$;
\item \textbf{place} it there.
\end{itemize}
}%
\only<12->{\vspace{0.45pc}%
The letters on the bottom now form $q(u)$. \\
}
\only<13->{%
\textbf{Proposition.} Equal letters can be processed in any order.
}

\end{itemize}
\vspace{10cm}
\end{frame}

\begin{frame}
\fti{Action of queues on words, 2: formal definition}
\begin{itemize}

\item Let $q$ be a queue, and $u$ a word. Define a word $q(u)$ as follows:

In the beginning, $v = q(u)$ is a word whose letters are unset. \\
Choose a permutation $\tup{i_1, i_2, \ldots, i_n}$ of $\tup{1, 2, \ldots, n}$
such that $u_{i_1} \leq u_{i_2} \leq \cdots \leq u_{i_n}$.

\begin{description}
\item[Phase I.]
  For $i = i_n, i_{n-1}, \ldots, i_{\abs{q}+1}$, do the following: \\
    Find the first site $j$ weakly to the left (cyclically) of $i$ such that $j \notin q$ and $v_j$ is not set.
    Then set $v_j = u_i + 1$.

\item[Phase II.]
  For $i = i_1, i_2, \ldots, i_{\abs{q}}$, do the following: \\
    Find the first site $j$ weakly to the right (cyclically) of $i$ such that $j \in q$ and $v_j$ is not set.
    Then set $v_j = u_i$.
\end{description}
\pause

\item \textbf{Proposition.}
\begin{itemize}
\item The resulting word $v = q(u)$ does not depend on the choice of permutation $(i_1, i_2, \dotsc, i_n)$.
\item Phase I and Phase II can be done in parallel. % (as long as in-phase order is preserved).
\end{itemize}

\end{itemize}
\vspace{10cm}
\end{frame}

\begin{frame}
\fti{Remark on TASEP connection}
\begin{itemize}

\item This action of queues on words is the
      ``generalized Ferrari--Martin algorithm'' of
      Arita, Ayyer, Mallick and Prolhac ({\red \arxiv{1104.3752}},
      J. Phys. A, 2011), extending a simpler procedure by
      Ferrari and Martin
      ({\red \arxiv{math-ph/0509045}}).
      % inspired by the ``discrete MLQs''
      % of Aas and Linusson ({\red \arxiv{1501.04417}}, now in Ann. Inst. Henri Poincaré D). \\
      % Main difference: They have no Phase I, but their words have
      % empty positions. \pause
      % \\ (Our picture subsumes theirs -- fill the empty positions
      % with high letters.) \pause
      \pause

\item Their motivation: compute stationary distribution of
      multi-species TASEP (totally asymmetric simple exclusion process) on a circle. \\
      The algorithm intertwines different TASEPs, and lets one transport
      the stationary distribution from one to another.

\pause
\item Aas and Linusson ({\red \arxiv{1501.04417}}, Ann. Inst. Henri Poincaré D)
      later attempted to obtain explicit formulas for steady state
      probabilities. \\
      Our work proves two of their conjectures.

\end{itemize}
%\vspace{10cm}
\end{frame}

\begin{frame}
\fti{Types of words}
\begin{itemize}

\item The \defn{type} of a word $u$ is the sequence $\mm = (m_1, m_2, \ldots)$,
      where $m_k = \tup{ \# \text{ of all sites } i \text{ such that } u_i = k }$.
      \pause \\
      Example: The word $1255135$ has type
      \only<1-3>{$\tup{2, 1, 1, 0, 3, 0, 0, 0, \ldots}$.}%
      \only<4->{$\tup{2, 1, 1, 0, 3}$.}

\pause \vspace{0.45pc}
\item We omit trailing zeroes from infinite sequences. \\
      That is, we abbreviate $\tup{m_1, m_2, \ldots, m_k, 0, 0, 0, \ldots}$
      as $\tup{m_1, m_2, \ldots, m_k}$.

\pause\pause \vspace{0.45pc}
\item A word $u$ is \defn{packed with $\ell$ classes} if
      its type $\mm$ has $m_1, m_2, \ldots, m_\ell > 0$ and
      $m_{\ell+1} = m_{\ell+2} = \cdots = 0$.
      \pause \\ \vspace{0.3pc}
      Example: The word $1255135$ is not packed.
      \\ The word $1244134$ is packed with $4$ classes.

% \pause
% \item If $u$ is a word of type $\mm = (m_1, m_2, \ldots)$, then we set
      % $p_i(\mm) := m_1 + m_2 + \cdots + m_i$ for each $i$.

% \pause
% \item Instead of the type $\mm$ of a word $u$, we can also store the
      % \textbf{multiset}
      % $P(u) := \set{p_1(\mm), p_2(\mm), p_3(\mm), \ldots}$
      % (which contains $n$ infinitely often, but is otherwise finite).
      % \\ \pause
      % Example:
      % $P(1255135) = \set{2, 3, 4, 4, 7, 7, 7, \ldots}$.

% \pause
% \item If $q$ is a queue of size $k$, and if $u$ is a word, then
      % $P(q(u))$ is obtained by inserting a $k$ into $P(u)$.

\end{itemize}
%\vspace{10cm}
\end{frame}

\begin{frame}
\fti{MLQs}
\begin{itemize}

\item A \defn{MLQ} (short for ``multiline queue'') is a tuple of queues. \pause \vspace{0.35pc}

\item If $\qq = \tup{q_1, q_2, \ldots, q_k}$ is an MLQ, and $u$ is a word,
      then
      \[
      \qq(u) := q_k \tup{ q_{k-1} \tup{ \cdots \tup{ q_1(u) } } } .
      \]

\pause
\item Let $\ell > 0$, and let $\sigma$ be a permutation of $\ive{\ell-1}$. \\
      Let $\mm = (m_1, m_2, \ldots, m_{\ell})$ be a sequence of positive integers. \\
      A \defn{$\sigma$-twisted MLQ of type $\mm$}
      means an MLQ $\qq = \tup{q_1, q_2, \ldots, q_{\ell-1}}$
      such that
      \begin{align*}
      \abs{q_i} &= m_1 + m_2 + \cdots + m_{\sigma(i)} \text{ for all }i ,
      \qquad \text{ and } \\
      n &= m_1 + m_2 + \cdots .
      \end{align*}
      \only<4-8>{Example:
      $n = 6$ and $\mm = \tup{2, 3, 1}$ and $\ell = 3$ and $\sigma = \tup{2,1}$
      (one-line notation). Then, a $\sigma$-twisted MLQ of type $\mm$ is
      an MLQ $\qq = \tup{q_1, q_2}$ with $\abs{q_1} = m_1 + m_2 = 2+3 = 5$
      and $\abs{q_2} = m_1 = 2$.
      }
      \only<5-6>{
      For example, $\qq = \tup{\set{1, 3, 4, 5, 6}, \set{2, 3}}$
      }%
      \only<6>{ and
      $\qq(1 1 1 1 1 1) = 3 1 1 2 2 2$.
      }%
      \only<7-8>{
      For example, $\qq = \tup{\set{1, 3, 4, 5, 6}, \set{4, 5}}$
      }
      \only<8>{ and
      $\qq(1 1 1 1 1 1) = 2 3 2 1 1 2$.
      }

\only<9>{
\item Equivalently: A \defn{$\sigma$-twisted MLQ of type $\mm$}
      can be defined as an MLQ $\qq = \tup{q_1, q_2, \ldots, q_{\ell-1}}$ such that
      \begin{itemize}
      \item the word $\qq(1\dotsm 1)$ has type $\mm$
      (where $1\dotsm 1$ is the word whose entries all equal $1$);
      \item we have
      $0 < \abs{q_{\sigma^{-1}(1)}} < \abs{q_{\sigma^{-1}(2)}} < \cdots < \abs{q_{\sigma^{-1}(\ell-1)}}$.
      \end{itemize}
}

% \only<10>{
% \item When $\sigma$ is the identity permutation,
      % a $\sigma$-twisted MLQ of type $\mm$ is just called
      % an \defn{(ordinary) MLQ of type $\mm$}.}

\end{itemize}
\vspace{10cm}
\end{frame}

\begin{frame}
\fti{Generating functions, 1: definition}

\begin{itemize}
\item Now, let $x_1, x_2, \ldots, x_n$
be commuting variables. \pause

\item For any $\ell\geq1$, any permutation $\sigma$ of $\left[  \ell-1\right]
$, and any packed word $u$ of type $\mathbf{m}$ with $\ell$ classes, we
define the \defn{$\sigma$-spectral weight $\left\langle u\right\rangle
_{\sigma}$} by
\[
\left\langle u\right\rangle _{\sigma}:=\sum_{\substack{\mathbf{q}\text{ is a
}\sigma\text{-twisted}\\\text{MLQ of type }\mathbf{m}\\\text{satisfying
}u=\mathbf{q}\left(  1\cdots1\right)  }}\operatorname*{wt}\mathbf{q}.
\]
Here:

\begin{itemize}
\item $1\cdots1$ denotes the word whose all entries are $1$.

\item $\operatorname*{wt}\mathbf{q}:=\prod_{p=1}^{k}\prod_{i\in q_{p}}x_{i}$
for any MLQ $\mathbf{q}=\left(  q_{1},q_{2},\ldots,q_{k}\right)  $.
\end{itemize}
\pause

\only<3>{
Example: Recall that $\left(  \left\{  1,3,4,5,6\right\}  ,\left\{
4,5\right\}  \right)  $ is a $\sigma$-twisted MLQ of type $\mathbf{m}$ for
{$n=6$ and $\mathbf{m}=\left(  2,3,1\right)  $ and $\ell=3$ and $\sigma
=\left(  2,1\right)  $ (one-line notation) satisfying $\mathbf{q}%
(111111)= 2 3 2 1 1 2$. It contributes a monomial}%
\[
\left(  x_{1}x_{3}x_{4}x_{5}x_{6}\right)  \left(  x_{4}x_{5}\right)
=x_{1}x_{3}x_{4}^{2}x_{5}^{2}x_{6}
\quad \text{to }
\left\langle {2 3 2 1 1 2}\right\rangle _{\sigma} .
\]
}

\only<4->{
\item Set $\swt{u} := \swt{u}_{\id}$
for the permutation $\id$ of $\ive{\ell-1}$.
}

\end{itemize}
\vspace{10cm}
\end{frame}

\begin{frame}
\fti{Generating functions, 2: more examples}

\begin{itemize}

\item Example: For $n = 5$, $\ell = 5$ and $\mm = \tup{1,1,2,1}$, we have
\begin{align*}
\swt{13234} & = x_1 x_2 x_3^2 x_4 (x_1^2 + x_1 x_4 + x_1 x_5 + x_4 x_5 + x_5^2).
\end{align*}

\pause
\item Examples: For $n = 5$, $\ell = 5$ and $\mm = \tup{1,1,1,1,1}$, we have
\begin{align*}
\swt{13245} & = x_1 x_2 x_3^2 x_4 (x_1^2 + x_1x_4 + x_1x_5 + x_4^2 + x_4x_5 + x_5^2)
\\ & \hspace{20pt} \cdot (x_1x_2x_3 + x_1x_2x_5+x_1x_3x_5+x_2x_3x_5),
\\ \swt{14235} & = x_1x_2x_3^2x_4^2 (x_1^3x_2 + x_1^3x_3 + x_1^3x_5 + x_1^2x_2x_3 + x_1^2x_2x_4
\\ & \hspace{60pt} + 2x_1^2x_2x_5 + x_1^2x_3x_4 + 2x_1^2x_3x_5 + x_1^2x_4x_5
\\ & \hspace{60pt}  + x_1^2x_5^2 + x_1x_2x_3x_5+ x_1x_2x_4x_5 + 2x_1x_2x_5^2
\\ & \hspace{60pt}  + x_1x_3x_4x_5 + 2x_1x_3x_5^2 + x_1x_4x_5^2 + x_1x_5^3 
\\ & \hspace{60pt} + x_2x_3x_5^2 + x_2x_4x_5^2 + x_2x_5^3 + x_3x_4x_5^2 + x_3x_5^3).
\end{align*}

\end{itemize}

\vspace{10cm}
\end{frame}

\begin{frame}
\fti{The symmetry theorem, 1: statement}

\begin{itemize}
\item \textbf{Theorem.} For any $\ell\geq1$, any permutation $\sigma$ of
$\left[  \ell-1\right]  $, and any packed word $u$ of type $\mathbf{m}$ with
$\ell$ classes, we have
\[
\left\langle u\right\rangle _{\sigma}=\left\langle u\right\rangle .
\]

\pause
\item This yields the ``commutativity conjecture'' by Arita, Ayyer, Mallick
and Prolhac on the TASEP ({\red \arxiv{1104.3752}}).

\pause
\item This is proven bijectively, using a \textquotedblleft duality
transformation\textquotedblright\ on MLQs that leaves their action on words unchanged.

\item \textbf{Main lemma.} If $q_{1}$ and $q_{2}$ are two queues, then there
are two queues $q_{1}^{\prime}$ and $q_{2}^{\prime}$ satisfying
\begin{align*}
&\left\vert q_{1}^{\prime}\right\vert =\left\vert q_{2}\right\vert
\qquad\text{and}\qquad\left\vert q_{2}^{\prime}\right\vert =\left\vert
q_{1}\right\vert \qquad \text{and} \\
& \tup{\prod_{i \in q'_1} x_i} \tup{\prod_{i \in q'_2} x_i}
  = \tup{\prod_{i \in q_1} x_i} \tup{\prod_{i \in q_2} x_i}
\end{align*}
such that every word $u$ satisfies%
\[
q_{1}^{\prime}\left(  q_{2}^{\prime}\left(  u\right)  \right)  =q_{1}\left(
q_{2}\left(  u\right)  \right)  .
\]

\end{itemize}
\vspace{10cm}
\end{frame}

\begin{frame}
\fti{The symmetry theorem, 2: idea of proof}

\begin{itemize}
\item The construction of $q_{1}^{\prime}$ and $q_{2}^{\prime}$ is combinatorial:

\begin{itemize}
\item Encode the pair $\left(  q_{1},q_{2}\right)  $ as a $2n$-letter word
$b = \tup{b_1, b_2, \ldots, b_{2n}}$
over the $3$-letter alphabet $\set{{\blue )}, {\blue (}, {\blue \circ}}$. Namely, for each $i$,

\begin{itemize}
\item let $b_{2i-1}$ be an opening parenthesis
\textquotedblleft${\blue (}$\textquotedblright\ if $i\in q_{1}$, otherwise a neutral
symbol \textquotedblleft${\blue \circ}$\textquotedblright;

\item let $b_{2i}$ be a closing parenthesis \textquotedblleft%
${\blue )}$\textquotedblright\ if $i\in q_{2}$, otherwise a neutral symbol
\textquotedblleft${\blue \circ}$\textquotedblright.
\end{itemize}

\only<2-5>{Convenient example:
\begin{align*}
n &= 10; \\
q_1 &= \set{2, 6, 7, 9}; \\
q_2 &= \set{1, 3, 5, 7, 8}.
\end{align*}
}%
\only<3>{%
Then,
\[
\hspace{-1cm}\begin{tabular}{cc|c|c|c|c|c|c|c|c|c|c|}
$ b $ & $ = $ & $ {\blue \circ)} $ & $ {\blue (\circ} $ & $ {\blue \circ)} $ & $ {\blue \circ\circ} $ & $ {\blue \circ)} $ & $ {\blue (\circ} $ & $ {\blue ()} $ & $ {\blue \circ)} $ & $ {\blue (\circ} $ & $ {\blue \circ\circ} $ \\
$i$ & = & 1 & 2 & 3 & 4 & 5 & 6 & 7 & 8 & 9 & 10
\end{tabular}
\]
}%
\only<4>{%
Then,
\[
\hspace{-1cm}\begin{tabular}{cccccccccccc}
$ b $ & $ = $ & $ {\blue \circ)} $ & $ {\blue (\circ} $ & $ {\blue \circ)} $ & $ {\blue \circ\circ} $ & $ {\blue \circ)} $ & $ {\blue (\circ} $ & $ {\blue ()} $ & $ {\blue \circ)} $ & $ {\blue (\circ} $ & $ {\blue \circ\circ} $
\end{tabular}
\]
}%
\only<5>{%
Then,
\[
 b    =    {\blue \circ)}    {\blue (\circ}    {\blue \circ)}    {\blue \circ\circ}    {\blue \circ)}    {\blue (\circ}    {\blue ()}    {\blue \circ)}    {\blue (\circ}    {\blue \circ\circ} 
\]
}

\pause\pause\pause\pause\pause
\item Match parentheses in $b$
\textquotedblleft the usual way\textquotedblright\ but
keeping in mind that the word wraps around cyclically.
\only<6-10>{%
In our above example:
}
\only<6>{
\[
 b    =    {\blue \circ)}    {\blue (\circ}    {\blue \circ)}    {\blue \circ\circ}    {\blue \circ)}    {\blue (\circ}    {\blue ()}    {\blue \circ)}    {\blue (\circ}    {\blue \circ\circ} 
\]
}
\only<7>{
\[
 b    =    {\blue \circ)}    {\blue (_1\circ}    {\blue \circ)_1}    {\blue \circ\circ}    {\blue \circ)}    {\blue (\circ}    {\blue ()}    {\blue \circ)}    {\blue (\circ}    {\blue \circ\circ} 
\]
}
\only<8>{
\[
 b    =    {\blue \circ)}    {\blue (_1\circ}    {\blue \circ)_1}    {\blue \circ\circ}    {\blue \circ)}    {\blue (\circ}    {\blue (_2)_2}    {\blue \circ)}    {\blue (\circ}    {\blue \circ\circ} 
\]
}
\only<9>{
\[
 b    =    {\blue \circ)}    {\blue (_1\circ}    {\blue \circ)_1}    {\blue \circ\circ}    {\blue \circ)}    {\blue (_3\circ}    {\blue (_2)_2}    {\blue \circ)_3}    {\blue (\circ}    {\blue \circ\circ} 
\]
}
\only<10>{
\[
 b    =    {\blue \circ)_4}    {\blue (_1\circ}    {\blue \circ)_1}    {\blue \circ\circ}    {\blue \circ)}    {\blue (_3\circ}    {\blue (_2)_2}    {\blue \circ)_3}    {\blue (_4\circ}    {\blue \circ\circ} 
\]
}

\pause\pause\pause\pause\pause
\item Replace the unmatched parentheses by their duals -- e.g., if they
were ${\blue )}$'s, make them ${\blue (}$'s.
\only<12-14>{\\
In our above example:}
\only<12>{
\[
 b\phantom{'}    =    {\blue \circ)_4}    {\blue (_1\circ}    {\blue \circ)_1}    {\blue \circ\circ}    {\blue \circ)}    {\blue (_3\circ}    {\blue (_2)_2}    {\blue \circ)_3}    {\blue (_4\circ}    {\blue \circ\circ} 
\]
}
\only<13>{
\[
 b\phantom{'}   =    {\blue \circ)_4}    {\blue (_1\circ}    {\blue \circ)_1}    {\blue \circ\circ}    {\blue \circ} {\red)}   {\blue (_3\circ}    {\blue (_2)_2}    {\blue \circ)_3}    {\blue (_4\circ}    {\blue \circ\circ} 
\]
}
\only<14>{
\[
 b'   =    {\blue \circ)_4}    {\blue (_1\circ}    {\blue \circ)_1}    {\blue \circ\circ}    {\blue \circ} {\red(}    {\blue (_3\circ}    {\blue (_2)_2}    {\blue \circ)_3}    {\blue (_4\circ}    {\blue \circ\circ} 
\]
}

\pause\pause\pause\pause
\item Turn the resulting word $b'$ into two sets $q_{1}^{\prime}$ and
$q_{2}^{\prime}$ as follows:

\begin{itemize}
\item $q_{1}^{\prime}=\left\{  i\in\left[  n\right]  \ \mid\ \text{either }
b'_{2i-1} \text{ or } b'_{2i} \text{ is a
\textquotedblleft}{\blue (}\text{\textquotedblright}\right\}  $;

\item $q_{2}^{\prime}=\left\{  i\in\left[  n\right]  \ \mid\ \text{either }
b'_{2i-1} \text{ or } b'_{2i} \text{ is a
\textquotedblleft}{\blue )}\text{\textquotedblright}\right\}  $.
\end{itemize}

\end{itemize}

\end{itemize}
\vspace{10cm}
\end{frame}

\begin{frame}
\fti{The symmetry theorem, 3: comments}

\begin{itemize}

\item Note that
      \begin{itemize}
      \item if $\abs{q_1} < \abs{q_2}$, then $q'_1$ is obtained from $q_1$
            by adding some elements from $q_2$, whereas $q'_2$ is obtained
            from $q_2$ by removing these elements;
      \item if $\abs{q_1} = \abs{q_2}$, then $q'_1 = q_1$ and $q'_2 = q_2$;
      \item if $\abs{q_1} > \abs{q_2}$, then $q'_1$ is obtained from $q_1$
            by removing some elements, whereas $q'_2$ is obtained
            from $q_2$ by adding these elements.
      \end{itemize}
      \vspace{0.25pc}

\pause
\item This is closely connected to the Lascoux-Sch\"{u}tzenberger action of the
symmetric group on words
(a.k.a. the Weyl group action on the word crystal of type A).
\vspace{0.25pc}

\item Note: the map $\left(  q_{1},q_{2}\right)  \mapsto\left(  q_{1}^{\prime
},q_{2}^{\prime}\right)  $ is an involution. \pause
\vspace{0.25pc}

\item Actually, it is a known object from crystal base theory: the
      combinatorial R-matrix for two single columns.
\end{itemize}
\vspace{10cm}
\end{frame}

\begin{frame}
\fti{A Jacobi-Trudi-like formula}

\begin{itemize}

\only<1-2>{
\item But can we compute $\swt{u}$ without enumerating all MLQs?}

\only<2>{
\item We have a partial answer (which subsumes two conjectures by
Aas and Linusson).
}

\pause\pause
\item \textbf{Theorem.}
  Let $B = \set{b_1 < b_2 < \cdots < b_r} \subseteq \ive{n}$. \\
  Let $v_1v_2 \dotsm v_r$ be a weakly decreasing (non-cyclic) packed word of length $r$ with $\ell-1$ classes. \\
  Define a word $u$ of length $n$ by $u_i = v_j$ if $i = b_j$ for some $j$, otherwise $u_i = \ell$. \\
  Then
  \[
  \swt{u} = \left( \prod_{i\in B} x_i \right) \det\bigl( h_{i-j-1+\ell-v_j}(x_1, x_2, \dotsc, x_{b_j}) \bigr)_{1\leq i,j\leq r} .
  \]
  \pause
  Example: $n = 8$ and $r = 4$ and $B = \set{1 < 3 < 4 < 7}$ and $\ell=4$
  and $v_1 v_2 \dotsm v_r = 3321$. Then,
  \begin{align*}
  \swt{{\blue 3} {\red 4} {\blue 3 2} {\red 4} {\red 4} {\blue 1}}
  &=
  \tup{x_1 x_3 x_4 x_7} \\
  &\qquad
  \scalebox{0.8}{$
  \abs{
  \begin{matrix}
  h_0(x_1) & h_{-1}(x_1, x_2, x_3) & h_{-1}(x_1, x_2, x_3, x_4) & h_{-1}(x_1, x_2, \ldots, x_7) \\
  h_1(x_1) & h_{0}(x_1, x_2, x_3) & h_{0}(x_1, x_2, x_3, x_4) & h_{0}(x_1, x_2, \ldots, x_7) \\
  h_2(x_1) & h_{1}(x_1, x_2, x_3) & h_{1}(x_1, x_2, x_3, x_4) & h_{1}(x_1, x_2, \ldots, x_7) \\
  h_3(x_1) & h_{2}(x_1, x_2, x_3) & h_{2}(x_1, x_2, x_3, x_4) & h_{2}(x_1, x_2, \ldots, x_7)
  \end{matrix}
  }
  $}
  \end{align*}

\end{itemize}
\vspace{10cm}
\end{frame}

\begin{frame}
\fti{Bonus problem}
\begin{center}
{\LARGE \bf Bonus problem} \\
\noindent\rule[0.5ex]{\linewidth}{1pt}
{\Large \bf Dual stable Grothendieck polynomials}
\end{center}
\vspace{1cm}
\end{frame}

\begin{frame}
\fti{Reminder on Schur functions}

\begin{itemize}

\item The following is not related to MLQs (or is it?), but a conjecture
      I'm very curious to hear ideas about. \\
      (And it's a Jacobi-Trudi type formula, too.)

\pause \item Fix a commutative ring $\kk$. \\
      Recall that for any skew partition $\lm$, the
      \defn{(skew) Schur function $s_{\lm}$} is defined as the power series
      \[
      \sum_{T \text{ is an SST of shape } \lambda / \mu} \xx^{\cont T} \in \kk\left[\left[x_1, x_2, x_3, \ldots\right]\right] ,
      \]
      where ``SST'' is short for ``semistandard Young tableau'', and where
        \[
        \xx^{\cont T}
        = \prod_{k \geq 1} x_k^{\text{number of times } T \text{ contains entry } k} .
        \]

\pause \item Let us generalize this by extending the sum and introducing extra parameters.

\end{itemize}
\end{frame}

\begin{frame}
\fti{Dual stable Grothendieck polynomials, 1: RPPs}

\begin{itemize}

\item A \defn{reverse plane partition (RPP)} is defined like an SST
      (semistandard Young tableau),
      but entries increase \textbf{weakly} both along rows and down columns. For example,
        \begin{ytableau}
        \none & 1 & 2 & 2 \\
        \none & 2 & 2 \\
        2 & 4
        \end{ytableau}
        is an RPP.
        \pause \\
        (In detail: An RPP is a map $T$ from a skew Young diagram to
        $\set{\text{positive integers}}$ such that
        \[
        T\tup{i, j} \leq T\tup{i, j+1} \text{ and }
        T\tup{i, j} \leq T\tup{i+1, j}
        \]
        whenever
        these are defined.)

\pause
\item Let $\kk$ be a commutative ring, and fix any elements
      $t_1, t_2, t_3, \ldots \in \kk$.

\end{itemize}
\vspace{10cm}
\end{frame}

\begin{frame}
\fti{Dual stable Grothendieck polynomials, 2: definition}

\begin{itemize}

\item Given a skew partition $\lambda / \mu$, we define the \defn{refined dual stable Grothendieck polynomial $\wtg_{\lambda / \mu}$} to be the formal power series
\[
\sum_{T \text{ is an RPP of shape } \lambda / \mu} \xx^{\ircont T} \ttt^{\ceq T} \in \kk\left[\left[x_1, x_2, x_3, \ldots\right]\right] ,
\]
where
\[
\xx^{\ircont T}
= \prod_{k \geq 1} x_k^{\text{number of columns of } T \text{ containing entry } k}
\]
and
\[
\ttt^{\ceq T}
= \prod_{i \geq 1} t_i^{\text{number of } j \text{ such that } T\left(i, j\right) = T\left(i+1, j\right)}
\]
(where $T\left(i, j\right) = T\left(i+1, j\right)$ implies, in particular, that both $\left(i, j\right)$ and $\left(i+1, j\right)$ are cells of $T$).
\newline
This is a formal power series in $x_1, x_2, x_3, \ldots$ (despite the name ``polynomial'').

\end{itemize}
\vspace{10cm}
\end{frame}

\begin{frame}
\fti{Dual stable Grothendieck polynomials, 3: examples on $\xx^{\ircont T}$}

%\textbf{Examples on $\xx^{\ircont T}$:}

\begin{itemize}

\item Recall:
\[
\xx^{\ircont T}
= \prod_{k \geq 1} x_k^{\text{number of columns of } T \text{ containing entry } k} .
\]

\item If $T = \begin{ytableau}
\none & 1 & 2 & 2 \\
\none & 2 & 2 \\
2 & 3
\end{ytableau}$, then $\xx^{\ircont T} = x_1 x_2^4 x_3$.
The $x_2$ has exponent $4$, not $5$, because the two $2$'s in column $3$ count only once.

\pause
\item If $T$ is an SST, then $\xx^{\ircont T} = \xx^{\cont T}$.

\end{itemize}
\end{frame}

\begin{frame}
\fti{Dual stable Grothendieck polynomials, 3: examples on $\ttt^{\ceq T}$}

%\textbf{Examples on $\ttt^{\ceq T}$:}

\begin{itemize}

\item Recall that
\[
\ttt^{\ceq T}
= \prod_{i \geq 1} t_i^{\text{number of } j \text{ such that } T\left(i, j\right) = T\left(i+1, j\right)}
\]

\item If $T = \begin{ytableau}
\none & 1 & 2 & 2 \\
\none & 2 & 2 \\
2 & 3
\end{ytableau}$, then $\ttt^{\ceq T} = t_1$, due to $T\left(1, 3\right) = T\left(2, 3\right)$.

\pause
\item If $T$ is an SST, then $\ttt^{\ceq T} = 1$.

\item In general, $\ttt^{\ceq T}$ measures ``how often'' $T$ breaks the SST condition.

\end{itemize}
\end{frame}

\begin{frame}
\fti{Dual stable Grothendieck polynomials, 5}

\begin{itemize}

\item If we set $t_1 = t_2 = t_3 = \cdots = 0$, then $\wtg_{\lm} = s_{\lm}$.

\pause
\item If we set $t_1 = t_2 = t_3 = \cdots = 1$, then $\wtg_{\lm} = g_{\lm}$, the \defn{dual stable Grothendieck polynomial} of Lam and Pylyavskyy ({\red \arxiv{0705.2189}}).

\item The general case, to our knowledge, is new.

\pause
\item \textbf{Theorem (Galashin, G., Liu, {\red \arxiv{1509.03803}}):} The power series $\wtg_{\lm}$ is symmetric in the $x_i$ (not in the $t_i$).

\pause
\item \textbf{Example 1:} If $\lambda = \left(n\right)$ and $\mu = \left(\right)$, then $\wtg_{\lm} = h_n$, the $n$-th complete homogeneous symmetric function.

\pause
\item \textbf{Example 2:} If $\lambda = \left(\underbrace{1, 1, \ldots, 1}_{n \text{ ones}}\right)$ and $\mu = \left(\right)$, then $\wtg_{\lm} = e_n\left(t_1, t_2, \ldots, t_{n-1}, x_1, x_2, x_3, \ldots\right)$, where $e_n$ is the $n$-th elementary symmetric function.

\pause
\item \textbf{Example 3:} If $\lambda = \left(2,1\right)$ and $\mu = \left(\right)$, then $\wtg_{\lm} = \sum\limits_{a\leq b; \ a < c} x_a x_b x_c + t_1 \sum\limits_{a \leq b} x_a x_b = s_{(2,1)} + t_1 s_{(2)}$.

\end{itemize}
\end{frame}

\begin{frame}
\fti{Jacobi-Trudi identity?}

\begin{itemize}

\item \textbf{Conjecture:}
Let the conjugate partitions of $\lambda$ and
$\mu$ be $\lambda^{t}=\left(  \left(  \lambda^{t}\right)  _{1},\left(
\lambda^{t}\right)  _{2},\ldots,\left(  \lambda^{t}\right)  _{N}\right)  $ and
$\mu^{t}=\left(  \left(  \mu^{t}\right)  _{1},\left(  \mu^{t}\right)
_{2},\ldots,\left(  \mu^{t}\right)  _{N}\right)  $. Then,%
\begin{align*}
&  \widetilde{g}_{\lambda/\mu}\\
&  =\det\left(  \left(  e_{\left(  \lambda^{t}\right)  _{i}-i-\left(  \mu
^{t}\right)  _{j}+j}\left(  \mathbf{x},\mathbf{t}\left[  \left(  \mu
^{t}\right)  _{j}+1:\left(  \lambda^{t}\right)  _{i}\right]  \right)  \right)
_{1\leq i\leq N,\ 1\leq j\leq N}\right)  .
\end{align*}
Here, $\left(  \mathbf{x},\mathbf{t}\left[  k:\ell\right]  \right)  $ denotes
the alphabet $\left(  x_{1},x_{2},x_{3},\ldots,t_{k},t_{k+1},\ldots,t_{\ell
-1}\right)  $.

\textbf{Warning:} If $\ell\leq k$, then $t_{k},t_{k+1},\ldots,t_{\ell-1}$
means nothing. No \textquotedblleft antimatter\textquotedblright\ variables!

\pause
\item This would generalize the Jacobi-Trudi identity for Schur
      functions in terms of $e_i$'s.

\pause
\item I have some even stronger conjectures, with less evidence...

\pause
\item The case $\mu = \varnothing$ has been proven by Damir Yeliussizov
      in {\red \arxiv{1601.01581}}.

\end{itemize}

\end{frame}

\begin{frame}
\fti{Thank you}

\begin{itemize}

\item Ricky Liu for the invitation.

\item Erik Aas and Travis Scrimshaw for collaboration.

\item you for attending.

\end{itemize}

\end{frame}

\end{document}

